\begin{center} 
\emph{``You can't cross the sea merely by standing and staring at the water.'' -- Rabindranath Tagore}
\end{center}

\section{Cross Products}\label{sec:cross}
\begin{Def}
Let $\bb u, \bb v\in F^3$. Then we define the \textbf{cross product} $\bb u \times \bb v$ to be the vector in $F^3$ of the form
\[\bb u \times \bb v = (u_2v_3-u_3v_2,\ u_3v_1 - u_1v_3,\ u_1v_2-u_2v_1).\]
\end{Def}\vs

In the language of determinants, we can define the cross product as 
\begin{equation}\label{eq:cross}\bb u \times \bb v = \left( \dtx{rr}{u_2 & v_2 \\ u_3 & v_3},\ -\dtx{rr}{u_1 & v_1 \\ u_3 & v_3},\ \dtx{rr}{u_1 & v_1 \\ u_2 & v_2}\right).\end{equation} Continuing with this observation, the three coordinates in the cross product are the three cofactors of the determinant $\dtx{rrr}{\bb e_1 & u_1 & v_1 \\ \bb e_2 & u_2 & v_2 \\ \bb e_3 & u_3 & v_3} = \dtx{rrr}{\bb e_1 & \bb e_2 & \bb e_3 \\ u_1 & u_2 & u_3 \\ v_1 & v_2 & v_3}$ when expanded across the first column. Thus, \[\bb u \times \bb v = \dtx{rrr}{\bb e_1 & u_1 & v_1 \\ \bb e_2 & u_2 & v_2 \\ \bb e_3 & u_3 & v_3} = \dtx{rrr}{\bb e_1 & \bb e_2 & \bb e_3 \\ u_1 & u_2 & u_3 \\ v_1 & v_2 & v_3} = \dtx{rr}{u_2 & v_2 \\ u_3 & v_3}\bb e_1 -\dtx{rr}{u_1 & v_1 \\ u_3 & v_3}\bb e_2 + \dtx{rr}{u_1 & v_1 \\ u_2 & v_2}\bb e_3.\]\vs 

There are important distinctions between the cross product and the dot product worth mentioning. First, the cross product of two vectors is itself a vector in $F^3$. For this reason, it is sometimes called the \emph{vector product}. Conversely, the dot product of two vectors is a scalar in $F$. For this reason, the dot product is sometimes called the \emph{scalar product}. Similarly, the outer product is sometimes called the \emph{matrix product} since $\bb u \otimes \bb v$ is a matrix. Second, while the dot product is defined for all vectors in $F^n$, the cross product is only defined here for vectors in $F^3$.\\

\begin{Exam} Let $\bb u = (1,2,-2)$ and $\bb v = (3,0,1)$. Then their cross product is given as
\[\bb u \times \bb v = (2(1) - (-2)(0), (-2)(3) - 1(1), 1(0) - 2(3)) = (2-0, -6-1, 0-6) = \fbox{$(2, -7, -6)$}.\qedhere\]
\end{Exam}\vs

\begin{Def} If $\bb u, \bb v, \bb w\in \R^3$, then $\bb u \cdot (\bb v \times \bb w)$ is called the \textbf{scalar triple product} of $\bb u$, $\bb v$, and $\bb w$. 
\end{Def}\vs

\begin{Thm} If $\bb u, \bb v, \bb w\in \R^3$, then 
\[\bb u \cdot (\bb v \times \bb w) = \dtx{rrr}{u_1 & v_1 & w_1 \\  u_2 & v_2 & w_2 \\  u_3 & v_3 & w_3}.\]
\end{Thm}
\begin{proof}
\begin{eqnarray*}
\bb u \cdot (\bb v \times \bb w) &=& \bb u \cdot \left( \dtx{rr}{v_2 & w_2 \\ v_3 & w_3},\ -\dtx{rr}{v_1 & w_1 \\ v_3 & w_3},\ \dtx{rr}{v_1 & w_1 \\ v_2 & w_2}\right)\\
&=& u_1\dtx{rr}{v_2 & w_2 \\ v_3 & w_3} -u_2\dtx{rr}{v_1 & w_1 \\ v_3 & w_3} + u_3\dtx{rr}{v_1 & w_1 \\ v_2 & w_2} = \dtx{rrr}{u_1 & v_1 & w_1 \\  u_2 & v_2 & w_2 \\  u_3 & v_3 & w_3},
\end{eqnarray*} where the last equality is seen by cofactor-expansion across the first column.
\end{proof}\vs

\begin{Cor} If $\bb u, \bb v, \bb w \in \R^3$, then the area of the parallelogram spanned by $\bb u$ and $\bb v$ is $\Vert \bb u\times \bb v\Vert$ and the volume of the parallelepiped spanned by $\bb u$, $\bb v$, and $\bb w$ is $|\bb u \cdot (\bb v\times \bb w)|$.
\end{Cor}\vs

\begin{Exam} Compute the scalar triple product $\bb u \cdot (\bb v\times \bb w)$ for the vectors $\bb u = (3,-2,-5)$, $\bb v = (1,4,-4)$, and $w=(0,3,2)$.
\[\bb u \cdot (\bb v\times \bb w) = \dtx{rrr}{3 & 1 & 0 \\  -2 & 4 & 3 \\  -5 & -4 & 2} = 3\dtx{rr}{4 & 3 \\ -4 & 2} - \dtx{rr}{-2 & 3\\ -5 & 2} = 3(8+12) - (-4+15) = 60 - 11 = \fbox{$49$}. \qedhere\]
\end{Exam}\vs

\begin{Thm}\label{thm:cross} Let $\bb u, \bb v, \bb w\in \R^3$ and let $k\in \R$. Then \vspace{-0.15 in}
\begin{multicols}{2}
\begin{enumerate}[!THM!, start=1]
\item\label{item:crossortho} $\bb u \cdot (\bb u \times \bb v) = 0$;
\item $\bb v \cdot (\bb u \times \bb v) = 0$;
\item $\Vert \bb u \times \bb v\Vert^2 = \Vert \bb u \Vert^2\Vert \bb v\Vert^2 - (\bb u \cdot \bb v)^2$;
\item $\Vert \bb u \times \bb v\Vert = \Vert\bb u\Vert \Vert \bb v\Vert\sin\theta$;
\item $\bb u \times (\bb v \times \bb w) = (\bb u\cdot \bb w)\bb v - (\bb u \cdot \bb v)\bb w$;
\item $(\bb u \times \bb v)\times \bb w = (\bb u\cdot \bb w)\bb v - (\bb v\cdot \bb w)\bb u$;
\item $\bb u \times \bb 0 = \bb 0 \times \bb u = \bb 0$;
\item\label{item:crossskew} $\bb u \times \bb v = -(\bb v \times \bb u)$;
\item\label{item:crossdist} $\bb u \times (\bb v + \bb w) = \bb u \times \bb v + \bb u \times \bb w$;
\item $(\bb u + \bb v) \times \bb w = \bb u \times \bb w + \bb v \times \bb w$;
\item $k(\bb u \times \bb v) = (k\bb u) \times \bb v = \bb u \times (k\bb v)$;

\item\label{item:crosszero} $\bb u \times \bb u = \bb 0$;
\end{enumerate}
\end{multicols}
\end{Thm}

%Prior to the proof, t
There are a few important observations to make. First, the cross product is noncommutative, that is, $\bb u \times \bb v \neq \bb v\times \bb u$. It is instead what we call \emph{anti-commutative}. Second, the cross product is nonassociative, that is, $\bb u \times (\bb v \times \bb w) \neq (\bb u \times \bb v)\times \bb w$. Instead, cross products satisfy what is known as the \emph{Jacobi identity}: 
\[\bb u \times (\bb v \times \bb w) + \bb v \times (\bb w \times \bb u) + \bb w\times (\bb u \times \bb v) = \bb 0.\footnotemark[2]\]
Finally, the cross product is necessarily orthogonal to the two factor vectors. This provides a usual tool for creating normal vectors in $F^3$.\\

%\begin{proof}
%Most of these results are consequences of properties of determinants since the triple scalar product is a $3\times 3$ determinant and the cross product is a vector made up of cofactors of $\dtx{rrr}{\bb e_1 & u_1 & v_1 \\ \bb e_2 & u_2 & v_2 \\ \bb e_3 & u_3 & v_3}$. Let use demonstrate this by proving $\eqref{item:crossortho}$ and $\ref{item:crossskew}$. \\
%
%To prove $\eqref{item:crossortho}$, note that 
%\[\bb u \cdot (\bb u \times \bb v) = \dtx{rrr}{u_1 & u_1 & v_1 \\  u_2 & u_2 & v_2 \\  u_3 & u_3 & v_3} = 0,\] since the first two columns are identical.\\
%
%To prove $\eqref{item:crossskew}$, note that 
%\[\bb u \times \bb v = \dtx{ccc}{\bb e_1 & \bb e_2 & \bb e_3 \\ u_1 & u_2 & u_3 \\ v_1 & v_2 & v_3} = -\dtx{ccc}{\bb e_1 & \bb e_2 & \bb e_3 \\ v_1 & v_2 & v_3 \\ u_1 & u_2 & u_3 } = -(\bb v\times \bb u).\]
%%\begin{eqnarray*}
%%\bb u \times \bb v &=& \left( \dtx{rr}{u_2 & v_2 \\ u_3 & v_3},\ -\dtx{rr}{u_1 & v_1 \\ u_3 & v_3},\ \dtx{rr}{u_1 & v_1 \\ u_2 & v_2}\right) = \left( -\dtx{rr}{v_2 & u_2 \\ v_3 & u_3},\ \dtx{rr}{v_1 & u_1 \\ v_3 & u_3},\ -\dtx{rr}{v_1 & u_1 \\ v_2 & u_2}\right)\\
%%&=& -\left( \dtx{rr}{v_2 & u_2 \\ v_3 & u_3},\ -\dtx{rr}{v_1 & u_1 \\ v_3 & u_3},\ \dtx{rr}{v_1 & u_1 \\ v_2 & u_2}\right) = -(\bb v\times \bb u).
%%\end{eqnarray*} 
%
%The proofs of the remaining properties are similar and left as an exercise for the student.
%\end{proof}\pagebreak

We have seen before that an $m$-flat in $F^n$ can be constructed in two ways:\\
\emph{Bottom-Up}---Using $m$ linearly independent spanning vectors $\bb v_1,\ldots, \bb v_m$, we construct a \emph{vector equation} for which the flat is the solution set to the vector equation
\[\bb x = \bb x_0 + c_1\bb v_1 + c_2\bb v_2 + \ldots + c_m\bb v_m,\] where $\bb x_0$ is a vector on the flat, $c_1, c_2,\ldots, c_m\in F$ are scalar parameters. Of course, a vector equation corresponds to a linear system, which implies the flat is the solution set of this linear system.\\\\
\emph{Top-Down}---Using $n-m$ linearly independent normal vectors $\bb n_1, \ldots, \bb n_{n-m}$, we construct $n-m$ linearly independent \emph{scalar equations} of the form $\bb n_{i}\cdot (\bb x - \bb x_0) = 0$ or $\bb n_i \cdot \bb x = \bb n_i \cdot \bb x_0$. If $\bb n_i = (a_1, a_2, \ldots, a_n)\in F^n$, then this scalar equation has the form 
\[a_1x_1 + a_2x_2 + \ldots + a_nx_n = d,\] for some $d\in F$. Then the flat is the solution to the system of all these scalar equations. These normal vectors, of course, belong to the orthogonal complement of the flat.\\

How does one transition between these two representations of the flats? If the \emph{Top-Down} representation is given, one could solve the linear system $A\bb x=\bb b$ to find $\bb x = \bb x_0 + c_1\bb v_1 + c_2\bb v_2 + \ldots + c_m\bb v_m,$ where $\bb x_0$ is a particular solution and $\{\bb v_1,\ldots, \bb v_m\}$ is a basis for $\nul(A)$. This gives the \emph{Bottom-Up} representation of the flat.\\

If we start with the \emph{Bottom-Up} representation $\bb x = \bb x_0 + c_1\bb v_1 + c_2\bb v_2 + \ldots + c_m\bb v_m$, we need to a matrix $A$ and vector $\bb b$ such that $\nul(A) = \Span\{\bb v_1, \ldots, \bb v_n\}$. As $\nul(A)^\perp = \row(A)$, we need to find vector orthogonal to the spanning vectors. This can be best accomplished by creating a matrix $B$ such that $\row(B) = \nul(A) = \Span\{\bb v_1, \ldots, \bb v_n\}$. Then $\nul(B) = \row(B)^\perp = \nul(A)^\perp = \row(A)$. With these normal vectors, we can construct the linear system for the flat.\\

%NEW
\begin{Exam} Consider the $2$-flat in $F^4$ given by the vector equation $\bb x = \bb x_0 + s\bb u + t\bb v$ where $\bb x_0 = (1,2,3,4)$, $\bb u = (1,0,0,-1)$, and $\bb v= (2,1,-3,0)$. Find a linear system $A\bb x=\bb x$ for which this flat is the solution set.\\

Let $B = \mtx{rrrr}{1&0&0&-1\\2&1&-3&0}$. Then \vspace{-0.25 in}
\[B \sim \mtx{rrrr}{1&0&0&-1 \\ 0&1&-3&-2}.\quad \text{Hence,}\ \row(B)^\perp=\nul(B) = \Span\left\{\vr{0\\3\\1\\0}, \vr{1\\2\\0\\1}\right\}.\] 
Therefore, the flat is the solution to the following linear system:
\[\begin{linear} (0,3,1,0)\cdot (x_1-1, x_2-2, x_3-3, x_4-4)\ &=\ &0\\
(1,2,0,1)\cdot (x_1-1, x_2-2, x_3-3, x_4-4)\ &=\ &0
\end{linear} \qRightarrow \begin{linear} 0x_1\ &+\ &3x_2\ &+\ &1x_3\ &+\ &0x_4\ &=\ &0(1) + 3(2) + 1(3) + 0(4)\\
1x_1\ &+\ &2x_2\ &+\ &0x_3\ &+\ &1x_4\ &=\ &1(1) + 2(2) + 0(3) + 1(4)\end{linear}\]
\[\qRightarrow \begin{linear}  & &3x_2\ &+\ &x_3\ & & &=\ & 9\\
x_1\ &+\ &2x_2\ & & &+\ &x_4\ &=\ & 9\end{linear}\]
One can easily check that the vectors in the flat above are solutions to this linear system.
\end{Exam}\vs

Now, when one wants to find the equation of a hyperplane, then the single normal can be found using determinants in a manner similar to \eqref{eq:cross}. Of course, when $n=3$, this is just the cross product.\\

\begin{Exam} Find an equation for the plane in $\R^3$ which passes through $(1,-2,1)$, $(-1,0,1)$, and $(3,2,0)$.\\ %NEW

We can construct the equation using a normal vector. First, consider the plane as a vector equation. Let $\bb u = (3,2,0)-(1,-2,1) = (2,4,-1)$ and $\bb v = (3,2,0)-(-1,0,1) = (4,2,-1)$. If $\bb x_0 = (-1,0,1)$, then the plane is given as $\bb x = s\bb u + t\bb v + \bb x_0$. As the vectors $\bb u$ and $\bb v$ give the ``slope'' of the plane, we want to find a vector orthogonal to both $\bb u$ and $\bb v$. Such a vector is $\bb u \times \bb v$. Note
\[\bb u \times \bb v = \dtx{rrr}{\bb e_1 & 2 & 4 \\ \bb e_2 & 4 & 2 \\ \bb e_3 & -1 & -1} = \dtx{rr}{4&2\\-1&-1}\bb e_1 - \dtx{rr}{2&4\\-1&-1}\bb e_2 + \dtx{rr}{2&4\\4&2}\bb e_3 = \mtx{c}{-4+2\\-(-2+4)\\4-16} = \vr{-2\\-2\\-12}.\]
Thus, the equation for the plane can be given as $-2(x-(-1))-2(y-0) -12(z-1) = 0\\ \qRightarrow -2x-2-2y-12z+12 = 0 \qRightarrow \fbox{$x+y+6z = 5$}$.
\end{Exam}\vs

%%%%%%%%%%%%%%%%%% Exercises %%%%%%%%%%%%%%%%%%%
\startExercises{cross}

\noindent For Exercises \ref{exer:crossstart}-\ref{exer:crossstop}, compute $\bb u\times \bb v$.
\begin{enumerate}[!HW!, start=1, label=$\spadesuit$ \arabic*., ref=\arabic*]
\begin{multicols}{2}
\item\label{exer:crossstart} $\bb u = \vr{1\\-1\\2}, \bb v  =\vr{3\\2\\-1}$ %NEW 
\item $\bb u =\vr{1\\2\\4}, \bb v = \vr{2\\6\\-1}$ %NEW 
\end{multicols}
\item\label{exer:crossstop} $\bb u =\vr{1\\1\\1}, \bb v = \vr{-2\\0\\4}$ %NEW 
\end{enumerate}

\noindent For Exercises \ref{exer:crosssimplifystart}-\ref{exer:crosssimplifystop}, simplify the expression.
\begin{enumerate}[!HW!, label=$\spadesuit$ \arabic*., ref=\arabic*]
\begin{multicols}{2}
\item\label{exer:crosssimplifystart} $\left(\vr{3\\4\\0}+\vr{1\\-3\\1}\right) \times \vr{3\\1\\2}$ %NEW 
\item\label{exer:crosssimplifystop} $\left(\vr{0\\3\\4}-2\vr{-1\\2\\5}\right) \times \left(2\vr{4\\1\\1} - \vr{2\\-1\\3}\right)$ %NEW 
\end{multicols}
\end{enumerate}

\noindent For Exercises \ref{exer:constructlinearsystemaffinesetstart}-\ref{exer:constructlinearsystemaffinesetstop}, use normal vectors to construct a system of linear equations so that the given affine set is the solution set to your linear system. 
\begin{enumerate}[!HW!, label=$\spadesuit$ \arabic*., ref=\arabic*]
\item\label{exer:constructlinearsystemaffinesetstart} The plane in $\R^3$ which passes through $(1,-2,1)$, $(-1,0,1)$, and $(3,2,0)$. %NEW
\item The hyperplane in $\R^4$ which passes through $(2,-1,3,-1)$, $(3,0,-2,2)$, $(1,-2,0,2)$, and $(-1,-2,-2,4)$. %NEW
\item\label{exer:constructlinearsystemaffinesetstop} The plane in $\R^4$ which passes through $(1,2,3,4)$, $(2,3,0,1)$, and $(0,1,2,4)$.  %NEW
\end{enumerate}\vs

\begin{enumerate}[!HW!]
\item Prove \thmref{thm:cross} \ref{item:crossdist}.

\item Prove \thmref{thm:cross} \ref{item:crosszero}.
\end{enumerate}


%%%%%%%%%%%%%%%%%%% Footnotes %%%%%%%%%%%%%%%%%%%
 \mbox{}\vfill
 
 \footnotetext[2]{The Jacobi Identity in addition to axioms (g)--(k) makes $F^3$ equipped with the cross product into a special type of vector space known as a \textbf{Lie algebra}.}
\pagebreak