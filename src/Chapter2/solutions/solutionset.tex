\startSolutions{solutionset}{Solution Sets of Linear Systems}

\begin{enumerate}[!HW!, start=1]
\begin{multicols}{4}
\item Yes; $x_3$
\item $\Span\left\{\vr{-1\\0\\2\\0}\right\}$ %anon
\itemspade $\Span\left\{\vr{2\\1\\0}\right\}$
\item $\Span\left\{\vr{2\\-2\\1\\0}, \vr{5\\-4\\0\\3}\right\}$ %Runtian Tu
\end{multicols}
\begin{multicols}{3}
\itemspade $\Span\left\{\vr{-2\\1}\right\}$\columnbreak
\itemspade $\Span\left\{\vr{2\\1\\0\\0\\0}, \vr{-5\\0\\3\\-4\\1}\right\}$\columnbreak
\itemspade $\Span\left\{\vr{1\\-2\\1\\0\\0}, \vr{-3\\4\\0\\1\\0}, \vr{-5\\0\\0\\0\\1}\right\}$
\end{multicols}
\begin{multicols}{2}
\item \mbox{$\bb x = \vr{5\\0\\0\\-8}+s\vr{3\\1\\0\\0}+t\vr{-4\\0\\1\\0}$} %Allyson Vest
\itemspade $\bb x = \vr{1\\2\\0} + t\vr{2\\3\\1}$
\end{multicols}
\begin{multicols}{2}
\itemspade $\bb x = \vr{1\\2\\3\\0} + t\vr{0\\0\\0\\1}$
\itemspade $\bb x = \mtx{c}{1+i\\3+4i\\0} + t\mtx{c}{0\\-2\\1}$
\end{multicols}
\itemspade $\bb x\equiv \vr{0\\1\\1\\0\\0\\0} + a\vr{1\\1\\1\\1\\0\\0}+b\vr{1\\0\\0\\0\\1\\0}+c\vr{1\\1\\0\\0\\0\\1}$


\itemspade We saw in \examref{exam:solutionset} that $\bb x_0 = \vr{6\\2\\0}$ and $\bb x_1 = \vr{5\\2\\1}$ are two solution to the linear system, but $\bb x = \bb x_0 + \bb x_1 \equiv \vr{4\\4\\1} \pmod 7$, which is not a solution $\bb x$ since 
\[A\bb x = \mtx{rrrr}{3&5&3\\4&5&4\\6&1&6}\vr{4\\4\\1} = \vr{3(4)+5(4) + 3(1) \\ 4(4) + 5(4) +4(1) \\ 6(4) + 1(4) +6(1)} \equiv \vr{0\\ 5\\ 6} \not\equiv \vr{0\\6\\3}.\]
\itemspade We saw in \examref{exam:solutionset} that $\bb x_0 = \vr{6\\2\\0}$. Then $2\bb x_0 \equiv \vr{5\\4\\0} \pmod 7$. But then 
\[A\bb x = \mtx{rrrr}{3&5&3\\4&5&4\\6&1&6}\vr{5\\4\\0} = \vr{3(5)+5(4) + 3(0) \\ 4(5) + 5(4) +4(0) \\ 6(5) + 1(4) +6(0)} \equiv \vr{0\\ 5\\ 6} \not\equiv \vr{0\\6\\3}.\]  
\itemspade Hint: Suppose $\bb x_0$ and $\bb x_1$ are solutions to the linear system $A\bb x = \bb b$. Multiply the equation $\bb x = (1-t)\bb x_0 + t\bb x_1$ by the matrix $A$, using also the fact that $\bb x_0$ and $\bb x_1$ are solutions to the linear system. Recall that multiplication by a matrix distributes across vector addition and commutes with scalar multiplication. It is a linear transformation, after all.
\item Hint: Like the hint to the previous exercise, suppose $\bb x= a_0\bb x_0 + a_1\bb x_1+\ldots + a_m\bb x_m$ such that $\bb x_0, \bb x_1, \ldots, \bb x_m$ are solutions to $A\bb x=\bb b$ and $a_0+a_1\ldots + a_m=1$. Then simplify 
\[A(a_0\bb x_0 + a_1\bb x_1+\ldots + a_m\bb x_m).\]
\end{enumerate}

\vspace{-15 pt}