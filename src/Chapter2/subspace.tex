\begin{center} 
\emph{``For the wise man looks into space and he knows there is no limited dimensions.'' -- Lao Tzu}
\end{center}

\section{Subspaces}\label{sec:subspace}
Lines, planes, and hyperplanes through the origin are just special examples of subspaces of $F^n$.\\

\begin{Def} Let $V$ be a vector space, such as $F^n$. A \textbf{subspace} of $V$ is any set $W \subseteq V$ such that:
\begin{enumerate}[!THM!, start=1]
\item $\bb 0 \in W$,
\item For each $\bb u, \bb v \in W$, the sum $\bb u + \bb v \in W$,
\item For each $\bb v \in W$ and scalar $c\in F$, the vector $c\bb v \in W$.
\end{enumerate}
\end{Def}\vs

In words, a subspace is nonempty subset of $V$ \emph{closed} under addition and scalar multiplication.  It should be mentioned that  closure under addition and scalar multiplication is equivalent to closure under linear combinations.\\

When considering vector spaces before, we saw that a set of things is called a vector space when we can appropriately add and scale all the objects, which are then called vectors. Necessary to this definition is that the sum of two vectors and a scaled vector still be vectors! A subspace is a vector space inside of a vector space, that is, each subspace of $V$ is also a vector space in its own right. The sum of two vectors or a scaled vectors originating from $W$ will remain in $W$ still. In regard to the axioms of a vector space, the subspace inherits the axioms from the ambient vector space. For example, since ALL vectors commute in the larger vector space, all vectors will commute in the smaller subset. \\

The vector space $V$ in consideration will nearly always be $F^n$ for some field $F$. It should be mentioned that $F^n$ is a subspace of $F^n$ because it satisfies all three conditions. Also, the \textbf{zero space} $F^0=\{\bb 0\}$ is also a subspace.\\

\begin{Exam}\label{exam:spansubspace}
If $\bb v_1, \bb v_2 \in F^n$ and $W = \Span\{\bb v_1, \bb v_2\}$, then we claim that $W$ is a subspace of $F^n$. To prove this claim, we must show that $W$ satisfies the three conditions in the definition. Let $c, s_1, s_2, t_1, t_2 \in F$. Then
\begin{enumerate}[!THM!, start =1]
\item $\bb 0 = 0\bb v_1 + 0\bb v_2 \in W$,
\item $(s_1\bb v_1 + s_2\bb v_2) + (t_1\bb v_1 + t_2\bb v_2) = (s_1+t_1)\bb v_1 + (s_2+t_2)\bb v_2 \in W$,
\item $c(s_1\bb v_1 + s_2\bb v_2) = (cs_1)\bb v_1 + (cs_2)\bb v_2 \in W$.
\end{enumerate} Therefore, $W$ is a subspace of $F^n$. By mathematical induction, this same argument shows that a span for any number of vectors is also a subspace of $F^n$.
\end{Exam}\vs

Because of this, we often say that a span of a set of vector $\mathcal{S}$ is the \textbf{subspace spanned by} $\mathcal{S}$. Likewise, if $H$ is a subspace of $F^n$ and $W = \Span\{\mathcal{S}\}$ for some $\mathcal{S}\subseteq F^n$, then we say that $\mathcal{S}$ is a \textbf{spanning set} for $W$.\\

By our consideration of lines, planes, hyperplanes, and affine sets (flats) which pass through the origin are subspaces because it is just a span of vectors. In fact, every coset is simply the translation of a subspace. On the other hand, a translation of a subspace will likely not be a subspace since it no longer contains the zero vector.\\

\begin{Exam} In $\R^2$, all lines through the origin are 1-dimensional subspaces. On the other hand, other lines are not subspaces since they do not contain the origin which is the zero vector of $\R^2$.\\

Let $W$ be the set of all points $(x,y)\in \R^2$ for which $x\ge 0$ and $y\ge 0$, that is, $W$ is the first quadrant of the plane. This is not a subspace. On the one hand, the set contains $\bb 0$ and is closed under addition. For example, if $x_1, x_2, y_1, y_2\ge 0$, then $x_1+x_2, y_1+y_2\ge 0$ and $(x_1,y_2) + (x_2,y_2) \in W$. On the other hand, if $x, y\ge 0$, then $-x,-y\le 0$ and $-(x,y) = (-x,-y) \notin W$. Hence, $W$ is not a subspace of $\R^2$.
\end{Exam}\vs

Let $A = \mtx{rrrr}{\bb a_1 & \bb a_2 & \ldots & \bb a_n}$ be an $m\times n$ matrix. Then recall that the \textbf{column space} of $A$, denoted $\col A$, is $\Span\{\bb a_1, \bb a_2, \ldots, \bb a_n\}$. The column space of $A$ is the set of all $\bb b\in F^m$ such that $A\bb x = \bb b$ is consistent. By \examref{exam:spansubspace}, $\col A$ is a subspace of $F^m$.\\

\begin{Exam} Let $A = \mtx{rrr}{1&8&7\\7&6&9\\8&7&6}$ and $\bb b = \vr{3\\3\\7}$ over $\Z_{11}$. Determine whether $\bb b \in \col A$.\\

We need to determine whether $\bb b$ is a linear combination of the column vectors of $A$. This is the same as solving the matrix equation $A\bb x = \bb b$. To do this, we use row reduction:
\[\mtx{rrr|r}{1&8&7&3\\7&6&9&3\\8&7&6&7} \sim \mtx{rrr|r}{1&8&7&3\\0&5&4&4\\0&9&5&5} \sim \mtx{rrr|r}{1&8&7&3\\0&5&4&4\\0&0&0&0} \pmod{11}.\] Thus, the system is consistent, which implies that $\bb b$ is in $\col A$.
\end{Exam}\vs

%%%%%%%%%%%%%%%%%%%% More Characterizations of Subspaces %%%%%%%%%%%%%%%%%
%\begin{Thm} If $V$ is a vector space and $W \subseteq V$, then $W$ is a subspace of $V$ if and only if 
%\begin{enumerate}[!THM!, start=1]
%\item $\bb 0 \in W$,\\
%\item For each $\bb u, \bb v \in W$ and scalars $s,t\in \R$, the linear combination $s\bb u + t\bb v \in W$.
%\end{enumerate}
%\end{Thm}
%\begin{proof}
%Because of $(i)$, it suffices to show that $(ii)$ is equivalent to $W$ being closed under addition and scalar multiplication. Suppose that $(ii)$ holds. If $\bb u, \bb v\in W$ then $\bb u + \bb v = 1\bb u+1\bb v\in W$, by $(ii)$. Also, if $c\in \R$, then $c\bb u = c\bb u + 0\bb v\in W$, again by $(ii)$. Thus, $W$ is closed under addition and scalar multiplication. Conversely, suppose that $W$ is closed under addition and scalar multiplication. If $\bb u, \bb v\in W$ and $s,t\in \R$, then $s\bb u, t\bb v\in W$, by closure of scalar multiplication. Likewise, $s\bb u + t\bb v\in W$, by closure of addition. This proves $(ii)$. 
%\end{proof}\vs
%
%\begin{Cor} If $V$ is a vector space and $W \subseteq V$, then $W$ is a subspace of $V$ if and only if 
%\begin{enumerate}[!THM!,start=1]
%\item $W\neq \emptyset$, that is, $W$ is nonempty,\\
%\item For each $\bb u, \bb v \in W$ and scalars $s,t\in \R$, the linear combination $s\bb u + t\bb v \in W$.
%\end{enumerate}
%\end{Cor}
%\begin{proof}
%By the previous theorem, it suffices to show that $(i)$ is equivalent to $\bb 0 \in W$. If $\bb 0 \in W$, then $W\neq \emptyset$, which proves $(i)$. Suppose $(i)$, that is, $W\neq \emptyset$. Then there exists some vector $\bb v\in W$. By $(ii)$, we have that $\bb 0 = 1\bb v + (-1)\bb v \in W$. This shows that $W$ contains the zero vector.
%\end{proof}\vs
%
%To show that a set if nonempty, we must show that there is some vector inside of the set. For the most part, the easiest vector to identify is the zero vector. So, we will most often show that subspaces contain the zero vector.\\

%\begin{Exam} The set of upper triangular matrices in $M_{mn}$ is a subspace. To see this, notice that the zero ``vector,'' that is, the zero matrix is upper triangular. Also, linear combinations of upper triangular matrices are upper triangular, since adding and scaling zero entries will produce only zero entries. This same reasoning also shows that the set of lower triangular matrices, the set of diagonal matrices, and the set of symmetric matrices are all subspaces of $M_{mn}$.\\
%
%On the other hand, the set of invertible matrices is not a subspace of $M_{nn}$. To see this, merely note that the zero matrix is NOT invertible, and hence the set of invertible matrices does not contain the zero matrix. Likewise, the set of elementary matrices is not a subspace.
%\end{Exam}\vs
%%%%%%%%%%%%%%%%%%%% More Characterizations of Subspaces %%%%%%%%%%%%%%%%%

\begin{Exam}\mbox{}
\begin{enumerate}
\item Let $V = \R^\infty$ be the set of all real-valued sequences. This is a vector space since we can add two sequences:
\[\{x_1, x_2, x_3,\ldots \} + \{y_1, y_2, y_3,\ldots\} = \{x_1+y_1, x_2+y_2, x_3+y_3,\ldots \}\in\R^\infty,\] and scale a sequence:
\[c\{x_1, x_2, x_3,\ldots \} = \{cx_1, cx_2, cx_3,\ldots \}.\]  Notice the zero vector of this vector space is the constant \textbf{zero sequence}:
\[\bb 0 = \{0,0,0,\ldots\}.\] 

Let $W$ be the subset of $\R^\infty$ of convergent sequences. Since the zero sequence converges (to zero), it is contained in $W$. By limit properties, the sum of convergent sequences is convergent and converges to the sum of limits. Likewise, a multiple of a convergent sequence converges to a multiple of the limit. Therefore, $W$ is a subspace of $\R^\infty$. By similar reasoning, the set of all sequence which converge to zero is a subspace of $\R^\infty$.\\

\item Let $\R^{X} = \{f : X\to \R\}$, that is, the set of all real-valued functions where $X\subseteq \R$, e.g. $\R^\R$ is the set of real-valued function defined on the entire $x$-axis. Then $\R^X$ is also a vector space. Note that if $f, g\in \R^X$, then the sum of ``vectors'' is the function defined by the rule:
\[(f+g)(x) = f(x) +g(x),\] and scalar multiplication is given by the rule:
\[(cf)(x) = c[f(x)].\] For this vector space, the function $f(x) = 0$ for all $x\in X$ is the zero vector, called the \textbf{zero function}.\\

Let $\P$ denote the set of all polynomials with real coefficients. Viewing $\P$ as a subset of $\R^X$, as the domain of any polynomial can be restricted to $X$, then $\P$ is a subspace of $\R^\R$. To see this, note that the \textbf{zero polynomia}l is the same as the zero function and is contained in $\P$. Second, the sum of two polynomials is again a polynomial. Third, a polynomial times by a real number is again a real-valued polynomial. Therefore, $\P$ is a subspace of $\R^\R$.\\

Let $\P_n$ be the set of polynomials with degree at most $n$. By the same reasoning as before, $\P_n$ is a subspace of $\P$ and hence a subspace of $\R^X$. In fact, $\P_n = \Span(1, x, x^2, \ldots, x^n)$. Furthermore,  
\[\P_0 \le \P_1\le \P_2 \le \P_3\le \ldots \le \P \le \R^X.\]

\item Let $\mathcal{C}(X)$ be the set of all real-valued continuous functions on the domain $X$. This is a subset of $\R^X$. By facts from Calculus, the zero function is continuous, the sum of continuous functions is continuous, and the multiple of a continuous function is continuous. Thus, $\mathcal{C}(X) \le \R^X$.\\

Likewise, we can define the set $\mathcal{C}^1(X)$ to be the set of all real-valued continuously differentiable ($f'$ is continuous) functions on the domain $X$. Likewise, Calculus tells us that constant functions are  differentiable, sums of differentiable functions are differentiable, and multiples of differentiable functions are differentiable. Thus, $\mathcal{C}^1(X) \le \R^{X}$. Of course, since differentiable functions are necessarily continuous, we have $\mathcal{C}^1(X) \le \mathcal{C}(X) \le \R^X$. \\

Likewise, we can define $\mathcal{C}^n(X)$ to be the set of functions $f$ for which $f^{(n)}$ is continuous on $X$. By similar reasoning, each of these sets are subspaces of $\R^X$. Let $\mathcal{C}^\infty(X)$ be the set of all functions in $\R^X$ for which all higher derivatives exists (and necessarily are continuous). This is likewise a subspace and called the space of \textbf{smooth functions}. In fact, we have the following descending sequences of subspaces:
\begin{eqnarray*}
\R^X &\ge& \mathcal{C}(X) \ge \mathcal{C}^1(X) \ge \mathcal{C}^2(X) \ge \ldots \ge \mathcal{C}^n(X) \ge \ldots \ge \mathcal{C}^{\infty}(X)\\
&\ge& \mathcal{P} \ge \ldots \ge \mathcal{P}_n \ge \ldots \ge \mathcal{P}_2 \ge \mathcal{P}_1 \ge \mathcal{P}_0.
\end{eqnarray*}

All these vector spaces allude to why Linear Algebra is extremely useful for Advanced Calculus (often called Real Analysis). Linear algebra is everywhere in calculus!\\

With the exception of those concepts directly related to calculus, such as continuity and derivatives, the above examples of function spaces can be adapted to any field $F$. \hfill$\qedhere$

%\item We have seen that the solution set to the homogeneous matrix equation $A\bb x = \bb 0$ is a subspace of $\R^n$, when $A$ is an $m\times n$ matrix. This is just the null space of $A$. In a similar way, the solution to a homogeneous differential equation of order $n$ is a subspace of $C^n[a,b]$. This is yet another reason why Linear Algebra greatly benefits Calculus. \hfill$\qedhere$
\end{enumerate}
\end{Exam}\vs

%\begin{Thm} Let $V$ and $W$ be subspaces of a vector space $U$. Then $V\cap W \le U$.
%\end{Thm}
%\begin{proof}
%Since $V, W\le U$, it must be that $\bb 0 \in V, W$. Thus, $\bb 0 \in V\cap W$. Likewise, if $\bb u, \bb v\in V\cap W$, then $\bb u, \bb v\in V$. Since $V$ is a subspace, it must be that $s\bb u + t\bb v \in V$ for all $s,t\in \R$. Similarly, if $\bb u, \bb v \in W$, then $s\bb u + t\bb v\in W$. Therefore, $s\bb u + t\bb v \in V\cap W$. Therefore, $V\cap W$ is a subspace of $U$.
%\end{proof}\vs

%\begin{Exam} Let $S=\Span(x^2-2x+3, -2x^2+3x+1)$ be a subspace of $\P_2$. Determine if $p(x) = 10x^2-17x+9$ is in $S$.\\ %Holt p.287 Example 1
%
%Typically, we can check if a vector is in the span of a set of vectors by solving a linear system. We will do the same here. In fact, we will do EXACTLY the same thing as we have done in the past when we switch to coordinates. Using the standard basis $\{1, x, x^2\}$, we get
%\[\bb u = \vr{3\\-2\\1} = [x^2-2x+3],\quad \bb v = \vr{1\\3\\-2} = [-2x^2+3x+1],\quad [p(x)] = \vr{9\\-17\\10}.\] Then we solve the linear system:
%\[\mtx{cc|c}{\bb u & \bb v & [p(x)]} = \mtx{rr|r}{3 & 1 & 9\\ -2 & 3 & -17 \\ 1 & -2 & 10} \sim \mtx{rr|r}{1 & 0 & 4\\ 0 & 1 & -3\\ 0 & 0 & 0}.\] Therefore, $p(x) = 4(x^2-2x+3) - 3(-2x^2+3x+1)$ and $p(x) \in S$.
%\end{Exam}\vs
%
%\begin{Exam} Let $S=\Span\left(\mtx{rr}{1&0\\2&1}, \mtx{rr}{0&-1\\1&3}, \mtx{rr}{4&1\\-2&1}\right)$ be a subspace of $M_{2,2}$. Determine if $\bb v = \mtx{rr}{2&5\\-3&4}$ is in $S$.\\ %Holt p.287 Example 3
%
%\begin{multicols}{2}
%Using the standard basis \[\left\{\mtx{rr}{1&0\\0&0}, \mtx{rr}{0&1\\0&0}, \mtx{rr}{0&0\\1&0}, \mtx{rr}{0&0\\0&1}\right\},\] we need to solve the linear system displayed to the right. In echelon form, we see the linear system is inconsistent. Therefore, $\bb v \notin S$.
%
%\[\mtx{rrr|r}{1 & 0 & 4 & 2\\ 0 & -1 & 1 & 5\\ 2 & 1 & -2 & -3\\ 1 & 3 & 1  & 4} \sim \mtx{rrr|r}{1 & 0 & 4 & 2\\ 0 & -1 & 1 & 5\\ 0 & 0 & -9 & -2 \\ 0 & 0 & 0 & 17}.\] 
%\end{multicols}
%\end{Exam}
%
%\begin{Exam} Let $S = \Span(1, x, e^x)$ be a subspace of $C(-\infty, \infty)$. What is $\dim(S)$?\\
%
%Since $S$ is defined as the span of $\{1, x, e^x\}$, we already have a spanning set for $S$. To find $\dim(S)$, we need a basis of $S$. So is this spanning set linearly independent? Thus, we need to solve the vector equation $c_1(1) + c_2(x) + c_3(e^x) = 0$, where $c_1, c_2, c_3\in \R$. In $C(-\infty, \infty)$, the zero vector is the constant zero function. Thus, the equation needs to hold for every value of $x$. As we have three functions in the equation, we evaluate at three values $x=-1, 0, 1$. 
%\[\begin{linear} c_1\ &-\ &c_2\ &+\ &e^{-1}c_3\ &=\ &0\\
% c_1\ & & &+\ &c_3\ &=\ &0\\
%c_1\ &+\ &c_2\ &+\ &ec_3\ &=\ &0\\
%\end{linear}\quad \sim\quad \mtx{rrr|r}{1 & -1 & e^{-1} & 0\\ 1 & 0 & 1 & 0\\ 1& 1 & e & 0}\quad \sim\quad \mtx{rrr|r}{1 & 0 & 0 &0\\ 0& 1 & 0 & 0\\ 0 &0 & 1 & 0}.\] Therefore, $\{1, x, e^x\}$ is linearly independent, forming a basis for $S$. Then $\dim(S) = 3$. \hfill$\qedhere$
%\end{Exam}\vs

%%%%%%%%%%%%%%%%%%% Exercises %%%%%%%%%%%%%%%%%%%
\startExercises{subspace}

\noindent For Exercises \ref{exer:subspacefailstart}-\ref{exer:subspacefailstop}, explain why the subset of $\R^2$ does NOT form a subspace, that is, explain which of the three axioms are satisfied and which ones fail. Provide a counterexample to justify when they fail. Answers may vary.
\begin{enumerate}[!HW!, start=1, label=$\spadesuit$ \arabic*., ref=\arabic*]
\begin{multicols}{2}
\item\label{exer:subspacefailstart} $W$ is the unit circle, that is,\\ $W = \{(x,y)\mid x^2+y^2=1\}$.
\item $W$ is the unit disc, that is,\\ $W = \{(x,y)\mid x^2+y^2\le1\}$.
\end{multicols}
\begin{multicols}{2}
\item $W$ is the standard parabola, that is,\\ $W = \{(x,y)\mid y=x^2\}$.
\item $W$ is the union of the $x$- and $y$-axis, that is,\\ $W = \{(x,y)\mid x=0\text{ or }y=0\}$.
\end{multicols}
\end{enumerate}
\begin{enumerate}[!HW!]
\begin{multicols}{2}
\itemspade $W$ is the upper half-plane, that is,\\ $W= \{(x,y)\mid y\ge 0\}$.
\item\label{exer:subspacefailstop} $W$ is the punctured plane, that is,\\ $W = \{(x,y)\mid x^2+y^2\ge 1\}$. %Cyrus Kaveh
\end{multicols}
\end{enumerate}

\noindent For Exercises \ref{exer:subspacefunctionstart}-\ref{exer:subspacefunctionstop}, decide if the subset of the function space $\R^\R$ forms a subspace. Justify your answer in a similar style to Exercises \ref{exer:subspacefailstart}-\ref{exer:subspacefailstop}. Answers may vary.
\begin{enumerate}[!HW!, label=$\spadesuit$ \arabic*., ref=\arabic*]
\begin{multicols}{2}
\item\label{exer:subspacefunctionstart} The set of functions whose $y$-intercept is $1$, that is, $f(0)=1$.
\item The set of functions whose $y$-intercept is $0$, that is, $f(0)=0$.
\end{multicols}
\begin{multicols}{2}
\item The set of functions whose end behavior on the right is $\infty$, that is, $\dlim_{x\to\infty} f(x) = \infty$.
\item\label{exer:subspacefunctionstop} The set of odd functions, that is, $f(-x)=-f(x)$.
\end{multicols}
\end{enumerate}

%%%%%%%%%%%%%%%%%%% Footnotes %%%%%%%%%%%%%%%%%%%
% \mbox{}\vfill
\pagebreak