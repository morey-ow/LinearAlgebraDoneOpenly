\begin{center} 
\emph{``If the world's a veil of tears, Smile till rainbows span it.'' -- Lucy Larcom}
\end{center}

\section{Vector Equations}\label{sec:span}
Linear combinations are the bread and butter of vectors. Algebraically speaking, this is what we do with vectors, that is, we combine some vectors to construct new vectors. While it is simple to take a list of vector, combine them, and see what pops out, the converse is not as simple. Given a list of vectors $\bb a_1, \bb a_2, \ldots, \bb a_n\in F^m$, can we determine if $\bb b\in F^m$ is a linear combination of these vectors? In other words, we seek to solve the vector equation:
\begin{equation}\label{eq:vectoreqn} x_1\bb a_1 + x_2\bb a_2 + \ldots + x_n\bb a_n = \bb b.\end{equation} In Equation \eqref{eq:vectoreqn}, note that the vectors $\bb a_1,\ldots, \bb a_n, \bb b$ are fixed and the variables are the scalars $x_1, x_2, \ldots, x_n\in F$. \\

To solve the vector equation \eqref{eq:vectoreqn}, suppose that the $i$th component of $\bb a_j$ is denoted $a_{ij}$, that is, 
\[\bb a_j = \mtx{c}{a_{1j} \\a_{2j} \\ \vdots\\ a_{mj}}.\] Likewise, let the $j$th component of $\bb b$ be $b_j$. Then Equation \eqref{eq:vectoreqn} becomes
\begin{eqnarray*} 
x_1\bb a_1 + x_2\bb a_2 + \ldots + x_n\bb a_n &=& \bb b\\
x_1\mtx{c}{a_{11} \\a_{21} \\ \vdots\\ a_{m1}} + x_2\mtx{c}{a_{12} \\a_{22} \\ \vdots\\ a_{m2}} + \ldots + x_n\mtx{c}{a_{1n} \\a_{2n} \\ \vdots\\ a_{mn}} &=& \mtx{c}{b_1\\b_2\\\vdots\\ b_m}\\
\mtx{c}{a_{11}x_1 \\a_{21}x_1 \\ \vdots\\ a_{m1}x_1} + \mtx{c}{a_{12}x_2 \\a_{22}x_2 \\ \vdots\\ a_{m2}x_2} + \ldots + \mtx{c}{a_{1n}x_n \\a_{2n}x_n \\ \vdots\\ a_{mn}x_n} &=& \mtx{c}{b_1\\b_2\\\vdots\\ b_m}\\
\mtx{c}{a_{11}x_1 + a_{12}x_2 + \ldots + a_{1n}x_n \\ a_{21}x_1 + a_{22}x_2 + \ldots + a_{2n}x_n \\ \vdots \\ a_{m1}x_1 + a_{m2}x_2 + \ldots + a_{mn}x_n}  &=& \mtx{c}{b_1\\b_2\\\vdots\\ b_m}\\
\end{eqnarray*} 

Examining this last equality of vectors reveals a system of linear equations, namely 
\begin{equation} \label{eq:linearsystem}
\begin{linear}
a_{11}x_1\ &+\ &a_{12}x_2\ &+\ &\ldots\ &+\ &a_{1n}x_n\ &=\ &b_1 \\ 
a_{21}x_1\ &+\ &a_{22}x_2\ &+\ &\ldots\ &+\ &a_{2n}x_n\ &=\ &b_2 \\ 
\vdots\quad &&\vdots\quad &&\ddots\ && \vdots\quad && \vdots\ \\ 
a_{m1}x_1\ &+\ &a_{m2}x_2\ &+\ &\ldots\ &+\ &a_{mn}x_n\ &=\ &b_m,
\end{linear}
\end{equation} whose augmented matrix is 
\[\mtx{rrrr|r}{ a_{11} & a_{12} & \ldots & a_{1n} & b_1 \\a_{21} & a_{22} & \ldots & a_{2n} & b_1 \\ \vdots & \vdots & \ddots & \vdots & \vdots \\ a_{m1} & a_{m2} & \ldots & a_{mn} & b_m   }.\] These two problems have the same solution set.\\

\begin{Thm} Let $\bb x = (x_1, x_2, \ldots, x_m)\in F^m$ be a vector over a field $F$. Then $\bb x$ is a solution to the vector equation \eqref{eq:vectoreqn} if and only if $\bb x$ is a solution to the linear system \eqref{eq:linearsystem}.
\end{Thm}

%NEW %UPDATE ROW OPERATIONS NOTATION
\begin{Exam} Let $\bb a_1 = \vr{1\\2\\3\\4}$, $\bb a_2 = \vr{1\\0\\0\\1}$, and $\bb a_3 = \vr{3\\0\\5\\7}$. Let $\bb b = \vr{-7 \\ 4 \\ -4 \\ -9}$. Is $\bb b$ a linear combination of $\bb a_1, \bb a_2, \bb a_3$?\\

To solve the vector equation, 
\[x_1\vr{1\\2\\3\\4} + x_2 \vr{1\\0\\0\\1} + x_3\vr{3\\0\\5\\7} = \vr{-7 \\ 4 \\ -4 \\ -9},\] we solve the linear system whose augmented matrix is 
\[\mtx{rrr|r}{1&1&3&-7\\2&0&0&4\\3&0&5&-4\\4&1&7&-9}.\] We next row reduce the matrix via the elementary row operations:
\[\mtx{rrr|r}{\fbox{1}&1&3&-7\\2&0&0&4\\3&0&5&-4\\4&1&7&-9}{\color{red}\begin{array}{c} \mbox{}\\ (\text{Row 2} - 2\text{Row 1}) \\ (\text{Row 3} - 3\text{Row 1}) \\ (\text{Row 4} - 4\text{Row 1})\end{array}} 
\sim  \mtx{rrr|r}{\fbox{1}&1&3&-7\\0&\fbox{$-2$}&-6&18\\0&-3&-4&17\\0&-3&-5&19}{\color{red}\begin{array}{c} \mbox{}\\ (-\frac{1}{2}\text{Row 2}) \\ \mbox{} \\ (\text{Row 4} - \text{Row 3})\end{array}}\]
\[\sim  \mtx{rrr|r}{\fbox{1}&1&3&-7\\0&\fbox{1}&3&-9\\0&-3&-4&17\\0&0&-1&2}{\color{red}\begin{array}{c} \mbox{}\\ \mbox{} \\ (\text{Row 3} + 3\text{Row 2}) \\ (-\text{Row 4})\end{array}} 
\sim  \mtx{rrr|r}{\fbox{1}&1&3&-7\\0&\fbox{1}&3&-9\\0&0&\fbox{5}&-10\\0&0&1&-2}{\color{red}\begin{array}{c} \mbox{}\\ \mbox{} \\ (\text{Interchange}) \\ (\text{Interchange})\end{array}} \]
\[\sim  \mtx{rrr|r}{\fbox{1}&1&3&-7\\0&\fbox{1}&3&-9\\0&0&\fbox{1}&-2\\0&0&5&-10}{\color{red}\begin{array}{c} \mbox{}\\ \mbox{} \\ \mbox{} \\ (\text{Row 4} - 5\text{Row 3})\end{array}}
\sim  \mtx{rrr|r}{\fbox{1}&1&3&-7\\0&\fbox{1}&3&-9\\0&0&\fbox{1}&-2\\0&0&0&0}
\]  This last matrix is in echelon form. We will continue to row reduced echelon form. 
\[\sim  \mtx{rrr|r}{\fbox{1}&1&3&-7\\0&\fbox{1}&3&-9\\0&0&\fbox{1}&-2\\0&0&0&0}{\color{red} \begin{array}{c} (\text{Row 1} - 3\text{Row 3})\\  (\text{Row 2} - 3\text{Row 3}) \\ \mbox{} \\ \mbox{}\end{array}} \sim  \mtx{rrr|r}{\fbox{1}&1&0&-1\\0&\fbox{1}&0&-3\\0&0&\fbox{1}&-2\\0&0&0&0}{\color{red} \begin{array}{c} (\text{Row 1} - \text{Row 2})\\  \mbox{} \\ \mbox{} \\ \mbox{}\end{array}} \sim \mtx{rrr|r}{\fbox{1}&0&0&2\\0&\fbox{1}&0&-3\\0&0&\fbox{1}&-2\\0&0&0&0}
\] Therefore, the solution is $(2,-3,-2)$. In fact, we see that 
\[2\vr{1\\2\\3\\4} - 3 \vr{1\\0\\0\\1} - 2\vr{3\\0\\5\\7} = \vr{-7 \\ 4 \\ -4 \\ -9}. \qedhere\]
\end{Exam}\vs

%NEW %UPDATE ROW OPERATIONS NOTATION
\begin{Exam} Let $\bb a_1 = \vr{1\\2\\3}$, $\bb a_2 =\vr{1\\0\\2}$, $\bb a_3 =\vr{2\\2\\4}$, $\bb a_4 = \vr{1\\1\\4}$, and $\bb b = \vr{4\\2\\1}$ be vectors over $\Z_5^3$. Is $\bb b$ a linear combination of $\bb a_1$, $\bb a_2$, $\bb a_3$, $\bb a_4$?\\

To solve the vector equation,
\[x_1\vr{1\\2\\3}+x_2\vr{1\\0\\2} + x_3\vr{2\\2\\4}+ x_4\vr{1\\1\\4} = \vr{4\\2\\1},\] we work with the augmented matrix
\[\mtx{cccc|c}{\fbox{1}&1&2&1&4 \\  2&0&2&1&2\\ 3&2&4&4&1}{\color{red} \begin{array}{c} \mbox{}\\  (\text{Row 2} - 2\text{Row 1}) \\ (\text{Row 3} - 3\text{Row 1}) \end{array}} 
\sim \mtx{cccc|c}{\fbox{1}&1&2&1&4 \\  0&\fbox{3}&3&4&4\\ 0&4&3&1&4}{\color{red} \begin{array}{c} \mbox{}\\  \mbox{} \\ (\text{Row 3} - 3\text{Row 2}) \end{array}}\]
\[\sim \mtx{cccc|c}{\fbox{1}&1&2&1&4 \\  0&\fbox{3}&3&4&4\\ 0&0&\fbox{4}&4&2}{\color{red} \begin{array}{c} \mbox{}\\  \mbox{} \\ (4\text{Row 3}) \end{array}}
\sim \mtx{cccc|c}{\fbox{1}&1&2&1&4 \\  0&\fbox{3}&3&4&4\\ 0&0&\fbox{1}&1&3}{\color{red} \begin{array}{c} (\text{Row 1} -2\text{Row 3})\\  (\text{Row 2} -3\text{Row 3}) \\ \mbox{} \end{array}}\]
\[\sim \mtx{cccc|c}{\fbox{1}&1&0&4&3 \\  0&\fbox{3}&0&1&0\\ 0&0&\fbox{1}&1&3}{\color{red} \begin{array}{c} \mbox{}\\  (2\text{Row 2}) \\ \mbox{} \end{array}}
\sim \mtx{cccc|c}{\fbox{1}&1&0&4&3 \\  0&\fbox{1}&0&2&0\\ 0&0&\fbox{1}&1&3}{\color{red} \begin{array}{c} (\text{Row 1} -\text{Row 2})\\  \mbox{} \\ \mbox{} \end{array}}\]
\[\sim \mtx{cccc|c}{\fbox{1}&0&0&2&3 \\  0&\fbox{1}&0&2&0\\ 0&0&\fbox{1}&1&3}\]

The resulting linear system is then 
\[\begin{linear}
x_1\ &&&+\ &2x_4\ &=\ &3\\
&x_2\ &&+\ &2x_4\ &=\ &0\\
&&x_3\ &+\ &x_4\ &=\ &3\\
\end{linear} \qRightarrow  \begin{linear}
x_1\ &=\ &3x_4\ &+\ &3\\
x_2\ &=\ &3x_4\ \\
x_3\ &=\ &4x_4\ &+\ &3\\
\end{linear}\] Therefore, there are five possibilities: $(3,0,3,0)$, $(1,3,2,1)$, $(4, 1, 1, 2)$, $(2, 4, 0, 3)$, and $(0, 2, 4, 4)$. This gives five possible linear combinations such as:
\[3\vr{1\\2\\3}+ 3\vr{2\\2\\4}  = \vr{1\\2\\3}+3\vr{1\\0\\2} + 2\vr{2\\2\\4}+ \vr{1\\1\\4} = \vr{4\\2\\1}.\qedhere \]
\end{Exam}

\begin{Def}\label{def:linearspan}
Given vectors $\bb v_1, \bb v_2, \ldots, \bb v_n \in F^m$, the \textbf{(linear) span} of $\bb v_1, \bb v_2, \ldots, \bb v_n$, denoted $\Span\{\bb v_1, \bb v_2, \ldots, \bb v_n\}$, is the set of all linear combinations of $\bb v_1, \bb v_2, \ldots, \bb v_n$ inside of $F^m$, that is, 
We say that a subset $S\subseteq F^m$ is \textbf{spanned} by $\bb v_1, \bb v_2, \ldots, \bb v_n$ if $S=\Span\{\bb v_1, \bb v_2, \ldots, \bb v_n\}$ and that $\{\bb v_1, \bb v_2, \ldots, \bb v_n\}$ is a \textbf{spanning set} for $S$. 
\end{Def}\vs

Asking whether a vector $\bb b \in \Span\{\bb a_1, \ldots, \bb a_n\}$ is equivalent to asking whether  there exists a solution $\bb b$ to the vector equation 
\[x_1\bb a_1 +  \ldots + x_n\bb a_n = \bb b.\] 

{\color{red} (These last two examples need to be replaced/updated).}
\begin{Exam} Let $\bb a_1 = \vr{1 \\ -2 \\ -5}$, $\bb a_2 = \vr{2 \\ 5 \\ 6}$, and $\bb b = \vr{7 \\ 4 \\ -3}$. Is $\bb b \in \Span\{\bb a_1, \bb a_2\}$?\\
As explained above, answering the question is equivalent to solving the vector equation $x_1\bb a_1 + x_2\bb a_2 = \bb b$. Likewise, this vector equation can be solved by row reduction on the augmented matrix $\mtx{rr|r}{ \bb a_1 & \bb a_2 & \bb b}$:
\[\mtx{rr|r}{1 & 2 & 7 \\ -2 & 5 & 4 \\ -5 & 6 & -3} \sim \mtx{rr|r}{1 & 2 & 7 \\ 0 & 9 & 18 \\ 0 & 16 & 32} \sim \mtx{rr|r}{1 & 2 & 7 \\ 0 & 1 & 2 \\ 0 & 1 & 2}  \sim \mtx{rr|r}{1 & 0 & 3 \\ 0 & 1 & 2 \\ 0 & 0 & 0}.\] Therefore, the solution is $x_1 = 3$ and $x_2 = 2$, that is, 
\[\bb b  = 3\bb a_1 + 2 \bb a_2 = 3\vr{1 \\ -2 \\ -5} + 2\vr{2 \\ 5 \\ 6} = \vr{7 \\ 4 \\ -3}.\qedhere\]
\end{Exam}\vs

\begin{Exam} Let $\bb a_1 = \vr{1 \\ -2 \\ 3}$, $\bb a_2 = \vr{5 \\ -13 \\ -3}$, and $\bb b = \vr{-3 \\ 8 \\ 1}$. Is $\bb b \in \Span\{\bb a_1, \bb a_2\}$?\\

Like above, we row reduce the augmented matrix $\mtx{rr|r}{ \bb a_1 & \bb a_2 & \bb b}$:
\[\mtx{rr|r}{1 & 5 & -3 \\ -2 & -13 & 8 \\ 3 & -3 & 1} \sim \mtx{rr|r}{1 & 5 & -3 \\ 0 & -3 & 2 \\ 0 & -18 & 10} \sim \mtx{rr|r}{1 & 5 & -3 \\ 0 & -3 & 2 \\ 0 & 0 & -2},\] which is now in echelon form. The third equation is $0=-2$, which implies that the system in inconsistent. Therefore, $\bb b\notin \Span\{\bb a_1, \bb a_2\}$
\end{Exam}


%SOME EXERCISES ARE DIRTY
%%%%%%%%%%%%%%%%%%% Exercises %%%%%%%%%%%%%%%%%%%
\startExercises{span}

\noindent For Exercises \ref{exer:vectoreqnstart}-\ref{exer:vectoreqnstop}, express the given system of equations as a vector equation.
\begin{enumerate}[!HW!, start=1]%, label=$\spadesuit$ \arabic*., ref=\arabic*]
\begin{multicols}{3}
\item\label{exer:vectoreqnstart} $\begin{linear} %Abby Allen
x_1\ &+\ &3x_2 &&&=\ &7\\
7x_1\ &+\ &6x_2\ &-\ &x_3\ &=\ &8\\
3x_1\ &+\ &x_2\ &+\ &2x_3\ &=\ &3
\end{linear}$ 
\itemspade $\begin{linear} %NEW
3x_1\ & +\ &7x_2\ &-\ &5x_3\ &=\ &-5\\
-x_1\ &-\ &2x_2\ &+\ &6x_3\ &=\ &4
\end{linear}$ 
\itemspade $\begin{linear} %NEW
-2x_1\ &+\ &6x_2\ &-\ &11x_3\ &=\ &6\\
x_1\ &-\ &2x_2\ &+\ &5x_3\ &=\ &-3\\
3x_1\ &-\ &19x_2\ &+\ &31x_3\ &=\ &-17
\end{linear}$
\end{multicols}
\begin{multicols}{3}
\item \label{exer:vectoreqnstop} 
$\begin{linear} 
x_1\ &+\ &2x_2\ &+\ &2x_3\ &=\ &1\\
3x_1\ &+\ &x_2\ & & &=\ &4\\
5x_1\ &+\ &2x_2\ &+\ &x_3\ &=\ &3
\end{linear}$ %Ashley Taylor
\end{multicols}
\end{enumerate}

\noindent For Exercises \ref{exer:combocheckstart}-\ref{exer:combocheckstop}, determine if $\bb b$ is a linear combination of the other vectors $\{\bb a_1, \bb a_2, \ldots\}$. If so, write $\bb b$ as a linear combination. The set of vectors $\{\bb a_1, \bb a_2,\ldots\}$ will be listed first, followed by the vector $\bb b$. Answers may vary. 
\begin{enumerate}[!HW!, label=$\spadesuit$ \arabic*., ref=\arabic*]
\begin{multicols}{2}
\item\label{exer:combocheckstart}  $\left\{\vr{-1\\3}, \vr{2\\-2}\right\}$,\ $\vr{1\\5}$ %anon
\item $\left\{\vr{-3\\-1\\1}, \vr{2\\2\\3}\right\}$,\ $\vr{5\\4\\3}$  %anon
\end{multicols}
\end{enumerate}
\begin{enumerate}[!HW!]
\begin{multicols}{2}
\itemspade $\left\{\vr{3\\-2\\1}, \vr{-2\\3\\-3}\right\}$, $\vr{1\\6\\-9}$ %anon
\item $\left\{\mtx{c}{1+i\\1-i\\3-2i}, \mtx{c}{1+2i\\1+i\\-2+i}\right\}$, $\mtx{c}{4+5i\\4-2i\\7-5i}$ %anon
\end{multicols}
\item\label{exer:combocheckstop} $\left\{\vr{1\\-1\\2\\1}, \vr{2\\1\\-1\\1}, \vr{-1\\2\\2\\-1}, \vr{1\\-1\\2\\2}\right\}$,\ $\vr{6\\3\\14\\8}$ %Christopher Newton
\end{enumerate}

\noindent For Exercises \ref{exer:spancheckstart}-\ref{exer:spancheckstop}, determine if $\bb b \in \Span\{\bb a_1, \bb a_2, \ldots\}$. If so, write $\bb b$ as a linear combination of the $\bb a_i$.  The set of vectors $\{\bb a_1, \bb a_2,\ldots\}$ will be listed first, followed by the vector $\bb b$. Answers may vary.
\begin{enumerate}[!HW!]
\item\label{exer:spancheckstart} $\left\{\vr{-4\\9\\6}, \vr{10\\6\\4}, \vr{7\\-12\\-8}\right\}$, $\vr{12\\4\\-6}$ %Kaden Allred
\end{enumerate}
\begin{enumerate}[!HW!, label=$\spadesuit$ \arabic*., ref=\arabic*]
\item\label{exer:spancheckstop} $\left\{\vr{1\\1\\2}, \vr{-2\\-1\\-2}, \vr{-1\\1\\2}\right\}$, $\vr{-4\\-1\\-2}$ %anon
\end{enumerate}

\begin{enumerate}[!HW!]
\itemspade For any list of vectors $\bb v_1,\ldots, \bb v_n$ in $F^m$, show that $\Span\{\bb v_1,\ldots, \bb v_n\}$ contains the zero vector.

\item Suppose that $\bb u_1, \bb u_2\in \Span\{\bb v_1, \bb v_2,\bb v_3\}$. Prove that $\Span\{\bb u_1, \bb u_2\}\subseteq \Span\{\bb v_1, \bb v_2, \bb v_3\}$.

\item Suppose that $\bb v_3 \in \Span\{\bb v_1, \bb v_2\}$. Prove that $\Span\{\bb v_1, \bb v_2\} = \Span\{\bb v_1, \bb v_2, \bb v_3\}$.

\item Prove that $\Span\{\bb v_1, \bb v_2\}=\Span\{\bb v_1, \bb v_1+\bb v_2\}$.

\item Suppose that $\bb u_1, \bb u_2,\ldots, \bb u _k\in \Span\{\bb v_1, \bb v_2, \ldots, \bb v_n\}$. Prove that $\Span\{\bb u_1, \bb u_2, \ldots, \bb u_k\}\subseteq \Span\{\bb v_1, \bb v_2,\ldots,  \bb v_n\}$.

\item Suppose that $\bb v_{n+1} \in \Span\{\bb v_1, \bb v_2,\ldots, \bb v_n\}$. Prove that $\Span\{\bb v_1, \bb v_2, \ldots, \bb v_n\} = \Span\{\bb v_1, \bb v_2, \ldots, \bb v_n, \bb v_{n+1}\}$.

\item \label{hw:changespanner} Let $c_i\in F$ be scalars such that $c_n\neq 0$. Prove that $\Span\{\bb v_1, \bb v_2, \ldots, \bb v_n\} = \Span\left\{\bb v_1, \bb v_2, \ldots, \bb v_{n-1}, \sum_{i=1}^n c_i\bb v_n\right\}.$

\item \label{hw:changespanner2}Suppose that $\bb u\in \Span\{\bb v_1, \bb v_2, \ldots, \bb v_n\}$ such that $\bb u\neq \bb 0$. Then there exists an $i$ such that \[\Span\{\bb v_1, \bb v_2,\ldots, \bb v_{i-1}, \bb v_i, \bb v_{i+1}, \ldots, \bb v\} = \Span\{\bb v_1, \bb v_2,\ldots, \bb v_{i-1}, \bb u, \bb v_{i+1}, \ldots, \bb v\}.\]
\end{enumerate}

%%%%%%%%%%%%%%%%%%% Footnotes %%%%%%%%%%%%%%%%%%%
\pagebreak