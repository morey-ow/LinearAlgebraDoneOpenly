\begin{center} 
\emph{``All mankind... being all equal and independent, no one ought to harm another in his life, health, liberty or possessions.'' -- John Locke}
\end{center}

\section{Linear Independence}\label{sec:independence}
\begin{Def} A set of vectors $\{\bb a_1, \bb a_2, \ldots, \bb a_n\}\subseteq F^m$ is said to be \textbf{linearly independent} if the vector equation 
\[x_1\bb a_1 + x_2\bb a_2 + \ldots + x_n\bb a_n =  \bb 0\] has only the trivial solution. Otherwise, the set is said to be \textbf{linearly dependent}. 
\end{Def}

In other words, a set of vectors is linearly dependent if there exists some weights $c_1, c_2, \ldots, c_n$, with at least one not zero, such that 
\[c_1\bb a_1 + c_2\bb a_2 + \ldots + c_n\bb a_n = \bb 0.\] The previous equation is then called a \textbf{linear dependence relation}.\\

Expressed another way, suppose $A = \mtx{rrrr}{ \bb a_1 & \bb a_2 & \ldots & \bb a_n}$, then $\{\bb a_1, \ldots, \bb a_n\}$ is linearly independent if and only if $A\bb x = \bb 0$ has no nontrivial solution, since 
\[A\bb x = \mtx{rrrr}{ \bb a_1 & \bb a_2 & \ldots & \bb a_n}\mtx{c}{x_1 \\ x_2 \\ \vdots \\ x_n} = x_1\bb a_1 + x_2\bb a_2 + \ldots + x_n\bb a_n =  \bb 0.\] The set $\{\bb a_1, \ldots, \bb a_n\}$ is linearly dependent if and only if $A\bb x = \bb 0$ has a nontrivial solution.\\

\begin{Exam}\label{exam:lineardependent} Let $\bb v_1 = \vr{1 \\ 2 \\3}$, $\bb v_2 = \vr{3\\1\\-2}$, and $\bb v_3 = \vr{-3\\4\\13}$.\\

\begin{enumerate}
\item Determine if the set $\{\bb v_1, \bb v_2, \bb v_3\}$ is linearly independent.\\

To determine if the set is linearly independent, it suffices to compute the echelon form of the corresponding homogeneous system. In fact, the final row of zeros is irrelevant.
\[\mtx{rrr}{1&3&-3\\2&1&4\\3&-2&13} \sim \mtx{rrr}{1&3&-3\\0&-5&10\\0&-11&22 } \sim \mtx{rrr}{1&3&-3\\0&1&-2\\0&1&-2 }\sim \mtx{rrr}{1&3&-3\\0&1&-2\\0&0&0 }.\] Thus, $x_3$ is a free variable of the system, which implies that the homogeneous equation $A\bb x = \bb 0$ does have a nontrivial solution. Therefore, $\{\bb v_1, \bb v_2, \bb v_3\}$ is linearly dependent.\\

\item Find a linear dependence relation among $\bb v_1$, $\bb v_2,$ and $\bb v_3$.

To determine a dependence relation, we finish the row reduction from above.
\[\mtx{rrr}{1&3&-3\\0&1&-2\\0&0&0 } \sim \mtx{rrr}{1&0&3\\0&1&-2\\0&0&0 }.\] Therefore, 
\[\left\{\begin{alignedat}{100}
&x_1\ && & +\ &3x_3\ &=\ &0&\\
&&& x_2\ & -\ &2x_3\ &=\ &0&\\
&&&&&0\ &=\ &0&
\end{alignedat}\right. 
\] Thus, $\bb x = \vr{x_1\\x_2\\x_3} = \vr{-3x_3\\2x_3 \\ x_3} = x_3\vr{-3\\2\\1}$. Thus, $-3\bb v_1 +2 \bb v_2 + \bb v_3 = \bb 0$ (corresponding to $x_3=1$). Of course, there are infinitely many dependence relations. For example, $-15\bb v_1 + 10\bb v_2 + 5\bb v_3 = \bb 0$ (corresponding to $x_3 = 5$).\\

We also note that we have shown that $\bb v_3\in \Span\{\bb v_1, \bb v_2\}$ since $-3\bb v_1 +2 \bb v_2 + \bb v_3 = \bb 0$ implies $\bb v_3 = 3\bb v_1 - 2\bb v_2$. This observation is always the case for linear dependence.\hfill$\qedhere$
\end{enumerate}
\end{Exam}\vs

\begin{Exam} Determine if the columns of the matrix $A = \mtx{rrr}{1&-1&-1\\-1&2&4\\2&-4&-7}$ are linearly independent.\\

It suffices to show that $A\bb x = \bb 0$ has no nontrivial solutions, that is, that an echelon form of $A$ has no zero rows.
\[\mtx{rrr}{\fbox{$1$}&-1&-1\\-1&2&4\\2&-4&-7}\sim \mtx{rrr}{\fbox{$1$}&-1&-1\\0&\fbox{$1$}&3\\0&-2&-5}\sim \mtx{rrr}{\fbox{$1$}&-1&-1\\0&\fbox{$1$}&3\\0&0&\fbox{$1$}}.\] Therefore, the columns of $A$ are linearly independent.
\end{Exam}\vs

These examples illustrate an important point: a set of vector is linearly dependent if and only if the corresponding homogeneous linear system has a free variable. In other words, a set of vectors is linearly independent if and only if the associated coefficient matrix has a pivot in each column.\\

The following properties provided, in some cases, a quick test to determine the linear independence or dependence of a set of vectors.\\

\begin{Thm}\label{prop:indTest1}\mbox{}
\begin{enumerate}
\item If a set contains a linearly dependent subset, then the set itself is linearly dependent.\\
\item The set $\{\bb v\}$ is linearly independent if and only if $\bb v \neq \bb 0$.\\
\item If a set $S \subseteq F^n$ contains the zero vector, then $S$ is linearly dependent.\\
\item A set $S = \{\bb v_1, \ldots, \bb v_p\}\subseteq F^n$  for $p>1$ is linearly dependent if and only if at least one of the vectors of $S$ is a linear combination of the others.\\
\item The set $\{\bb u, \bb v\}$ is linearly independent if and only if neither vector is a multiple of the other, that is $\bb v \neq c\bb u$ nor $\bb u \neq c\bb v$ for any $c\in F$.\\
\item Suppose $S = \{\bb v_1, \ldots, \bb v_p\} \subseteq F^n$. If $p > n$, then $S$ is linearly dependent.\\
\end{enumerate}
\end{Thm}
\begin{proof}\mbox{}
\begin{enumerate}
%\item This was addressed above.\\
\item Suppose $S$ is linearly dependent and $S\subseteq T$. Then set every vector in $T$ which is not in $S$ to have weight zero. Then a dependence relation on $S$ becomes a dependence relation on $T$.\\
\item Note that  $c\bb v = \bb 0$ if and only if $c = 0$ or $\bb v = \bb 0$.\\
\item It follows from the previous two parts. \\%Set the weight of the zero vector to one and set the weight of any other vector in $S$ to zero. This gives a dependence relation.
\item Suppose first that $S$ is linearly dependent and \[c_1\bb v_1 + \ldots + c_p\bb v_p = \bb 0\] is a dependence relation. Now there exists some $j$ such that $c_j\neq 0$. Without the loss of generality, we may suppose that $j=p$. Then 
\begin{eqnarray*}
c_1\bb v_1 + \ldots + c_{p-1}\bb v_{p-1} &=& -c_p\bb v_p\\
\frac{c_1}{-c_p}\bb v_1 + \ldots + \frac{c_{p-1}}{-c_p}\bb v_{p-1} &=& \bb v_p.
\end{eqnarray*} So, $\bb v_p \in \Span\{\bb v_1, \ldots \bb v_{p-1}\}$. \\

Conversely, if $v_p = c_1\bb v_1 + \ldots + c_{p-1}\bb v_{p-1}$, then \[c_1\bb v_1 + \ldots + c_{p-1}\bb v_{p-1} - v_p = \bb 0\] is a dependence relation, that is, $\{\bb v_1, \ldots, \bb v_p\}$ is linearly dependent.

\item This follows immediately from the previous part when $p=2$. \\ %Suppose that $\{\bb u, \bb v\}$ is linearly dependent. If $a\bb u + b\bb v = \bb 0$, then $a\bb u = - b\bb v$. If this is a dependence relation, then $a$ or $b$ is nonzero. Without the loss of generality, suppose that $a\neq 0$. Then $\bb u  = -\frac{b}{a}\bb v$. The converse is similar.

\item Let $A = \mtx{rrr}{\bb v_1 & \ldots & \bb v_p}$. Then $A$ is an $n\times p$ matrix and the equation $A\bb x = \bb 0$ corresponds to an under-determined system of $n$ equation with $p$ unknowns. Thus, the system has a free variable by \thmref{prop:1.1determine}, which implies the linear dependence of the column vectors. %If $p > n$, then there are more variables than equations. So there must be a free variable (Think about what would happen after you row reduced this matrix). Hence, $A\bb x = \bb 0$ has a nontrivial solution and the columns are linearly dependent.
\hfill$\qedhere$
\end{enumerate}
\end{proof}\vs

\begin{Exam} Determine by inspection if the given sets are linearly independent.
\begin{enumerate}
\begin{multicols}{2}
\item $\left\{\vr{7\\1\\6}, \vr{9\\0\\2}, \vr{4\\2\\5}, \vr{3\\1\\9}\right\}$.\\

This set of vectors is linearly dependent because any set of 4 vectors in $\R^3$ must be dependent.\\
\end{multicols}
\begin{multicols}{2}
\item $\left\{\vr{3\\1\\5}, \vr{0\\0\\0}, \vr{2\\1\\7}\right\}$.\\

This set of vectors is linearly dependent because it contains the zero vector.\\
\end{multicols}\pagebreak
\begin{multicols}{2}
\item $\left\{\vr{15\\0\\20\\-5}, \vr{-12\\0\\-16\\4}\right\}$.

This set of vectors is linearly dependent because the second vector is just $(-5/4)$ times the first vector.\vfill\hfill$\qedhere$
\end{multicols} 
\end{enumerate}
\end{Exam}

%%%%%%%%%%%%%%%%%%% Exercises %%%%%%%%%%%%%%%%%%%
\startExercises{independence}
\noindent For Exercises \ref{true:independencestart}-\ref{true:independencestop}, determine with the statement is true or false. If false, correct the statement so that it is true.
\begin{enumerate}[!HW!,start=1]
\item\label{true:independencestart} If a set $S\subseteq F^n$ contains the zero vector, then $S$ is linearly independent. %Carson Blickenstaff
\item If a set contains a linearly dependent subset, then the set itself is linearly dependent. %Carson Blickenstaff
\item The set $\{\bb u, \bb v\}$ is linearly independent if and only if neither vector is a multiple of the other, that is, $\bb v \neq c\bb u$ and $\bb u \neq c\bb v$ for any $c\in F$. %Carson Blickenstaff
\item Suppose $S=\{\bb v_1,\ldots, \bb v_p\}\subseteq F^n$. If $p>n$, then $S$ is linearly dependent.  %Carson Blickenstaff
\item The set $\{\bb v\}$ is linearly independent if and only if $\bb v\neq 0$.  %Carson Blickenstaff
\item\label{true:independencestop} A set $S$ of two or more vectors is linearly dependent if and only if at least one of the vectors in $S$ is a linear combination of the others.\\  %Carson Blickenstaff
\end{enumerate}

\noindent QUICK! For Exercises \ref{exer:quickLIrealstart}-\ref{exer:quickLIrealstop}, determine whether of the set of vectors is linearly independent or dependent using \thmref{prop:indTest1} in LESS THAN 10 SECONDS! Justify your quick response. 
\begin{enumerate}[!HW!, label=$\spadesuit$ \arabic*., ref=\arabic*]
\begin{multicols}{2}
\item\label{exer:quickLIrealstart} $\left\{\vr{1\\2\\3}, \vr{0\\1\\4}, \vr{0\\0\\0}\right\}$
\item $\left\{\mtx{c}{1\\i}, \vr{1\\0}\right\}$
\end{multicols}
\begin{multicols}{2}
\item $\left\{\vr{1\\2\\3\\4}\right\} \pmod{11}$
\item\label{exer:quickLIrealstop} \mbox{$\left\{\vr{1\\0\\1\\1}, \vr{0\\0\\0\\1}, \vr{1\\1\\1\\1}, \vr{0\\1\\1\\0}, \vr{1\\0\\0\\1}\right\} \pmod 2$}
\end{multicols}
\end{enumerate}

\noindent For Exercises \ref{exer:LIrealstart}-\ref{exer:LIrealstop}, determine if the set of vectors is linearly independent or dependent. If linearly dependent, provide a nontrivial dependency relation. 
\begin{enumerate}[!HW!]
\begin{multicols}{2}
\item\label{exer:LIrealstart} $\left\{\vr{1\\0\\2},\vr{2\\2\\4},\vr{3\\6\\5}\right\}$ %Abby Allen
\itemspade $\left\{\vr{1\\-1\\3},\vr{2\\-2\\6},\vr{-2\\3\\-2}\right\}$
\end{multicols}
\begin{multicols}{2}
\itemspade $\left\{\vr{1\\-2\\-1\\-1}, \vr{-1\\2\\2\\4}, \vr{2\\-3\\-3\\2}\right\}$
\item $\left\{\vr{2\\4\\6},\vr{1\\3\\2},\vr{4\\1\\3}\right\}$ %Hailey Checketts
\end{multicols}
\begin{multicols}{2}
\itemspade $\left\{\mtx{r}{i\\1\\-i}, \mtx{c}{2\\1-2i\\-1+i}, \mtx{c}{0\\i\\i}\right\}$
\itemspade $\left\{\vr{1\\3\\1}, \vr{1\\0\\0},\vr{1\\2\\2}\right\} \pmod 5$
\end{multicols}
\end{enumerate}
\begin{enumerate}[!HW!,label=$\spadesuit$ \arabic*., ref=\arabic*]
%\begin{multicols}{2}
\item%spade
\label{exer:LIrealstop} $\left\{\vr{1\\2\\0\\1}, \vr{2\\6\\6\\4}, \vr{6\\5\\1\\1}, \vr{0\\5\\4\\4}\right\} \pmod 7$
%\end{multicols}
\end{enumerate}

%%%%%%%%%%%%%%%%%%% Footnotes %%%%%%%%%%%%%%%%%%%
\pagebreak