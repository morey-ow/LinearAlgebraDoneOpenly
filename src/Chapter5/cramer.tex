\begin{center} 
\emph{``Desires dictate our priorities, priorities shape our choices, and choices determine our actions. The desires we act on determine our changing, our achieving, and our becoming.'' -- Dallin H. Oaks}
\end{center}

\section{Cramer's Rule}\label{sec:cramer}
\begin{Def} Let $A = \mtx{ccc}{\bb a_1 & \ldots & \bb a_n}$ be an $n\times n$ matrix and let $\bb b \in F^n$. Then 
\[A_i(\bb b) = \mtx{ccccc}{\bb a_1 & \ldots & \bb b & \ldots & \bb a_n},\] where $\bb b$ replaced the $i$th column vector of $A$.
\end{Def}\vs

\begin{Thm}[Cramer's Rule] Let $A$ be a $n\times n$ nonsingular matrix. For any $b\in F^n$, the unique solution $\bb x$ of $A\bb x = \bb b$ has entries given by 
\[x_i = \dfrac{\det A_i(\bb b)}{\det A}.\]
\end{Thm}
\begin{proof}
Let $I$ denote the $n\times n$ identity matrix, with column vectors $\bb e_i$. If $A\bb x = \bb b$, then 
\begin{eqnarray*}
A\cdot I_i(\bb x) &=& A\mtx{ccccc}{\bb e_1 & \ldots & \bb x & \ldots & \bb e_n} =  \mtx{ccccc}{A\bb e_1 & \ldots & A\bb x & \ldots & A\bb e_n}\\
&=& \mtx{ccccc}{\bb a_1 & \ldots & \bb b & \ldots & \bb a_n} = A_i(\bb b).
\end{eqnarray*} Therefore, 
\[\det(A)\det(I_i(\bb x)) = \det(A\cdot I_i(\bb x)) = \det A_i(\bb b).\] But $\det I_i(\bb x) = x_i$, by row reduction along the $i$th column of $I_i(\bb x)$. Therefore, \[\det(A)\cdot x_i = \det A_i(\bb b),\] which finishes the proof.
\end{proof}\vs

Note that $ \det(A)x_i = \det A_i(\bb b)$ holds even if $\det A = 0$.\\

\begin{Exam} Use Cramer's rule to solve the system \\
$\left\{\begin{alignedat}{100}
&&3x_1\ &-\ &2x_2\ &=\ &6&\\
&-&5x_1\ &+\ &4x_2\ & =\ &8.&
\end{alignedat}\right.$\\

Let $A = \mtx{rr}{3&-2\\-5&4}$ and $\bb b = \vr{6\\8}$.  Thus, 
\[\det A = \dtx{rr}{3&-2\\-5&4} = 12-10 = 2,\quad \det A_1(\bb b) = \dtx{rr}{6&-2\\8&4} = 24+16 = 40,\quad \det A_2(\bb b) = \dtx{rr}{3&6\\-5&8} = 24+30 = 54.\] 
Then by Cramer's rule, the unique solution is given as
\[\bb x = \vr{x_1\\x_2} = \vr{\det A_1(\bb b)/\det A \\ \det A_2(\bb b)/\det A} = \vr{40/2\\ 54/2} = \vr{20\\27}. \qedhere\]
\end{Exam}\vs

\begin{Def} Let $A$ be an $n\times n$ matrix. Then the \textbf{adjugate} (or \textbf{adjoint}) of $A$, denoted $\adj A$, is given as 
\[\adj A = \mtx{c}{C_{ji}} = \mtx{cccc}{C_{11} & C_{21} & \ldots & C_{n1} \\ C_{12} & C_{22} & \ldots & C_{n2} \\ \vdots & \vdots & & \vdots \\ C_{1n} & C_{2n} & \ldots & C_{nn}}\]
\end{Def}\vs

Please note that the adjugate matrix is the \emph{transpose} of the matrix of cofactors.\\

\begin{Thm}\label{thm:adjugate} Let $A$ be an $n\times n$ matrix. Then $A\cdot \adj(A) = \adj(A) \cdot A = \det(A)I_n$. In particular, if $A$ is nonsingular then 
\[A^{-1} = \dfrac{1}{\det A}\adj A\]
\end{Thm}
\begin{proof}
The $j$th columns of $A^{-1}$ is a vector $\bb x$ such that \[A\bb x = \bb e_j.\] By Cramer's rule, the $i$th entry in $\bb x$ is given as
\[x_i = \dfrac{\det A_i(\bb e_j)}{\det A}.\] But by cofactor expansion across the $i$th column, we see that \[\det A_i(\bb e_j) = (-1)^{i+j}\det A_{ji} = C_{ji}.\] Therefore, the $(i,j)$ entry in $A^{-1}$ is $\dfrac{\det A_i(\bb e_j)}{\det A} =  \dfrac{C_{ji}}{\det A}$, which finishes the proof.
\end{proof}\vs

In particular, 
\[A\cdot \adj A = \adj A\cdot A = \det A\cdot I_n.\]

Note that is formula holds even if $\det A = 0$.\\

\begin{Exam} Find the inverse of the matrix $A = \mtx{rrr}{2&1&3\\1&-1&1\\1&4&-2}$.\\

We begin by finding the nine cofactors:
\[\begin{alignedat}{100}
&C_{11}\ &=\ &+\dtx{rr}{-1&1\\4&-2}\ &=\ &-2,\quad &C_{12}\ &=\ &-\dtx{rr}{1&1\\1&-2}\ &=\ &3,\quad &C_{13}\ &=\ &+\dtx{rr}{1&-1\\1&4}\ &=\ &5&\\
&C_{21}\ &=\ &-\dtx{rr}{1&3\\4&-2}\ &=\ &14,\quad &C_{22}\ &=\ &+\dtx{rr}{2&3\\1&-2}\ &=\ &-7,\quad &C_{23}\ &=\ &-\dtx{rr}{2&1\\1&4}\ &=\ &-7&\\
&C_{31}\ &=\ &+\dtx{rr}{1&3\\-1&1}\ &=\ &4,\quad &C_{32}\ &=\ &-\dtx{rr}{2&3\\1&1}\ &=\ &1,\quad &C_{33}\ &=\ &+\dtx{rr}{2&1\\1&-1}\ &=\ &-3&
\end{alignedat}\] Therefore, 
\[\adj A = \mtx{rrr}{-2 & 14 & 4\\ 3 & -7 & 1 \\ 5 & -7 & -3}.\] To finish, we need to compute $\det A$. We could compute it directly like in the previous sections, but instead we use the observation that $\adj A \cdot A = \det A\cdot I_n$. 
\[\adj A \cdot A = \mtx{rrr}{-2 & 14 & 4\\ 3 & -7 & 1 \\ 5 & -7 & -3}\mtx{rrr}{2&1&3\\1&-1&1\\1&4&-2} = \mtx{rrr}{14&0&0\\0&14&0\\0&0&14}.\] Thus, $\det A = 14$ and 
\[A^{-1} = \mtx{rrr}{-1/7 & 1 & 2/7\\ 3/14 & -1/2 & 1/14 \\ 5/14 & -1/2 & -3/14}. \qedhere\]
\end{Exam}\vs

In practice, it is not very practical to solve linear systems or compute inverse via Cramer's rule. The method of row-reduction is generally more efficient. On the other hand, the application of Cramer's rule and adjugate matrices is immeasurable in the theory of linear algebra.\\

For example, if $A$ is an integer matrix, that is, all of its entries are integers, then its determinant and cofactors will all be integers too. After all, determinants are calculated using addition, subtraction, and multiplication. No division required! Thus, $\adj A$ will be an integer matrix too. Thus, if $\det(A) =\pm 1$, we see that $A^{-1}$ will be an integer matrix as well. This fact is very useful for instructor who want to exercise homework questions with ``cute'' answers so that students feel happy about their linear algebra homework.\\

%%%%%%%%%%%%%%%%%% Exercises %%%%%%%%%%%%%%%%%%%
\startExercises{cramer}

\noindent For Exercises \ref{exer:cramerrulestart}-\ref{exer:cramerrulestop}, solve the linear system using Cramer's Rule.
\begin{enumerate}[!HW!, start=1]
\begin{multicols}{3}
\item\label{exer:cramerrulestart} \mbox{$\mtx{rr}{6&1\\2&4}\bb x \equiv \vr{1\\4} \pmod 7$} %Samuel Andersen
\itemspade $\mtx{rr}{1&3\\2&-3}\bb x = \vr{5\\1}$
\itemspade $\mtx{rr}{2&3\\4&2}\bb x \equiv \vr{1\\3} \pmod 5$
\end{multicols}
\begin{multicols}{2}
\item $\mtx{rrr}{1&3&5\\0&-2&3\\3&2&0}\bb x = \vr{2\\1\\2}$
\item\label{exer:cramerrulestop} $\mtx{rrrr}{1&2&0&1\\0&1&2&1\\2&1&1&2\\0&2&2&1}\bb x \equiv \vr{2\\0\\1\\1} \pmod 3$
\end{multicols}
\end{enumerate}

\noindent For Exercises \ref{exer:adjugatestart}-\ref{exer:adjugatestop}, compute the adjugate of the matrix $A$ below. Verify that $A(\adj A) = \det(A)I_n$.
\begin{enumerate}[!HW!, label=$\spadesuit$ \arabic*., ref=\arabic*]
\begin{multicols}{2}
\item\label{exer:adjugatestart} $\mtx{rr}{1&3\\2&-3}$
\item $\mtx{rr}{2&3\\4&2} \pmod 5$ 
\end{multicols}
\begin{multicols}{2}
\item $\mtx{rrr}{1&3&5\\0&-2&3\\3&2&0}$
\item\label{exer:adjugatestop} $\mtx{rrrr}{1&2&0&1\\0&1&2&1\\2&1&1&2\\0&2&2&1} \pmod 3$
\end{multicols}
\end{enumerate}

\noindent QUICK! For Exercises \ref{exer:quickadjstart}-\ref{exer:quickadjstop}, using \thmref{thm:adjugate}, the given matrix $A$ and its adjugate matrix $\adj(A)$, calculate the determinant and inverse matrix of $A$ in LESS THAN 60 SECONDS!
\begin{enumerate}[!HW!]
\item\label{exer:quickadjstart} $A=\mtx{rrr}{1&2&3\\-1&-1&1\\2&1&-5}$,\quad $\adj(A) = \mtx{rrr}{4&13&5\\-3&-11&-4\\1&3&1}$ %Daven Triplett
\item $A=\mtx{rrr}{-6&8&13\\6&-8&-12\\2&-2&-4}$,\quad $\adj(A) = \mtx{rrr}{8&6&8\\0&-2&6\\4&4&0}$ %Daven Triplett
\item\label{exer:quickadjstop} $A=\mtx{rrr}{6&-2&4\\-3&-1&1\\-4&5&5}$,\quad $\adj(A) = \mtx{rrr}{-10&30&2\\11&46&-18\\-19&-22&-12}$ %Daven Triplett
\end{enumerate}

%%%%%%%%%%%%%%%%%%% Footnotes %%%%%%%%%%%%%%%%%%%
 \mbox{}\vfill
 
\pagebreak