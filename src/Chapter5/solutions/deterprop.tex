\startSolutions{deterprop}{Properties of Determinants}

\begin{enumerate}[!HW!, start=1]
\begin{multicols}{6}
\itemspade $-2$
\itemspade $10$
\itemspade $6$
\itemspade $52$
\item $52$ %Hannah Simonson 
\item $-9+i$ %Hannah Simonson 
\end{multicols}
\begin{multicols}{6}
\itemspade $4$
\item $24$ %Jacob Kuhn 
\item $0$ %Hannah Simonson
\item $126$ %Jacob Kuhn 
\itemspade $1$
\item $27$ %Jacob Kuhn 
\end{multicols}
\begin{multicols}{5}
\itemspade $6$\columnbreak
\itemspade $127$\columnbreak
\item $729$\columnbreak %Devan Triplett
\itemspade \mbox{$\rank(A)=5$,}\\ \mbox{$\nullity(A)=0$}\columnbreak
\itemspade \mbox{$\corank(A)=5$,}\\ \mbox{$\conullity(A)=0$}
\end{multicols}
\itemspade Yes, it is consistent for any $\bb b$. $A\bb x=\bb b$ has no free variables. $A\bb x=\bb b$ has a unique solution.
\itemspade $I_5$
\itemspade $\ker(T)=\{\bb 0\}$, $\im(T) = \R^5$. Yes, $T$ is both one-to-one and onto.
\item \begin{proof} %Mitchell Zufelt
Let $A = \mtx{rr}{a&b\\c&d}$  and $B = \mtx{rr}{e&f\\g&h}$. Then $\det(A) = ad-bc$ and $\det(B)=eh-fg$. Hence,
\[\det(A)\det(B) = (ad-bc)(eh-fg) = adeh  - adfg-bceh+bcfg.\] On the other hand, $AB=\mtx{rr}{ae+bf&af+bh\\ ce+dg&cf+dh}$. Hence, \begin{multline*}\det(AB) = (ae+bf)(cf+dh)-(af+bh)(ce+dg) = aecf+aedh+bfch+bfdh-afce-afdg-bhce-bhdg\\ = adeh  - adfg-bceh+bcfg = \det(A)\det(B).\qedhere \end{multline*}
\end{proof}
\end{enumerate}

\vspace{-15 pt}