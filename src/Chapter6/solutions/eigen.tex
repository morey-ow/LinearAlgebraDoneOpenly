\startSolutions{eigen}{Eigenvalues and Eigenvectors}

\begin{enumerate}[!HW!, start=1]
\begin{multicols}{6}
\item yes,\\ $\lambda=5$ 
\item no
\item no 

\itemspade yes,\\ $\lambda = 2$
\itemspade no
\itemspade yes,\\ $\lambda=-1$
\end{multicols}

\begin{multicols}{6}
\itemspade yes,\\ $\lambda = 1$
\itemspade yes,\\ $\lambda = 0$
\itemspade no

\itemspade $c\vr{1\\0}$
\itemspade $c\vr{3\\1}$
\itemspade $c\vr{1\\1}$
\end{multicols}

\begin{multicols}{5}
\itemspade yes
\itemspade yes
\itemspade no
\end{multicols}

\begin{multicols}{3}
\itemspade $\lambda = 5$, $\text{mult} = 1$,\\
$\left\{\vr{2\\-4\\1}\right\}$;\\
$\lambda = -3$, $\text{mult} = 1$,\\ $\left\{\vr{1\\2\\0}\right\}$ \columnbreak

\itemspade $\lambda = -3$, $\text{mult} = 1$,\\
$\left\{\vr{-1\\2\\1}\right\}$;\\
$\lambda = 2$, $\text{mult} = 2$,\\
$\left\{\vr{1\\0\\2}, \vr{-1\\2\\0}\right\}$ \columnbreak

\item $\lambda=1$, $\text{mult}=1$,\\ 
$\left\{\vr{1\\0\\0}\right\}$;\\
$\lambda=2$, $\text{mult}=1$,\\ $\left\{\vr{1\\2\\1}\right\}$ %Yucheng Long
\end{multicols}

\item Hint: If $A\bb x = \lambda \bb x$, then what can we say about $A^2\bb x = A(A\bb x)$? Can we generalize this observation?
%\begin{proof}
%Since $A\bb x=\lambda \bb x$, we know recursively that $A^m\bb x = A^{m-1}(A\bb x) = A^{m-1}(\lambda \bb x) = \lambda(A^{m-1}\bb x) = \lambda(A^{m-2})(A\bb x) = \lambda(A^{m-2})(\lambda \bb x) = \lambda^2(A^{m-2}\bb x) =\ldots = \lambda^m\bb x$. Therefore, $\lambda^m$ is an eigenvalue of $A$ with associated eigenvector $\bb x$.
%\end{proof}

\item Hint: If $A$ is idempotent, then $A^2=A$. Use Exercise \ref{exer:squareeigenvalue}.
%\begin{proof}
%Since $A$ is idempotent, we know that $A^2=A$. Then $\lambda\bb x = A\bb x = A^2\bb x = \lambda^2\bb x$. So, $(\lambda^2-\lambda)\bb x = \bb 0$. Since $\bb x\neq \bb 0$, we conclude that $\lambda^2-\lambda = \lambda(\lambda - 1)=0$, which implies that $\lambda =0, 1$.
%\end{proof}

\item Hint: If $A$ is nilpotent, then $A^n=0$ for some $n$. Use Exercise \ref{exer:squareeigenvalue}.
%\begin{proof}
%Since $A$ is nilpotent, we know that $A^n=0$. Then $0 = 0\bb x = A^n\bb x = \lambda^n\bb x$. Since $\bb x\neq \bb 0$, we conclude that $\lambda^n=0$, which implies that $\lambda =0$.
%\end{proof}
\end{enumerate}