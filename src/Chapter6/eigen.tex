\begin{center} 
\emph{``Try not to become a man of success, but rather try to become a man of value.'' -- Albert Einstein}
\end{center}

\section{Eigenvalues and Eigenvectors}\label{sec:eigen}
\begin{Exam} Let $A=\mtx{rr}{3&-2\\1&0}$, $\bb u = \vr{-1\\1}$, and $\bb v = \vr{2\\1}$. Then 
\[A\bb u = \mtx{rr}{3&-2\\1&0}\vr{-1\\1} = \vr{-5\\-1}\qquad\text{and}\qquad A\bb v =\mtx{rr}{3&-2\\1&0}\vr{2\\1} = \vr{4\\2} = 2\bb v.\] Notice that while multiplication by $A$ sends $\bb u$ off to who-knows-where, multiplication by $A$ only stretches the vector $\bb v$. This chapter will explore more deeply this type of phenomenon.
\end{Exam}\vs

\begin{Def} Let $A$ be an $n\times n$ matrix. Then a nonzero vector $\bb x$ is an \textbf{eigenvector} of $A$ if there is a scalar $\lambda$ such that
\[A\bb x = \lambda \bb x.\] In this case, $\lambda$ is called an \textbf{eigenvalue} of $A$ corresponding to $\bb x$.
\end{Def}\vs

It is straight forward to check if a vector is an eigenvector.\\

\begin{Exam} Let $A = \mtx{rr}{1&6\\5&2}$, $\bb u = \vr{6\\-5}$, and $\bb v = \vr{3\\-2}$. Then 
\[A\bb u = \mtx{rr}{1&6\\5&2}\vr{6\\-5} = \vr{-24\\20} = -4\vr{6\\-5} = -4\bb u.\] On the other hand,
\[A\bb v = \mtx{rr}{1&6\\5&2}\vr{3\\-2} = \vr{-9\\11} \neq \lambda\vr{3\\-2},\] since 
\[\left\{\begin{alignedat}{100}
&-&9\lambda\ &=\ &&3&\\
&&11\lambda\ &=\ &-&2&
\end{alignedat}\right.\] has no solution. Therefore, $\bb u$ is an eigenvector of $A$ with eigenvalue $\lambda = -4$. The vector $\bb v$ is not an eigenvector of $A$.
\end{Exam}\vs

It can also be checked whether a scalar is an eigenvalue of a matrix, although this requires row reduction.\\

\begin{Exam} Let $A$ be the same matrix as the previous example. Is $7$ an eigenvalue of $A$?\\

Now, suppose that $\lambda$ is an eigenvalue of $A$ with eigenvector $\bb x$. Then 
\begin{eqnarray*}
A\bb x &=& \lambda \bb x\\
A\bb x - \lambda\bb x &=& \bb 0\\
A\bb x - \lambda I\bb x &=& \bb 0\\
(A-\lambda I)\bb x &=& \bb 0
\end{eqnarray*} Therefore, $\lambda = 7$ is an eigenvalue if and only if $A-7I$ has a nontrivial solution to the homogeneous system $(A-7I)\bb x = \bb 0$. Now, 
\[A-7I = \mtx{rr}{1&6\\5&2} - \mtx{rr}{7&0\\0&7} = \mtx{rr}{-6&6\\5&-5}.\] But, 
\[\mtx{rr|r}{-6&6&0\\5&-5&0} \sim \mtx{rr|r}{1&-1&0\\5&-5&0} \sim \mtx{rr|r}{1&-1&0\\0&0&0}.\] Therefore, $\bb x = \vr{1\\1}$ is a nontrivial solution to this homogeneous system and hence an eigenvector of $A$. Note that 
\[A\bb x = \mtx{rr}{1&6\\5&2}\vr{1\\1} = \vr{7\\7} = 7\vr{1\\1} = 7\bb x.\] Therefore, $7$ is an eigenvalue of $A$ with eigenvector $\vr{1\\1}$.
\end{Exam}\vs

Notice in the last example, that $\bb x = \vr{1\\1}$ was an eigenvector for $A$ with eigenvalue $7$. Now, this is not the only eigenvector for $7$. In fact, $\bb y = \vr{2\\2}$ is also an eigenvector. Note that $A\bb y = \vr{14\\14} = 7\bb y$. Now, $\bb y = 2\bb x$. In fact, any nonzero multiple of $\bb x$ is an eigenvector of $A$ since 
\[A(c\bb x) = c(A\bb x) = c(\lambda x) = \lambda(c\bb x).\] Furthermore, any nontrivial solution to the homogeneous system $(A-\lambda I)\bb x  = \bb 0$ is an eigenvector. But this is the null space of $(A-\lambda I)$, a subspace of $\R^n$. This leads to the next definition.\\

\begin{Def} Let $A$ be an $n\times n$ matrix with eigenvalue $\lambda$. Then the null space of $(A-\lambda I)$ is called the \textbf{eigenspace} of $A$ corresponding to $\lambda$. The dimension of the eigenspace is called the \textbf{geometric multiplicity} of the eigenvalue $\lambda$ and corresponds to the nullity of the matrix $A-\lambda I$.
\end{Def}\vs

Since eigenspaces are null spaces of a matrix, they are always subspaces of $F^n$.\\

\begin{Exam} Let $A = \mtx{rrr}{4&-1&6\\2&1&6\\2&-1&8}$. An eigenvalue of $A$ is $\lambda =2$. Find a basis for the eigenspace corresponding to $\lambda =2$.\\

We form 
\[A-2I = \mtx{rrr}{4&-1&6\\2&1&6\\2&-1&8} - \mtx{rrr}{2&0&0\\0&2&0\\0&0&2} = \mtx{rrr}{2&-1&6\\2&-1&6\\2&-1&6}.\] Thus, 
\[\mtx{rrr}{2&-1&6\\2&-1&6\\2&-1&6} \sim \mtx{rrr}{2&-1&6\\0&0&0\\0&0&0}.\] This implies that the null space of $A-2I$ are the solutions to the equation 
\[2x_1 -x_2+6x_3=0.\] In other words, 
\[\nul(A-2I) = \Span\left\{\vr{1/2\\1\\0}, \vr{-3\\0\\1}\right\} = \Span\left\{\vr{1\\2\\0}, \vr{-3\\0\\1}\right\}.\] In fact, the eigenspace is 2-dimensional with basis $\left\{\vr{1\\2\\0}, \vr{-3\\0\\1}\right\}$.
\end{Exam}\vs

\begin{Thm} The eigenvalues of a triangular matrix are the entries on its main diagonal.
\end{Thm}
\begin{proof}
Let $A$ be a triangular matrix and let $\lambda$ be the $(i,i)$-entry of $A$. Then $A-\lambda I$ is a singular matrix because $\det(A-\lambda I) = 0$. Therefore, $(A-\lambda I)\bb x = \bb 0$ has a nontrivial solution, which implies that $\lambda$ is an eigenvalue of $A$.
\end{proof}\vs

This idea of determinants will return later in this chapter.\\

\begin{Exam} Let $A = \mtx{rrr}{3&6&-8\\0&0&6\\0&0&2}$ and $B = \mtx{rrr}{4&0&0\\-2&1&0\\5&3&4}$. Since both matrices are triangular, the eigenvalues of $A$ are $\lambda = 3, 0, 2$ and the eigenvalues of $B$ are $\lambda = 4, 1$.
\end{Exam}\vs

Notice from the previous example that $4$ appeared twice along the diagonal of $B$. This is a repeated eigenvalue of multiplicity two. This idea of multiplicity will also return later in this chapter.

%%%%%%%%%%%%%%%%%% Exercises %%%%%%%%%%%%%%%%%%%
\startExercises{eigen}

\noindent For Exercises \ref{exer:eigenvectorcheckhallstart}-\ref{exer:eigenvectorcheckhallstop}, determine if the vector is an eigenvector for $A = \mtx{rr}{2&6\\3&-1}$. If so, what is the eigenvalue? %Kaylee Hall
\begin{enumerate}[!HW!, start=1]
\begin{multicols}{3}
\item\label{exer:eigenvectorcheckhallstart} $\vr{6\\3}$
\item $\vr{1\\2}$ 
\item\label{exer:eigenvectorcheckhallstop} $\vr{1\\1}$ 
\end{multicols}
\end{enumerate}

\noindent For Exercises \ref{exer:eigenvectorcheckstart}-\ref{exer:eigenvectorcheckstop}, determine if the vector is an eigenvector for $A = \mtx{rrr}{-6&-21&-16\\6&17&12\\-5&-12&-8}$. If so, what is the eigenvalue?
\begin{enumerate}[!HW!, label=$\spadesuit$ \arabic*., ref=\arabic*]
\begin{multicols}{3}
\item\label{exer:eigenvectorcheckstart} $\vr{-2\\0\\1}$ 
\item $\vr{2\\-1\\-1}$ 
\item\label{exer:eigenvectorcheckstop} $\vr{1\\-1\\1}$
\end{multicols}
\end{enumerate}

\noindent For Exercises \ref{exer:eigenvectorcheckagainstart}-\ref{exer:eigenvectorcheckagainstop}, determine if the vector is an eigenvector for $A = \mtx{rrr}{2&-5&-2\\2&-7&-3\\-4&14&6}$. If so, what is the eigenvalue?
\begin{enumerate}[!HW!, label=$\spadesuit$ \arabic*., ref=\arabic*]
\begin{multicols}{3}
\item\label{exer:eigenvectorcheckagainstart} $\vr{-1\\-1\\2}$ 
\item $\vr{-1\\-2\\4}$ 
\item\label{exer:eigenvectorcheckagainstop} $\vr{0\\0\\0}$
\end{multicols}
\end{enumerate}

\noindent For Exercises \ref{exer:findeigenvectorstart}-\ref{exer:findeigenvectorstop}, find an eigenvector for the matrix $A$ and eigenvalue $\lambda$. Answers may vary.
\begin{enumerate}[!HW!, label=$\spadesuit$ \arabic*., ref=\arabic*]
\begin{multicols}{3}
\item\label{exer:findeigenvectorstart} $\mtx{rr}{3&1\\0&-2}$, $\lambda =3$ 
\item $\mtx{rr}{1&3\\-1&5}$, $\lambda = 2$ 
\item\label{exer:findeigenvectorstop} $\mtx{rr}{-2&-1\\1&-4}$, $\lambda=-3$ 
\end{multicols}
\end{enumerate}

\noindent For Exercises \ref{exer:eigenvaluecheckstart}-\ref{exer:eigenvaluecheckstop}, determine is the number $\lambda$ if an eigenvalue of $A = \mtx{rrrr}{-2&4&-2&1\\-15&29&-13&23\\-30&44&-19&34\\0&-4&2&-3}$.
\begin{enumerate}[!HW!, label=$\spadesuit$ \arabic*., ref=\arabic*]
\begin{multicols}{3}
\item\label{exer:eigenvaluecheckstart} $\lambda = 3$ 
\item $\lambda = -2$
\item\label{exer:eigenvaluecheckstop} $\lambda = 2$ 
\end{multicols}
\end{enumerate}

\noindent For Exercises \ref{exer:eigenbasisstart}-\ref{exer:eigenbasisstop}, find a basis for the eigenspace for the matrix and each eigenvalue $\lambda$ listed. Answers may vary. Also, determine the geometric multiplicities of each listed eigenvalue. 
\begin{enumerate}[!HW!, label=$\spadesuit$ \arabic*., ref=\arabic*]
\begin{multicols}{2}
\item\label{exer:eigenbasisstart} $\mtx{rrr}{-7&2&32\\-8&1&0\\-2&1&13}, \lambda = 5, -3$
\item\label{exer:eigenbasisstop} $\mtx{rrr}{-8&-5&5\\20&12&-10\\10&5&-3}, \lambda = -3, 2$ 
\end{multicols}
\end{enumerate}

\noindent For Exercises \ref{exer:diagonaleigenvaluestate}-\ref{exer:diagonaleigenvaluestate}, find a basis for the eigenspace for the matrix and each eigenvalue. Answers may vary. Also, determine the geometric multiplicities of each listed eigenvalue. 
\begin{enumerate}[!HW!]
\begin{multicols}{3}
    \item\label{exer:diagonaleigenvaluestate} \label{exer:diagonaleigenvaluestop} $\mtx{rrr}{1&-1&3\\0&1&2\\0&0&2}$ %Yucheng Long
\end{multicols}
\end{enumerate}

\begin{enumerate}[!HW!]
\item\label{exer:squareeigenvalue} Let $A$ be an $n\times n$ matrix. Let $\lambda$ be an eigenvalue of $A$ with associated eigenvector $\bb x$. Show that if $m$ is a positive integer then $\lambda^m$ is an eigenvalue of $A^m$. What is the associated eigenvector?\\ %NEW
\item Show that if $A$ is idempotent then $\lambda = 0, 1$.\\ %NEW
\item Show that if $A$ is nilpotent then $\lambda = 0$. %NEW
\end{enumerate}


%%%%%%%%%%%%%%%%%%% Footnotes %%%%%%%%%%%%%%%%%%%
 \mbox{}\vfill
 
\pagebreak