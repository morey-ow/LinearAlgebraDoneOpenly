\startSolutions{linearsystems}{Linear Systems}

\begin{enumerate}[!HW!, start=1]
\begin{multicols}{4}
\item no%Cameron Dix
\item no%Cameron Dix
\item yes%Cameron Dix
\item no%Cameron Dix
\end{multicols}
\begin{multicols}{4}
\itemspade no%Abby Allen
\itemspade no%Abby Allen
\itemspade yes %Abby Allen
\itemspade no%Abby Allen
\end{multicols}
\begin{multicols}{5}
\item no %anon
\item no %anon
\item no %anon
\item yes %anon
\item no %anon
\end{multicols}
\begin{multicols}{4}
\item consistent,\\ unique solution %Hannah Simonson
\item consistent,\\ multiple solutions %Christopher Newton
\item inconsistent%Hannah Simonson
\item consistent,\\ unique solution %Jacob Newey
\end{multicols}
\begin{multicols}{4}
\item inconsistent %Jacob Newey
\item consistent,\\ multiple solutions%Jacob Newey
\item consistent,\\ unique solution%Ellie McReaken
\item consistent,\\ multiple solutions%Ellie McReaken
\end{multicols}
\begin{multicols}{3}
\item inconsistent %Ellie McReaken
\item consistent,\\ unique solution%Hannah Simonson
\item homogeneous, $(0,0)$ %Yinglong Niu
\end{multicols}
\begin{multicols}{3}
\item nonhomogeneous %Yinglong Niu
\itemspade homogeneous, $(0,0)$ %Lucas Shaner
\itemspade nonhomogeneous %Lucas Shaner
\end{multicols}
\begin{multicols}{4}
\item nonhomogeneous%Lucas Shaner
\item nonhomogeneous%Lucas Shaner
\item nonhomogeneous%Yinglong Niu
\end{multicols}
\begin{multicols}{3} 
\item nonhomogeneous%Daven Triplett
\itemspade homogeneous, $(0,0,0,0)$ %Daven Triplett
\item nonhomogeneous%Daven Triplett
\end{multicols}

\item 
It is consistent since it is homogeneous. For example, $(0,0,0)$ is a solution. This is, in fact, the only solution. A sketch of the three planes shows they intersect at a single point. {\color{red} Would someone create this graphic to include here?}

\begin{multicols}{2}
\itemspade 
\mbox{}
%\answer{
\begin{center}
\begin{tabular}{c|c}
$z$ & $(x,y,z)$\\\hline
$2$ & $(7,5, 2)$\\
$3$ & $(8,8, 3)$\\
$-1$ & $(4, -4, -1)$\\
$-2$ & $(3,-7, -2)$\\
$\pi$ & $(\pi + 5, 3\pi-1, \pi)$\\
\end{tabular}
\end{center} 
%}

\itemspade 
Hint: Have you done Exercises \ref{exer:homogeneoussystemstart}--\ref{exer:homogeneoussystemhint} yet?
%\begin{proof}
%Suppose that the homogeneous linear system has $n$ variables $x_1, x_2, \ldots, x_n$ and $m$ equation. Suppose that the coefficient of $x_j$ in the $j$th equation is $a_{ij}$. Then the $i$th equation has the form 
%\[a_{i1}x_1 + a_{i2}x_2 + \ldots + a_{in}x_n = 0.\] Let $\bb 0$ be the $n$-tuple where each coordinate is 0, that is, $\bb 0 = (0,0,\ldots, 0)$. Note that
%\[a_{i1}(0) + a_{i2}(0) + \ldots + a_{in}(0) = 0+0+\ldots+0 = 0.\] Thus, $\bb 0$ is a solution to the $i$th equation of the linear system. As this equation is arbitrary, we see that $\bb 0$ is a solution to every equation in the system. Therefore, $\bb 0$ is a solution to the system, that is, the system is consistent.
%\end{proof} 
\end{multicols}
\end{enumerate}\vs