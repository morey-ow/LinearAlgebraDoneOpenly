\startSolutions{fields}{Fields}

\begin{enumerate}[!HW!, start=1]
\item 
Equality of integers is when the numbers are exactly the same. Congruence, on the other hand, is only when the remainders are equal, when divided by $p$. Additionally, congruence depends on the modulus, that is, two integers can be conguence with respect to one modulus but not another, e.g. $3\equiv 8 \pmod 5$ but $3\not\equiv 8 \pmod 7$. In contrast, equality is not relative but absolute.%Adam Maxwell

\item False. \emph{The remainder $a\pmod n$ is read ``$a$ modulo $n$'' and $\boldsymbol n$ is called the modulus of $\Z_n$.} %Carson Blickenstaff
\item True %Carson Blickenstaff
\item False. \emph{Modular division $a/b\pmod n$ is defined by multiplying by the inverse of $b$ with respect to modular \textbf{multiplication}}. %Carson Blickenstaff
\item False. \emph{The set $\Z_n$ is a field with respect to modular addition and multiplication if and only if $n$ is a \textbf{prime number}}. %Carson Blickenstaff
\item False. \emph{If $0\le a< n$, then $a\pmod n=\boldsymbol a$.} %Carson Blickenstaff

\begin{multicols}{7}
\item $3$ %Hamza Samha 
\item $4$ %Hamza Samha
\item $0$ %Hamza Samha
\item $3$ %Hamza Samha
\item $4$ %Jaden Torgerson
\item $2$ %Morganne Skelton
\item $5$ %Morganne Skelton
\end{multicols}
\begin{multicols}{7}
\itemspade $4$ 
\itemspade $2$ 
\item $3$ %Yinglong Niu
\item $1$ %Yinglong Niu
\itemspade $1$ 
\itemspade $11$ 
\itemspade $6$ 
\end{multicols}
\begin{multicols}{6}
\item $2$ %Yinglong Niu
\item $1$ %Heming Zu
\item $0$ %Mattie Zeigler
\item $3$ %Mattie Zeigler
\item $7$ %Sarah Walters
\item $x=-1$ %Jaxton Maez
\end{multicols}
\begin{multicols}{5}
\itemspade $x=-\dfrac{5}{6}$ \columnbreak
\itemspade $x=-\dfrac{1}{5} + \dfrac{8}{5}i$ \columnbreak
\item $x=1$\columnbreak %Kaylee Hall
\itemspade $x=1$ \columnbreak
\itemspade $x=3$
\end{multicols}
\begin{multicols}{7}
\itemspade $x=2$
\itemspade $x=5$
\item $x=3$  %Jaden Torgerson
\item $x=1$  %anon
\item $x=2$  %Courtney Flanigan
\item $x=8$  %Courtney Flanigan
\item $x=2$  %Courtney Flanigan
\end{multicols}
\begin{multicols}{7}
\item $x=1$  %Courtney Flanigan
\item $x=1$  %Courtney Flanigan
\item $x=5$ %Grayson Walker
\item $x=6$ %Da Huo
\item $x=3$ %anon
\end{multicols}


\itemspade 
Hint: As $\Z_6$ only contains six elements, one could check that the equation has no solution by trial and error. The lack of solution to this equation results from the fact that $2$ has no multiplicative inverse modulo 6. In fact, $2(3) \equiv 6 \equiv 0 \pmod 6$. Thus, if 2 did have a reciprocal, then $0$ would necessarily have one too. Now, we do not divide by zero. Not ever! The dinosaurs made that mistake and where are they now?
\end{enumerate}

\vspace{-15 pt}