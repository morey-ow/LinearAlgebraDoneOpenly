\startSolutions{nonsingular}{Matrix Inverses}

\begin{enumerate}[!HW!, start=1]
\begin{multicols}{4}
\item $\mtx{rr}{3&-5\\-1&2}$ %Riley Drishinkski
\item $\dfrac{1}{2}\mtx{rr}{-2&1\\3&-1}$ %Riley Drishinkski
\item $\dfrac{1}{17}\mtx{rr}{5&4\\7&9}$ %Cyrus Kaveh
\itemspade $\mtx{rr}{3&-5\\-1&2}$
\end{multicols}
\begin{multicols}{4}
\itemspade $\mtx{rr}{-7&5\\3&-2}$
\itemspade $\dfrac{1}{10}\mtx{rr}{1&-3\\2&4}$
\itemspade \mbox{$\mtx{rr}{4&1\\2&1}\pmod 5$}
\itemspade \mbox{$\mtx{rr}{4&1\\8&6} \pmod{11}$}
\end{multicols}
\begin{multicols}{4}
\itemspade $\dfrac{1}{1+i}\mtx{cc}{2& -1-i\\i&1}$
\itemspade $\bb x = %A^{-1}\bb b = 
\vr{29\\-25\\7}$
\itemspade $\bb x = %A^{-1}\bb b = 
\vr{18\\15\\-11\\8}$
\itemspade \mbox{$\bb x = %A^{-1}\bb b = 
\vr{0\\1\\1}\pmod 3$}
\end{multicols}

\begin{multicols}{3}
\itemspade nonsinuglar, the matrix has full rank
\itemspade singular, it is not a square matrix
\itemspade nonsingular, the mapping $\bb x\mapsto A\bb x$ is surjective (onto)
\end{multicols}
\begin{multicols}{3}
\itemspade singular, $A$ is not row equivalent to $I_5$\columnbreak
\itemspade not enough information, if the RREF is $I_3$ it is non-singular, otherwise it is singular.\columnbreak
\itemspade nonsingular, it has $3$ pivot columns
\end{multicols}
\begin{multicols}{3}
\itemspade singular, the mapping $\bb x\mapsto A\bb x$ is not injective (one-to-one)\columnbreak
\itemspade singular, $A$ is not row equivalent to $I_n$, otherwise this would suppose that $I_n$ is row equivalent to a singular matrix (a contradiction)\columnbreak
\itemspade singular, the columns of $A$ are linearly dependent
\end{multicols}

\begin{multicols}{3}
\item $X=(CBA)^{-1}$ %=A^{-1}B^{-1}C^{-1}$ \\ %Jaden Torgerson
\itemspade $X = -A^{-1}B$
\itemspade $X = BA - A^{-1}C$
\end{multicols}
\begin{multicols}{3}
\itemspade $X=BC^{-1}A^{-1}$
\itemspade $X=C^{-1}B-A$
\item $X=D^{-1}(ECA+B)$ %Jaden Torgerson
\end{multicols}

\item \begin{proof}[Proof of \corref{cor:inverseProps}] By \thmref{thm:inverseProps}, $A^\top $ is invertible and products of invertible matrices are invertible. Therefore, $AA^\top $ and $A^\top A$ are invertible.
\end{proof}

\item \begin{proof}[Proof of \corref{cor:nonsingularthm}]
If $AB$ is invertible, then there exists some matrix $C$ such that $(AB)C = I_n$. Thus, $A(BC) = I_n$, which shows that $A$ is invertible by (\ref{rightinverse}) in the Nonsingular Matrix Theorem. Similarly, $C(AB) = I_n$, which implies that $(CA)B = I_n$. Therefore, $B$ is likewise invertible by \ref{leftinverse} in the Nonsingular Matrix Theorem.
\end{proof}
\end{enumerate}

\vspace{-15 pt}