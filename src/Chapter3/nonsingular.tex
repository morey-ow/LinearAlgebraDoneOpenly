\begin{center} 
\emph{``My happiness grows in direct proportion to my acceptance, and in inverse proportion to my expectations.'' -- Michael J. Fox}
\end{center}

\section{Matrix Inverses}\label{sec:nonsingular}
When can we divide by a matrix? Does it even make sense to talk about matrix division? It depends on the matrix.\\

\begin{Def} An $n\times n$ matrix $A$ is  \textbf{nonsingular} (or \textbf{invertible}) if there exists an $n\times n$ matrix $B$ such that 
\[AB = BA = I_n.\] In this case, $B$ is called an \textbf{inverse} of $A$. If $A$ is not nonsingular, then it is \textbf{singular}.
\end{Def}\vs

\begin{Exam} Let $A = \mtx{rr}{2&3\\3&5}$ and $C = \mtx{rr}{5&-3\\-3&2}$. Then 
\[AC = \mtx{rr}{2&3\\3&5}\mtx{rr}{5&-3\\-3&2} = \mtx{rr}{1&0\\0&1}\] and \[CA = \mtx{rr}{5&-3\\-3&2}\mtx{rr}{2&3\\3&5} = \mtx{rr}{1&0\\0&1}.\]
Thus, $A$ is invertible with inverse $C$.
\end{Exam}\vs

We should mention that inverses are unique. Suppose that $A$ is invertible with inverses $B$ and $C$. Thus, 
\[B = BI_n = B(AC) = (BA)C = I_nC = C.\] We will denote the inverse of $A$ as $A^{-1}$. Therefore, 
\[AA^{-1} = A^{-1}A = I_n.\]\vs

\begin{Thm}\label{thm:inverseDet} Let $A = \mtx{rr}{a&b\\c&d}$. If $\det(A) := ad-bc\footnotemark[2]\neq 0$, then $A$ is nonsingular with inverse 
\[A^{-1} = \dfrac{1}{ad-bc}\mtx{rr}{d&-b\\-c&a}.\] If $ad-bc=0$, then $A$ is singular.
\end{Thm}


\begin{Exam} Find the inverse of $A = \mtx{rr}{1&2\\3&4}$, if it exists.\\

Since $\det(A) = 1(4)-(2)3 = 4-6 = -2$, $A$ is nonsingular, that is, it has an inverse, which is
\[A^{-1} = -\dfrac{1}{2}\mtx{rr}{4 &-2\\ -3 & 1} = \mtx{rr}{-2 & 1 \\ 3/2 & -1/2}. \qedhere\] 
\end{Exam}\vs

%Jaren Frandsen
\begin{Exam} Find the inverse of $A = \mtx{rr}{9&3\\6&2}$, if it exists.

Since $\det(A)=9(2)-3(6) = 18-18=0$, $A$ has no inverse, that is, $A$ is a singular matrix.
\end{Exam}

For a nonsingular matrix $A$, the equation $A\bb x = \bb b$ has a unique solution for all $\bb b \in F^n$ of the form $\bb x = A^{-1}\bb b$.\\

\begin{Exam} Solve the system of equations 
\[\left\{\begin{alignedat}{100}
&&x_1\ &+\ &2x_2\ &=\ &5&\\
&&3x_1\ &+\ &4x_2\ &=\ &6&.
\end{alignedat}\right.\]

We seek to solve the matrix equation $A\bb x = \bb b$, where $\bb b = \vr{5\\6}$ and $A$ is the matrix from the previous equation. Thus, 
\[A^{-1}(A\bb x) = A^{-1}\bb b \Rightarrow \bb x = A^{-1}\bb b = \mtx{rr}{-2 & 1 \\ 3/2 & -1/2}\vr{5\\6} = \mtx{c}{-4\\9/2}.\qedhere\]
\end{Exam}\vs

\begin{Thm}\label{thm:inverseProps} Let $A$ and $B$ be $n\times n$ invertible matrices, let $m$ be a positive integer, and $r$ is a nonzero real number. Then $A^{-1}$, $AB$, $A^\top $, $A^m$, and $rA$ are also invertible with:\vspace{-0.1 in}
\begin{enumerate}[!THM!, start=1]
\begin{multicols}{3}
\item $(A^{-1})^{-1} = A$\\

\item $(AB)^{-1} = B^{-1}A^{-1}$\footnotemark[8]\\

\item $(A^\top )^{-1} = (A^{-1})^\top $\\
\end{multicols}\vspace{-25 pt}
\begin{multicols}{2}
\item $(A^m)^{-1} = (A^{-1})^m := A^{-m}$\\

\item $(kA)^{-1} = \dfrac{1}{k}A^{-1}$.\\
\end{multicols}
\end{enumerate}
\end{Thm}

\begin{Cor}\label{cor:inverseProps} If $A$ is invertible, then $AA^\top $ and $A^\top A$ are likewise invertible.\end{Cor}\vs

\begin{Exam} Inverse matrices allow us to solve matrix equations when otherwise we may have used division. For example, solve the equation $(AX^{-1})^{-1}+B = C$, assuming all matrices are $n\times n$ and nonsingular (when necessary).\\

We begin by adding $-B$ to both sides of the equation. This gives $(AX^{-1})^{-1} = C-B$. Using the Shoe-Sock Principle, we see that the left-hand side of the equation becomes $(X^{-1})^{-1}A^{-1} = XA^{-1} = C-B$. Finally, if we multiply the equation on both sides by $A$ on the RIGHT side, we get
\begin{eqnarray*}
(XA^{-1})A &=& (C-B)A\\
XA^{-1}A) &=& CA-BA\\
XI_n &=& CA-BA\\
X&=& CA-BA
\end{eqnarray*} The side on which we are multiplying matters. Had we multiplied $A$ on the left, we would have got $AXA^{-1}$ which is not necessarily $X$. Do NOT assume matrices commute. Therefore, $X=\fbox{$CA-BA$}$.
\end{Exam}\vs

\begin{Thm}[The Nonsingular Matrix Theorem] \label{thm:nonsingular} Let $A$ be a square $n\times n$ matrix. Then the following statements are equivalent:
\begin{enumerate}[!THM!, start=1]
\begin{multicols}{2}
\item $A$ is a nonsingular matrix.\\
\item $A$ is an invertible matrix, that is, $A$ has a matrix inverse $A^{-1}$.\\
\item\label{leftinverse} There is an $n\times n$ matrix $C$ such that $CA = I_n$.\\
\item\label{rightinverse} There is an $n\times n$ matrix $C$ such that $AC = I_n$.\\
\item\label{item:nonsingularthmhomo} The equation $A\bb x =\bb 0$ has only the trivial solution.\\
\item\label{item:nonsingularthmnullity} The linear system $A\bb x = \bb b$ has no free variables for each $\bb b\in F^n$.\\
\item The equation $A\bb x = \bb b$ is consistent for each $\bb b\in F^n$.\\
\item $A^\top $ is an invertible matrix.\\
\item $A$ is row equivalent to $I_n$.\\
\item $A$ has $n$ pivot position.\\
\item The rank of $A$ is $n$.\\
\item The columns of $A$ span $F^n$.\\
\item The columns of $A$ form a linearly independent set.\\
\item The columns of $A$ form a basis for $F^n$.\\
\item The rows of $A$ span $F^n$.\\
\item The rows of $A$ form a linearly independent set.\\
\item The rows of $A$ form a basis for $F^n$.\\
\item The linear transformation $\bb x\mapsto A\bb x$ is injective (one-to-one).\\
\item The linear transformation $\bb x \mapsto A\bb x$ is surjective (onto).\\
\item The linear transformation $\bb x \mapsto A\bb x$ is bijective (one-to-one and onto).\\
\end{multicols}
\end{enumerate}
\end{Thm}
%\begin{proof}
%Through varies theorem, corollaries, and comments already proven, we see that (a)--(i) are all equivalent. We have not yet discussed how (j) and (k) are equivalent to the rest. We will first show that (a) and (j) are equivalent.\\
%
% Clearly, (a) implies (j). So, it suffices to show that (j) implies (a). Suppose that (j) holds, that is, there exists some matrix $C$ such that $CA = I_n$. Since (a) is equivalent to (g), we need only show that the only solution to $A\bb x = \bb 0$ is trivial. Suppose that $\bb{x_0}$ is any solution to $A\bb x = \bb 0$, that is, $A\bb{x_0} = \bb 0$. Multiplying on the left by $C$ then gives 
%\begin{eqnarray*}
%C( A\bb{x_0}) &=& C\bb 0\\
%(CA)\bb{x_0} &=& \bb 0\\
%\bb{x_0} &=& \bb 0,\quad\text{by (j).}
%\end{eqnarray*} Therefore, there is only the trivial solution to the homogeneous system, which implies (g).\\
%
%We next show that (a) is equivalent to (k). Like before, clearly (a) implies (k). To see that (k) implies (a), suppose that (k) holds, that is, 
%$AC = I_n$ for some matrix $C$. But by (j), this means that $C$ is invertible, because there exists some matrix $A$ such that $AC = I_n$. In particular, it must be that $A = C^{-1}$ by the uniqueness of the inverse of $C$. But $A = (A^{-1})^{-1}$, which shows that $A$ is invertible, implying (a).
%\end{proof}\vs

%We have already proven many of these equivalences. The remaining ones will be proven in the next lecture. More equivalences will be added in to this list in forthcoming lectures.\\

\begin{Cor}\label{cor:nonsingularthm} Let $A$ and $B$ be $n\times n$ matrices. If $AB$ is invertible, then $A$ and $B$ are likewise invertible.\\
\end{Cor}


\begin{Exam} Determine whether the following matrices are ``nonsingular,'' ``singular,'' or not enough information.
\begin{enumerate}
\item Suppose that $A$ is a $3\times 3$ matrix over $\C$ such that $\nullity(A)=0$.\\

By part \ref{item:nonsingularthmnullity} of the Nonsingular Matrix Theorem, we see that $A$ is nonsingular since the nullity is equal to the number of free variables in the linear system $A\bb x=\bb b$.\\

\item Suppose $A$ is a $5\times 5$ matrix over $\Z_2$. Furthermore, suppose that the equation $A\bb x = \bb 0$ has 8 solutions.\\

By part \ref{item:nonsingularthmhomo} of the Nonsingular Matrix Theorem, we see that $A$ is singular since the homogeneous equation $A\bb x=\bb0$ has nontrivial solutions.\\

\item Suppose $A$ is a $2\times 2$ matrix such that 
\[A\vr{1\\2} = \vr{1\\2}.\] 
We do not have enough information to decide if $A$ is nonsingular or not. On the one hand, the identity matrix $\mtx{cc}{1&0\\0&1}$ satisfies this condition and is nonsingular. On the other hand, $\mtx{rr}{1&0\\2&0}$ also satisfies this condition but is singular.
\end{enumerate}
\end{Exam}\vs


%%%%%%%%%%%%%%%%%% Exercises %%%%%%%%%%%%%%%%%%%
\startExercises{nonsingular}

\noindent For Exercises \ref{exer:twobytwoinverserealstart}-\ref{exer:twobytwoinverserealstop}, compute the inverse matrix of the given matrix using \thmref{thm:inverseDet}. Verify your inverse.  
\begin{enumerate}[!HW!, start]
\begin{multicols}{4}
\item\label{exer:twobytwoinverserealstart} $\mtx{rr}{2&5\\1&3}$ %Riley Drishinski
\item $\mtx{rr}{1&2\\3&4}$ %Riley Drishinski
\item $\mtx{rr}{9&-4\\-7&5}$ %Cyrus Kaveh
\itemspade $\mtx{rr}{2&5\\1&3}$
\end{multicols}
\begin{multicols}{4}
\itemspade $\mtx{rr}{2&5\\3&7}$
\itemspade $\mtx{rr}{4&3\\-2&1}$
\itemspade \mbox{$\mtx{rr}{3&2\\4&2} \pmod5$}
\itemspade \mbox{$\mtx{rr}{10&2\\5&3} \pmod{11}$}
\end{multicols}
\end{enumerate}
\begin{enumerate}[!HW!,label=$\spadesuit$ \arabic*., ref=\arabic*]
\begin{multicols}{4}
\item\label{exer:twobytwoinverserealstop} $ \mtx{cc}{1&1+i\\-i&2}$
\end{multicols}
\end{enumerate}

\noindent For Exercises \ref{exer:solveinversestart}-\ref{exer:solveinversestop}, solve linear system $A\bb x = \bb b$ using the matrix inverse. 
\begin{enumerate}[!HW!, label=$\spadesuit$ \arabic*., ref=\arabic*]
\item\label{exer:solveinversestart} $A = \mtx{rrr}{1&2&3\\-1&-1&1\\2&1&-5}$, $\bb b = \vr{0\\3\\-2}$, $A^{-1} = \mtx{rrr}{4&13&5\\-3&-11&-4\\1&3&1}$
\itemspade $A = \mtx{rrrr}{4&26&31&-15\\-3&-21&-25&12\\1&7&8&-4\\0&-1&-1&1}$, $\bb b = \vr{1\\2\\3\\4}$, $A^{-1} = \mtx{rrrr}{3&4&1&1\\-1&0&4&1\\0&-1&-3&0\\-1&-1&1&2}$
\item \label{exer:solveinversestop} $A \equiv \mtx{rrr}{1&0&1\\2&0&1\\1&2&0}$, $\bb b \equiv \vr{1\\1\\2}$, $A^{-1} \equiv \mtx{rrr}{2&1&0\\2&1&2\\2&2&0}\pmod 3$ 
\end{enumerate}

\noindent For Exercises \ref{exer:nonsingularmatrixthmstart}-\ref{exer:nonsingularmatrixthmstop}, determine whether the following matrices are ``nonsingular,'' ``singular,'' or not enough information. Explain your reasoning.
\begin{enumerate}[!HW!, label=$\spadesuit$ \arabic*., ref=\arabic*]
\begin{multicols}{3}
\item\label{exer:nonsingularmatrixthmstart} $A$ is a  $4\times 4$ matrix with $\rank(A)=4$.\columnbreak
\itemspade $A$ is a  $5\times 4$ matrix with $\rank(A)=4$.\columnbreak
\itemspade $A$ is a  $3\times 3$ matrix such that every vector in $F^3$ can be expressed as $A\bb x$, for some vector $\bb x\in F^3$. 
\end{multicols}\vspace{-15 pt}
\begin{multicols}{3}
\itemspade $A$ is a $5\times 5$ matrix with a two rows of zeros in it row-reduced echelon form.
\itemspade $A$ is a $3\times 3$ matrix in row-reduced echelon form.
\itemspade $A$ is a $3\times 3$ matrix in row-reduced echelon form and a pivot in each column.
\end{multicols}\vspace{-15 pt}
\begin{multicols}{3}
\itemspade $A$ is a $2\times 2$ matrix such that $A\bb x=A\bb y$ for two distinct vectors $\bb x, \bb y\in F^2$.
\itemspade $A$ is row equivalent to a singular matrix.
\item \label{exer:nonsingularmatrixthmstop} $A$ is a $3\times 3$ matrix with columns vectors $\bb a_1$, $\bb a_2$, $\bb a_3$ such that $\bb a_1 - 2\bb a_2 = \bb a_3$.
\end{multicols}
\end{enumerate}

\noindent For Exercises \ref{exer:matrixsolveeqnstart}-\ref{exer:matrixsolveeqnstop}, solve matrix equations for the matrix $X$ using matrix properties. You may assume all matrices are $n\times n$ and nonsingular. Be cautious! The order of multiplication matters!
\begin{enumerate}[!HW!]
\begin{multicols}{3}
\item\label{exer:matrixsolveeqnstart} $(BAX)^{-1} = C$  %Jaden Torgerson
\itemspade $AX+B=0$ 
\itemspade $A^{-1}(AX+C) = BA$
\end{multicols}
\begin{multicols}{3}
\itemspade $(XA)^{-1}B = C$
\itemspade $B(A+X)^{-1} = C$
\item\label{exer:matrixsolveeqnstop} $(DX-B)(CA)^{-1}=E$ %Jaden Torgerson
\end{multicols}
\end{enumerate}

\begin{enumerate}[!HW!]
\item Prove \corref{cor:inverseProps}.

\item Prove \thmref{cor:nonsingularthm}.

% \item Suppose that $A$ is an  $n\times n$ matrix and let $T : F^n \to F^n$ be the linear transformation $T(\bb x) = A\bb x$. Prove that $T$ is invertible if and only if $A$ is nonsingular. 
% \begin{proof}
% First, suppose that $A$ is nonsingular. Suppose that $A\bb x = A\bb y$. Then $A^{-1}A\bb x = A^{-1}A\bb y \Rightarrow \bb x = \bb y$. Thus, $T$ is injective. Next, suppose that $\bb b \in F^n$. Let $\bb x = A^{-1}\bb b$. Then $A\bb x = A(A^{-1}\bb b) = \bb b$. Thus, $T$ is surjective. Therefore, $T$ is bijective.\\

% Suppose next that $T$ is bijective. Then there exists an inverse linear transformation $S : F^n \to F^n$ such that $S\circ T = T\circ S = \Id$, the identity function. Let $B$ be the standard matrix of $S$. Then $BA = AB = I_n$. Therefore, $B = A^{-1}$.
% \end{proof}
\end{enumerate}

%%%%%%%%%%%%%%%%%%% Footnotes %%%%%%%%%%%%%%%%%%%
 \mbox{}\vfill
 \footnotetext[2]{The value $\det(A) = ad-bc$ is called the \textbf{determinant} of $A$. We will learn more about determinants in Chapter 5.}
 \footnotetext[8]{To remember this formula, think of the following proverb:
\begin{center}
\emph{Put Your Socks On, Then Your Shoes\\ Take Your Shoes Off, Then Your Socks}
\end{center}}
\pagebreak