\chapter{The Substitution and Elimination Methods}\label{chap:substitute}
Solving linear systems by graphing is very error-prone and problematic, even with the use of technology! Although the geometric approach might be necessary to comprehend solutions to linear systems, algebraic methods prove far superior to solving linear systems. In this appendix, we will discuss two such methods typically presented in a pre-linear algebra setting, beginning with substitution.\\

\textbf{The Substitution Method}
\begin{enumerate}[label=\arabic*., ref=\arabic*]
\item Solve one equation for one of the variables.\label{1}
\item Substitute the expression for this variable into the other equation.
\item Once you have determined one assignment, back substitute it into the equation from \ref{1}.
\end{enumerate}\vs

\begin{AppExam} Solve the system by substitution.
\[\begin{linear}
 &2x\ &+\ &&y\ &=\ &1& \\ 
&3x\ &+\ &4&y\ &=\ &14&,
\end{linear}\]
We will solve the first equation with respect to $y$. This gives us $y = 1-2x.$ We next will substitute this value of $y$ into the second equation, giving:
\[
3x+4y = 14\qRightarrow 3x + 4(1-2x) = 14\qRightarrow 3x + 4 - 8x = 14
\qRightarrow-5x = 10\qRightarrow x = -2.\] We next back substitute $x=-2$ into the above equation. This gives $y = 1-2(-2) = 1+4 = 5.$ Therefore, the solution is \fbox{$(-2,5)$}.
\end{AppExam}

% \begin{AppExam} Solve the system by substitution.
% \[\begin{linear}
%  &2x\ &-\ &&3y\ &=\ &-6& \\ 
% &3x\ &-\ &&y\ &=\ &5,&
% \end{linear}\]
% We will solve the second equation with respect to $y$. This gives us $ -y=-3x+5 \qRightarrow y =3x-5.$ We next will substitute this value of $y$ into the first equation, giving:
% \[2x-3y= -6\qRightarrow 2x-3(3x-5) = -6\qRightarrow 2x-9x+15=-6\qRightarrow -7x=-21\qRightarrow x=3\]
%  We next back substitute $x=3$ into from above. This gives $y = 3(3)-5 = 9-5=4.$ Therefore, the solution is \fbox{$(3,4)$}.
% \end{AppExam}\vs

\begin{AppExam} Solve \[\begin{linear} x\ &+\ &y\ &-\ &z\ &=\ &-1\\ 4x\ &-\ &3y\ &+\ &2z\ &=\ &16\\ 2x\ &-\ &2y\ &-\ &3z\ &=\ &5&. \end{linear}\]\vs

We will solve the first equation with respect to $z$ : $ z = x+y+1.$ We now substitute this expression into \emph{both} of the remaining equations.
\[\begin{linear} 4x\ &-\ &3y\ &+\ &2(x+y+1)\ &=\ &16\\ 2x\ &-\ &2y\ &-\ &3(x+y+1)\ &=\ &5\end{linear}\ \sim\ \begin{linear} 6x\ &-\ &y\ &+\ &2\ &=\ &16\\ -x\ &-\ &5y\ &-\ &3\ &=\ &5\end{linear}\ \sim\ \begin{linear} 6x\ &-\ &y\ &=\ &14\\ -x\ &-\ &5y\ &=\ &8&.\end{linear} \] We now have to solve a system of 2 equations with 2 unknowns. Again using substitution, notice that $ y = 6x-14.$ Plugging this into the last equation gives \[-x-5(6x-14) = 8\qRightarrow x - 30x + 70 = 8\qRightarrow -31x = -62\qRightarrow x = 2.\] We now substitute this value into the above equation for $y$ and get $y = 6(2) - 14 = 12-14 = -2.$ Lastly, we substitute both of these values back into above equation for $x$ and get $z = (2) + (-2) + 1 = 1.$ Therefore, the solution of the system is \fbox{$(2,-2,1)$}.
\end{AppExam}

Algebraically, one discovers that a particular system is inconsistent if during the process of solving one runs into a contradiction or a individual solution which has no solution.\\

\begin{AppExam} Find all solutions for \[\begin{linear} 8x\ &-\ &2&y\ &=\ &5&\\ -12x\ &+\ &3&y\ &=\ &7.&\end{linear}\]

We will solve the variable $x$ in the first equation, giving $x = \dfrac{1}{4}y+\dfrac{5}{8}$. Substituting this into the second equation gives 
\[-12\left(\dfrac{1}{4}y+\dfrac{5}{8}\right) + 3y = 7 \qRightarrow -3y - \dfrac{15}{2} + 3y = 7 \qRightarrow -\dfrac{15}{2} = 7,\] which is a contradiction. Therefore, the system is \fbox{inconsistent} and has no solutions. 
\end{AppExam}\vs

\textbf{The Elimination Method}
\begin{enumerate}[label=\arabic*., ref=\arabic*]
\item Choose a variable to eliminate and adjust the coefficients of said variable so they are alternating.
\item Add the equations and solve for the remaining variable.
\item Back substitute the variable value into one of the original equations.
\end{enumerate}\vs

Now, we try the elimination method to solve a linear system.\\

% \begin{Exam} Find all solutions by elimination for \[\begin{linear} 3&x\ &+\ &2y\ &=\  &&14&\\ &x\ &-\ &2y\  &=\ &&2&.\end{linear}\]\vs

% We will eliminate the variable $y$. Notice that the coefficients of $y$ are already alternating. If we add the equations, we get $4x = 16 \Rightarrow x = 4.$ Next, we will substitute $x=4$ into the first equation. This gives 
% \[3(4) + 2y = 14\qRightarrow 12 + 2y = 14\qRightarrow 2y = 2\qRightarrow y= 1.\]
%  Therefore, \fbox{$(4,1)$} is the solution of the system. 
% \end{Exam}\vs

\begin{AppExam} Find all solutions by elimination for \[\begin{linear} 2&x\ &-\ &3y\ &=\  &4&\\ 4&x\ &+\ &5y\  &=\ &3.&\end{linear}\]

We will eliminate the variable $y$. Notice that the coefficients of $y$ are already alternating but we need a common multiple. Between $3$ and $5$, the least common multiple is $3(5) = 15$. Thus, we will multiply the first equation by 5 and the second by 3. This gives
\[\begin{linear} 10&x\ &-\ &15y\ &=\  &&20&\\ 12&x\ &+\ &15y\  &=\ &&9&.\end{linear}\]\vs If we add the equations, we get $22x = 29 \Rightarrow x = \dfrac{29}{22}.$ Plugging this $x$ value into either equation will give the $y$-value. Alternatively, we could start the elimination process over again by  canceling out the $x$ this time. To do so, we will multiply the first equation by -2 and the second by nothing. This gives
\[\begin{linear} -4&x\ &+\ &6y\ &=\  &&-8&\\ 4&x\ &+\ &5y\  &=\ &&3&.\end{linear}\]\vs If we add the equations, we get $11y = -5 \Rightarrow y = -\dfrac{5}{11}.$
 Therefore, \fbox{$\left(\dfrac{29}{22}, -\dfrac{5}{11}\right)$} is the solution of the system. 
\end{AppExam}\vs

 \begin{AppExam} Solve \[\begin{linear} x\ &-\ &2y\ &-\ &z\ &=\ &8\\ 2x\ &-\ &3y\ &+\ &z\ &=\ &23\\ 4x\ &-\ &5y\ &+\ &5z\ &=\ &53. \end{linear}\]

First, we will eliminate $x$ from the system, which means eliminating $x$ from the first pair of equations and from the second pair of equations. With the first pair, multiply the first equation by $-2$ : $-2x+4y+2z = -16$. When we add the first 2 equation together, we get $y+3z=7.$ Next, we eliminate $x$ from the second pair of equations. We will multiply the second equation by $-2$, which gives $-4x+6y-2z=-46$. After we add this equation with the third, we get $3y+9z=21.$ We now have to solve the $2\times 2$ system \[\begin{linear} y\ &+\ &3z\ &=\ &7\\ 3y\ &+\ &9z\ &=\ &21.\end{linear}\] But upon further inspection, we notice that the first equation is 3 times the second equation. Therefore, the equations are dependent and the system has infinitely many solutions. \\

In order to get the general form of the solution, notice that we have $y = 7-3z$ and $x = 2y+z+8 = 2(7-3z)  +z+8 = -5z + 22$. Therefore, the general solution has the form \fbox{$(-5z+22, 7-3z, z)$}. 
\end{AppExam}\vs


%%%%%%%%%%%%%%%%%%% EXERCISES %%%%%%%%%%%%%%%%%%%%%%%%%%%
%\begin{linear} x\ &+\ &y\ &=\ &4\\ 2x\ &-\ &3y\ &=\ &1 \end{linear} %\left(\frac{13}{5}, \frac{7}{5}\right)$ %Kelton Palmer%anon
%\begin{linear} -x\ &+\ &y\ &-\ &5z\ &=\ &10\\ x\ &&&+\ &z\ &=\ &0\\-2x\ &+\ &3y\ &+\ &2z\ &=\ &6 \end{linear} %\left(\frac{3}{2}, 4, -\frac{3}{2}\right)$ %Kelton Palmer%anon