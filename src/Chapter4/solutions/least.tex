\startSolutions{least}{The Least-Squares Problem}

\begin{enumerate}[!HW!, start=1]
\begin{multicols}{3}
\itemspade $\dfrac{1}{11}\vr{20\\-8}$ %Anton 6.4.3 p. 386
\itemspade $\vr{-2\\3}$ %NEW
\itemspade $\vr{14\\30\\26}$  %Anton 6.4.6 p. 386
\end{multicols}

\itemspade $A^\top \bb b =\mtx{c}{1\\10},\quad A^\top A = \mtx{rr}{14&-3\\-3&21},\quad \widehat{\bb x} = \dfrac{1}{285}\mtx{c}{51\\143},\quad \widehat{\bb b} = \dfrac{1}{285}\vr{-92\\439\\470}$ %Anton 6.4.15 p. 386
\itemspade $A^\top \bb b =\mtx{c}{-6\\-4},\quad A^\top A = \mtx{cc}{42&0\\0&14},\quad \widehat{\bb x} = \dfrac{1}{7}\mtx{c}{-1\\-2},\quad \widehat{\bb b} = \vr{-1\\-1\\0}$ %Anton 6.4.16 p. 386

\itemspade $A= \mtx{rr}{-1&2\\2&2\\1&4},\quad A^\top \bb b =\mtx{c}{-12\\-6},\quad A^\top A = \mtx{rr}{6&6\\6&24},\quad \widehat{\bb x} = \dfrac{1}{3}\mtx{r}{-7\\1},\quad \widehat{\bb b} = \vr{3\\-4\\-1}$ %Anton 6.4.17 p. 386
\itemspade $A= \mtx{rrr}{2&1&-2\\1&0&-1\\1&1&0\\1&1&-1},\quad A^\top \bb b =\mtx{r}{30\\21\\-21},\quad A^\top A = \mtx{rrr}{7&4&-6\\4&3&-3\\-6&-3&6}, \widehat{\bb x} = \mtx{r}{6\\3\\4},\quad\widehat{\bb b} = \vr{7\\2\\9\\5}$ %Anton 6.4.18 p. 386

\itemspade $W=\Span\left\{\vr{1\\0\\-5},\ \vr{0\\1\\3}\right\},\quad A= \mtx{rr}{1&0\\0&1\\-5&3},\quad  A(A^\top A)^{-1}A^\top  = A\mtx{rr}{26&-15\\-15&10}^{-1}A^\top  = \dfrac{1}{35}A\mtx{rr}{10&15\\15&26}A^\top  = \dfrac{1}{35}A\mtx{rrr}{10&15&-5\\15&26&3} = \dfrac{1}{35}\mtx{rrr}{10&15&-5\\15&26&3\\-5&3&34}$ %Anton 6.4.25 p. 386
\itemspade $W=\Span\left\{\vr{2\\-1\\4}\right\},\quad A= \mtx{r}{2\\-1\\4} A(A^\top A)^{-1}A^\top  = A\mtx{r}{21}^{-1}A^\top  = \dfrac{1}{21}AA^\top  =\dfrac{1}{21}\mtx{rrr}{4&-2&8\\-2&1&-4\\8&-4&16}$ %Anton 6.4.26 p. 387

\item Hint: By \thmref{thm:leastsquaresolution}, $\widehat{\bb x}$ is the linear solution to the matrix equation $A^\top A\bb x = A^\top \bb b$. Since $A$ is invertible, show that the solution set of $A\bb x=\bb b$ and $A^\top A\bb x = A^\top \bb b$ are the same. %Jacob Jensen

\item Hint: You may use the fact $Q^\top Q = I_n$. Then substitute $A=QR$ into the equation from \thmref{thm:leastsquaresolution}. Also, although we know that $R$ is nonsingular and we can use $R^{-1}$, we do not know that $Q$ is nonsingular. Thus, we cannot use $Q^{-1}$ because it might not be defined.
%\begin{proof}
%\begin{eqnarray*}
%\widehat{\bb x} &=& (A^\top A)^{-1}A^\top \bb b = \Big[ (QR)^\top (QR)\Big]^{-1}(QR)^\top \bb b = \Big[ R^\top Q^\top QR\Big]^{-1}R^\top Q^\top \bb b\\
%&=& \Big[ R^\top I_nR\Big]^{-1}R^\top Q^\top \bb b =  \Big[ R^\top R\Big]^{-1}R^\top Q^\top \bb b \\
%&=& R^{-1}(R^\top )^{-1}R^\top Q^R\bb b = R^{-1}I_nQ^\top \bb b = R^{-1}Q^\top \bb b
%\end{eqnarray*}
%\end{proof}}
\end{enumerate}
