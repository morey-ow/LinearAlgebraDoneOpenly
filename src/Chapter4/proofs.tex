% \thmrepeat{thm:pythagorean}{Two vectors $\bb u, \bb v\in F^n$ are orthogonal if and only if \[\Vert \bb u + \bb v\Vert^2 = \Vert \bb u\Vert^2+ \Vert \bb v\Vert^2.\]}
% \begin{proof}
% $\Vert \bb u + \bb v\Vert^2 = \langle \bb u + \bb v,\ \bb u + \bb v\rangle = \langle \bb u,\ \bb u\rangle + 2\langle \bb u, \bb v\rangle + \langle \bb v,\ \bb v\rangle = \langle \bb u,\ \bb u\rangle + 2(0) + \langle \bb v,\ \bb v\rangle = \Vert\bb u\Vert^2 + \Vert \bb v\Vert^2.$
% \end{proof}\vs

\thmrepeat{thm:orthoindependent}{ If $S = \{\bb v_1, \ldots, \bb v_p\} \subseteq F^n$ is an orthogonal set of nonzero vectors, then $S$ is linearly independent.}
\begin{proof}
Suppose that 
\[c_1\bb v_1 + c_2\bb v_2 + \ldots + c_p\bb v_p = \bb 0.\] Then for each $i$,
\begin{eqnarray*}
\bb v_i \cdot (c_1\bb v_1 + c_2\bb v_2 + \ldots + c_p\bb v_p) &=& \bb v_i\cdot  \bb 0 \\
(\bb v_i \cdot c_i\bb v_1)  + ( \bb v_i\cdot c_2\bb v_2) + \ldots + ( \bb v_i\cdot c_i\bb v_i) + \ldots + ( \bb v_i\cdot c_p\bb v_p) &=&  0\\
c_1( \bb v_i\cdot \bb v_1) + c_2( \bb v_i\cdot \bb v_2) + \ldots + c_i(\bb v_i\cdot \bb v_i) + \ldots + c_p( \bb v_i\cdot \bb v_p) &=&  0\\
0+ 0 + \ldots + c_i\Vert \bb v_i\Vert^2 + \ldots + 0 &=&  0\\
c_i\Vert \bb v_i\Vert^2&=&  0. 
\end{eqnarray*} Therefore, $c_i = 0$ or $\Vert \bb v_i\Vert = 0$. Since $\bb v_i\neq \bb 0$, the latter is impossible. Therefore, $c_i = 0$ for all $i$, which implies that $S$ is linearly independent.
\end{proof}\vs

\thmrepeat{thm:orthocomplement}{ Let $W$ be a subspace of $F^n$. Then $W^\perp$ is also a subspace of $F^n$.}
\begin{proof}
Since $\bb 0$ is orthogonal to every vector, including those in $W$, $\bb 0 \in W^\perp$. Let $\bb x, \bb y\in W^\perp$. Then $\bb w\cdot \bb x = \bb w\cdot \bb y = 0$ for all $\bb w\in W$. Then $\bb w \cdot ( \bb x + \bb y) = \bb w\cdot \bb x + \bb w\cdot \bb y = 0+0 = 0$. Thus, $\bb x + \bb y\in W^\perp$. Finally, let $c\in F$. Then $\bb w\cdot (c\bb x) =  c(\bb w\cdot \bb x) = c(0) = 0$ for all $\bb w \in W$. Therefore, $c\bb x\in W^\perp$, which proves the $W^\perp$ is a subspace.
\end{proof}\vs
%\begin{Thm} Let $W$ be a subspace of $V$. Let $W^\perp = \{\bb x\in V \mid \langle \bb x,\ \bb w\rangle  = 0,\; \forall\bb w\in W\}$, called the \textbf{orthogonal complement} of $W$. Then $W^\perp$ is also a subspace of $V$.
%\end{Thm}
%\begin{proof}
%Since $\bb 0$ is orthogonal to every vector, including those in $W$, $\bb 0 \in W^\perp$. Let $\bb x, \bb y\in W^\perp$. Then $\langle \bb x,\ \bb w\rangle = \langle\bb y,\ \bb w\rangle = 0$ for all $\bb w\in W$. Then $\langle \bb x + \bb y,\ \bb w\rangle = \langle\bb x,\ \bb w\rangle + \langle\bb y,\ \bb w\rangle = 0+0 = 0$. Thus, $\bb x + \bb y\in W^\perp$. Finally, let $c\in \R$. Then $\langle c\bb x,\ \bb w\rangle =  c\langle\bb x,\ \bb w\rangle = c(0) = 0$ for all $\bb w \in W$. Therefore, $c\bb x\in W^\perp$, which proves the $W^\perp$ is a subspace.
%\end{proof}\vs


\thmrepeat{thm:rowortho}{ Let $A$ be an $m\times n$ matrix. Then 
\[(\row A)^\perp = \nul A.\]}
\begin{proof}
Let $A = \mtx{c}{a_{ij}}$.  Then $\bb a_i = \mtx{c}{\overline{a_{i1}} \\ \overline{a_{i2}} \\\vdots \\ \overline{a_{in}}}$ is a ``row'' vector of $A$. Let $\bb x = \mtx{c}{x_1\\x_2\\\vdots\\x_m}\in \nul A$. Then \[\bb a_i \cdot \bb x = a_{i1}x_1 + a_{i2}x_2 + \ldots + a_{in}x_n = 0.\] So, $\bb x$ is orthogonal to each row vector. Since inner products are bilinear, this shows that $\bb x$ is orthogonal to the entire row space.
\end{proof}\vs

% \thmrepeat{thm:CauchySchwarz}{ For all $\bb u, \bb v\in V$,
% \[ |\langle\bb u,\ \bb v\rangle| \le \Vert \bb u\Vert\Vert \bb v\Vert.\]}
% \begin{proof}
% If $\bb u = \bb 0$, then both sides of the inequality are 0 and the inequality holds. Suppose that $\bb u\neq \bb 0$. Then 
% \[\Vert \proj_{\bb u} \bb v\Vert = \left\Vert \dfrac{\langle \bb u, \bb v\rangle}{\langle \bb u, \bb u \rangle}\bb u\right\Vert = \left|\dfrac{\langle \bb u, \bb v\rangle}{\langle \bb u, \bb u \rangle}\right|\Vert \bb u \Vert =  \dfrac{|\langle \bb u, \bb v\rangle|}{\Vert{\bb u\Vert}^2} \Vert\bb u\Vert = \dfrac{|\langle \bb u, \bb v\rangle|}{\Vert \bb u \Vert}.\] Also,
% \[\Vert \proj_{\bb u} \bb v\Vert^2 \le \Vert \proj_{\bb u} \bb v\Vert^2 + \Vert \bb v - \proj_{\bb u} \bb v\Vert^2 = \Vert \bb v\Vert^2,\] by the Pythagorean Theorem. Therefore, \[\Vert\bb v\Vert \ge \dfrac{|\langle \bb u, \bb v\rangle|}{\Vert \bb u \Vert},\] which proves the inequality.
% \end{proof} \vs
\thmrepeat{thm:CauchySchwarz}{ For all $\bb u, \bb v\in F^n$,
\[ |\bb u \cdot \bb v| \le \Vert \bb u\Vert\Vert \bb v\Vert.\]}
\begin{proof}
If $\bb u = \bb 0$, then both sides of the inequality are 0 and the inequality holds. Suppose that $\bb u\neq \bb 0$. Then 
\[\Vert \proj_{\bb u} \bb v\Vert = \left\Vert \dfrac{\bb u \cdot \bb v}{\bb u \cdot \bb u}\bb u\right\Vert = \left|\dfrac{\bb u\cdot \bb v}{\bb u \cdot \bb u }\right|\Vert \bb u \Vert =  \dfrac{|\bb u \cdot \bb v|}{\Vert{\bb u\Vert}^2} \Vert\bb u\Vert = \dfrac{|\bb u \cdot \bb v|}{\Vert \bb u \Vert}.\] Also,
\[\Vert \proj_{\bb u} \bb v\Vert^2 \le \Vert \proj_{\bb u} \bb v\Vert^2 + \Vert \bb v - \proj_{\bb u} \bb v\Vert^2 = \Vert \bb v\Vert^2,\] by the Pythagorean Theorem. Therefore, \[\Vert\bb v\Vert \ge \dfrac{|\bb u \cdot \bb v|}{\Vert \bb u \Vert},\] which proves the inequality.
\end{proof} \vs

\thmrepeat{thm:triangleineq}{ For all $\bb u, \bb v\in F^n$, 
\[\Vert \bb u + \bb v\Vert  \le \Vert \bb u \Vert + \Vert \bb v\Vert.\] }
\begin{proof}
\begin{eqnarray*}
\Vert \bb u + \bb v\Vert^2 &=& (\bb u + \bb v)\cdot (\bb u + \bb v) = \bb u\cdot \bb u + \bb u\cdot \bb v + \bb v\cdot \bb u + \bb v\cdot \bb v = \Vert\bb u\Vert^2 + 2\text{Re}(\bb u\cdot \bb v) + \Vert \bb v\Vert^2\\
&\le& \Vert\bb u\Vert^2 + 2|\bb u\cdot \bb v| + \Vert \bb v\Vert^2\\
&\le& \Vert\bb u\Vert^2 + 2\Vert\bb u\Vert\Vert \bb v\Vert + \Vert \bb v\Vert^2, \qquad \text{by Cauchy-Schwarz Inequality},\\
&=& (\Vert \bb u\Vert + \Vert \bb v\Vert)^2.
\end{eqnarray*} Taking square roots finishes the proof.
\end{proof}
%\begin{proof}
%\begin{eqnarray*}
%\Vert \bb u + \bb v\Vert^2 &=& \langle \bb u + \bb v, \bb u + \bb v\rangle = \langle \bb u, \bb u\rangle + 2\langle \bb u, \bb v\rangle + \langle \bb v, \bb v\rangle = \Vert\bb u\Vert^2 + 2\langle \bb u, \bb v\rangle + \Vert \bb v\Vert^2\\
%&\le& \Vert\bb u\Vert^2 + 2|\langle \bb u, \bb v\rangle| + \Vert \bb v\Vert^2\\
%&\le& \Vert\bb u\Vert^2 + 2\Vert\bb u\Vert\Vert \bb v\Vert + \Vert \bb v\Vert^2, \qquad \text{by Cauchy-Schwarz Inequality},\\
%&=& (\Vert \bb u\Vert + \Vert \bb v\Vert)^2.
%\end{eqnarray*} Taking square roots finishes the proof.
%\end{proof}

\thmrepeat{thm:lawcosines}{ Let $\bb u, \bb v\in F^n$. Then 
\[\bb u \cdot \bb v = \Vert \bb u\Vert\Vert \bb v \Vert \cos\theta,\] where $\theta$ is the angle between\footnote[2]{We should mention that the situation is dramatically different between real and complex vector spaces. In the real case, we already have a well-defined notion of angles in the plane. Given that two vectors form a triangle which is always contained in a plane, the notion of angle to agree with the established notion of angle from Trigonometry. We will prove this case below. In the complex case, we do not have a notion of a complex angle lying around. Thus, there are two main schools of thought with regard to complex angles. First, as the field $\C$ can be viewed as a 2-dimensional real vector space, every vector space $\C^n$ can be viewed as a $2n$-dimensional real vector space too. Hence, the angle between the $n$-complex vectors is simply the real angle between the corresponding $2n$-real vectors. The second idea is to use the Cauchy-Schwarz Inequality. Since $0\le |\bb u\cdot \bb v| \le \Vert \bb u\Vert\Vert \bb v\Vert$, then $0\le \dfrac{\Vert \bb u\Vert\Vert \bb v\Vert}{|\bb u\cdot \bb v|} \le 1$. Hence, $\dfrac{\Vert \bb u\Vert\Vert \bb v\Vert}{|\bb u\cdot \bb v|}$ is a real number inside the range of cosine. Thus, we define $\theta$ by the relation $\theta = \cos^{-1}\left(\dfrac{\Vert \bb u\Vert\Vert \bb v\Vert}{|\bb u\cdot \bb v|}\right)$. This is the meaning we take here of complex angles, and, thus, no proof of the Law of Cosines is necessary here as it is immediate from the definition of ``the angle between.'' We could take this perspective for real angles, which we essentially already have, we choose to prove the Law of Cosines as a theorem instead of using it as a definition so that we do not have the burden of developing Trigonometry from our linear algebraic definition of $\theta$. } the two line segments from the origin to the points identified with $\bb u$ and $\bb v$.}
\begin{proof}
Consider the triangle formed by the vectors $\bb u$, $\bb v$, and $\bb u-\bb v$. Let $\theta$ be the angle between $\bb u$ and $\bb v$. Then
\[\Vert \bb u-\bb v\Vert ^2 = \Vert \bb u\Vert^2 - 2(\bb u \cdot \bb v) + \Vert \bb v\Vert^2.\] Considering the orthogonal projection $\widehat{\bb v} = \proj_{\bb u}(\bb v)$, we have that the triangle formed by $\bb v$, $\widehat{\bb v}$, and $\bb v - \widehat{\bb v}$ is right, with interior angle corresponding to the reference angle of $\theta$, call it $\widehat{\theta}$. Then $\cos \widehat{\theta}= \dfrac{\text{adj}}{\text{hyp}} = \dfrac{\Vert\widehat{\bb v}\Vert}{\Vert \bb v\Vert}$, or 
\[\Vert\bb v\Vert\cos\widehat{\theta} = \Vert\widehat{\bb v}\Vert = \left|\dfrac{\bb u\cdot \bb v}{\bb u\cdot \bb u}\right|\Vert \bb u\Vert = \dfrac{|\bb u\cdot \bb v|}{\Vert\bb u\Vert^2}\Vert \bb u\Vert  = \dfrac{|\bb u\cdot \bb v|}{\Vert\bb u\Vert}.\] We then conclude that $\Vert\bb u\Vert \Vert \bb v\Vert \cos\widehat{\theta} = |\bb u \cdot \bb v|$. Therefore, $\bb u \cdot \bb v = \Vert \bb u\Vert\Vert \bb v \Vert \cos\theta$, where the sign will depend if $\theta$ is acute, right, or obtuse.
\end{proof}\vs