\chapter{Notation Index}\label{apx:notation} %Started by Maria Langford
\ctrlT iff -- ``if and only if''

\ctrlT $\in$ -- element of a set

\ctrlT $\forall$ -- ``for all"

\ctrlT $\exists$ -- ``there exists''

\ctrlT $\subseteq$ -- subset\\

\ctrlT $\N$\label{note:natural} -- the set of natural numbers, that is, counting whole numbers, e.g. $0, 1, 2, 3, 4, \ldots$

\ctrlT $\Z$\label{note:integers} -- the set of integers, that is, whole numbers which are positive, zero, or negative, e.g.,  $\ldots, -3, -2, -1, 0, 1, 2, 3,\ldots$

\ctrlT $\Q$\label{note:rational} -- the set of rational numbers, that is, those numbers which can be expressed as a ratio of integers $a/b$ (where $b\neq 0$), e.g. $1/2, 0, -3/5, 7/2, 7/3, 5, \ldots$ Note that a rational number can be written in more than one way.

\ctrlT $\R$\label{note:real} -- the set of real numbers, that is, those numbers which are rational or irrational, e.g. $\pi$, $\sqrt{2}$, $3.5$, $4$, $-285$, $37.4568$, $\log_6(3)$, $e^7, \ldots$

\ctrlT $\C$\label{note:complex} -- the set of complex numbers, that is, those numbers of the form $a+bi$ where $i=\sqrt{-1}$ and $a,b$ are real numbers, e.g., $5, 2+4i, -i, 1/2 + 5i/7, \ldots$

\ctrlT $\Z_n = \{0, 1, 2, \ldots, n-1\}$\label{note:mod} -- the set of integers modulo $n$\\

\ctrlT $F^n$\label{note:Fn} - the vector space of all column vectors with $n$ entries chosen from the field of scalars $F$

%\ker T$