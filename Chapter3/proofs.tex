\thmrepeat{thm:rowspaceequivalence}{\label{proof:rowspaceequivalence} Two matrices $A$ and $B$ are row equivalent if and only if $\row A = \row B$. }
\begin{proof} We will show that if $A$ and $B$ are row equivalent, then $\row A = \row B$. Let $\{\bb r_1, \ldots, \bb r_m\}$ be the rows of $A$. Certainly, interchange would not change the row space since 
\[\Span\{\bb r_1, \ldots, \bb r_i, \ldots, \bb r_j,\ldots, \bb r_m\} = \Span\{\bb r_1, \ldots, \bb r_j, \ldots, \bb r_i,\ldots, \bb r_m\}.\] Additionally, scaling by a nonzero value $c$ does not affect the span of the rows since 
\[c\bb r_i \in \Span\{\bb r_1, \ldots, \bb r_i, \ldots, \bb r_m\}\qquad\text{and}\qquad \bb r_i = \frac{1}{c}(c\bb r_i) \in \Span\{\bb r_1, \ldots, c\bb r_i, \ldots,  \bb r_m\}.\] Finally, replacement also does not change the row space because 
\[\bb r_j + c\bb r_i \in \Span\{\bb r_1, \ldots, \bb r_i, \ldots, \bb r_j,\ldots, \bb r_m\}\qquad\text{and}\qquad \bb r_j = (\bb r_j +c\bb r_i) - c\bb r_i \in \Span\{\bb r_1, \ldots, \bb r_i, \ldots, \bb r_j+c\bb r_i,\ldots, \bb r_m\}.\] Since no row operation changes the row space, $\row A = \row B$ for all $A\sim B$.

In the other direction, if the rows of $A$ and $B$ are $\{\bb r_1, \ldots, \bb r_n\}$ and $\{\bb s_1, \ldots, \bb s_n\}$, respectively, then $\Span\{\bb r_1, \ldots, \bb r_n\} = \Span\{\bb s_1, \ldots, \bb s_n\}$. Applying  \secref{sec:span} \exerref{hw:changespanner2} repeatedly allows us to transform along the following lines (up to relabelling):
\[\Span\{\bb r_1, \bb r_2, \ldots, \bb r_n\} = \Span\{\bb s_1, \bb r_2, \ldots, \bb r_n\} = \Span\{\bb s_1, \bb s_2, \ldots, \bb r_n\} = \ldots =\Span\{\bb s_1, \ldots, \bb s_n\}.\] Each of these equalities is obtained by expressing $\bb s_i$ as a linear combination of the current spanning set, which inductively can be decomposed as some sequence of row scalings and row replacements. The relabelling process alluded to above can be accomplished by row interchanges. Therefore, $A$ is row equivalent to $B$.
\end{proof}\vs

\thmrepeat{thm:rowspaceequivalencerref}{\label{proof:rowspaceequivalencerref} If $U$ is an echelon form of the matrix $A$, then nonzero rows of $U$ form a basis for the row space of $A$. }
\begin{proof}
By \thmref{thm:rowspaceequivalence}, $\row(A) = \row(U)$ since $A$ and $U$ are row equivalent. Thus, a basis for $\row(U)$ is a basis for $\row(A)$. If $U$ is in echelon form, then no nonzero row can be written as a linear combination of the other rows (otherwise we could have zeroed out that row). Hence, we have a linearly independent spanning set for $\row(A)$, which is necessarily a basis.
\end{proof}\vs


\thmrepeat{thm:inverseDet}{\label{proof:inverseDet} Let $A = \mtx{rr}{a&b\\c&d}$. If $ad-bc\neq 0$, then $A$ is nonsingular with inverse 
\[A^{-1} = \dfrac{1}{ad-bc}\mtx{rr}{d&-b\\-c&a}.\] If $ad-bc=0$, then $A$ is singular. }
\begin{proof}
Suppose that $ad-bc \neq 0$. Let $B = \dfrac{1}{ad-bc}\mtx{rr}{d&-b\\-c&a}$. Note that 
\[AB =  \dfrac{1}{ad-bc}\mtx{rr}{ad-bc & -ab+ab \\ cd-cd & -bc+ad} = I_2.\] Similarly, $BA = I_2$. Thus, $B = A^{-1}$. \\

Suppose that $ad-bc=0$. If $a\neq 0$, then $\dfrac{b}{a}\vr{a\\c} = \mtx{c}{b\\ bc/a} = \vr{b\\d}$. If $a= 0$, then suppose $b\neq 0$. Similarly, $\dfrac{a}{b}\vr{b\\d} = \mtx{c}{a\\ad/b} = \vr{a\\c}$. If $a=b=0$, then either $\vr{0\\c}$ or $\vr{0\\d}$ is a multiple of the other. In all cases, one column vector is a multiple of the other. Thus, the columns are linearly dependent, which is singular by the Nonsingular Matrix Theorem.
\end{proof}\vs

\thmrepeat{thm:inverseProps}{ Let $A$ and $B$ be $n\times n$ invertible matrices, let $m$ be a positive integer, and $r$ is a nonzero real number. Then $A^{-1}$, $AB$, $A^\top $, $A^m$, and $rA$ are also invertible with:} \vspace{-0.1 in}
\begin{enumerate}[!THM!, start=1]
\begin{multicols}{3}
\item $(A^{-1})^{-1} = A$\\

\item $(AB)^{-1} = B^{-1}A^{-1}$\\

\item $(A^\top )^{-1} = (A^{-1})^\top $\\
\end{multicols}\vspace{-25 pt}
\begin{multicols}{2}
\item $(A^m)^{-1} = (A^{-1})^m := A^{-m}$\\

\item $(kA)^{-1} = \dfrac{1}{k}A^{-1}$.\\
\end{multicols}
\end{enumerate} \vspace{-25 pt}
\begin{proof}\mbox{}
\begin{enumerate}[!THM!, start=1]
\item Note that if $AA^{-1} = A^{-1}A = I_n$, then $A^{-1}$ has an inverse, $A$. Since inverses are unique, $(A^{-1})^{-1}  = A$.\\

\item Note that \[(AB)(B^{-1}A^{-1}) = A(BB^{-1})A^{-1} = A(I_n)A^{-1} = AA^{-1} = I_n.\] Similarly, $(B^{-1}A^{-1})(AB) = I_n$. Thus, $AB$ has an inverse and $(AB)^{-1} = B^{-1}A^{-1}$.\\

\item Next, note that \[(A^{-1})^\top A^\top  = (AA^{-1})^\top  = I_n^\top  = I_n.\] Similarly, $A^\top (A^{-1})^\top  = I_n$. Thus, $A^\top $ has an inverse with $(A^\top )^{-1} = (A^{-1})^\top $.\\

\item Next, note that \[(A^{-1})^mA^m = \underbrace{A^{-1}\cdots A^{-1}}_{m\text{-times}}\cdot\underbrace{A\cdots A}_{m\text{-times}} = I_n.\] Similarly, $A^m(A^{-1})^m = I_n$. Thus, $(A^{-1})^m = (A^m)^{-1}$.\\

\item Finally, note that \[\left(\dfrac{1}{r}A^{-1}\right)(kA) = \left(\dfrac{1}{k}\cdot k\right)(A^{-1}A) = 1\cdot I_n = I_n.\] Similarly, $(kA)\left(\dfrac{1}{k}A^{-1}\right) = I_n$. Thus, $\left(\dfrac{1}{k}A^{-1}\right) = (kA)^{-1}$.\hfill$\qedhere$
\end{enumerate}
\end{proof}\vs