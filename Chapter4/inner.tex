
\begin{center} 
\emph{``Success isn't measured by money or power or social rank. Success is measured by your discipline and inner peace.'' -- Mike Ditka}
\end{center}

\section{Inner Products}\label{sec:inner}
\begin{Def} We define the \textbf{dot product} $\cdot : \R^n\times \R^n \to \R$ of two vectors, $\bb u, \bb v \in \R^n$ by the rule
\[\bb u \cdot \bb v = \bb u^\top \bb v = \mtx{cccc}{u_1 & u_2 & \ldots & u_n}\mtx{c}{v_1 \\ v_2 \\ \vdots \\ v_n} = u_1v_1 + u_2v_2 + \ldots + u_nv_n.\]
\end{Def}\vs

\begin{Exam} Given $\bb u = \vr{1\\2\\3}$ and $\bb v = \vr{-5\\0\\3}$, their dot product is 
\[\bb u \cdot \bb v = 1(-5)) +2(0) + 3(3) = -5 +0 + 9 = \fbox{$4$}. \qedhere\]
\end{Exam}\vs

Let $A$ be an $m\times n$ matrix and let $\bb x\in \R^n$. Let $\bb r_1, \bb r_2,\ldots, \bb r_m\in \R^n$  be the row vectors of $A$. Thus,

$A = \mtx{c}{\bb r_1\\ \bb r_2\\ \vdots \\ \bb r_m}$. Then we can define the matrix-vector product using dot products: $A\bb x = \mtx{c}{\bb r_1\cdot \bb x\\ \bb r_2\cdot \bb x \\ \vdots\\ \bb r_m\cdot \bb x}.$\\\\

There are many algebraic reasons to extend the dot product over other fields, such as $\Z_p$ and $\C$, like in the above redefinition of matrix multiplication. For another example, the dot product is used in the creation of error-correcting codes which involved vectors of $\Z_2^n$, but the geometric benefits of the dot product are absent on these fields and this is the very goal of this chapter. Furthermore, over complex vector spaces, we vowed to never use the matrix transpose $\mbox{}^\top $. Instead, we use the conjugate transpose $\mbox{}^*$. The reason we made such a vow will be presented below in this section. This changes how we compute dot products of complex vectors.\\

\begin{Def}
We define the \textbf{Hermitian product}\footnotemark[2] $\cdot : \C^n\times \C^n \to \C$ of two vectors, $\bb u, \bb v \in \C^n$ by the rule
\[\bb u \cdot \bb v = \bb u^* \bb v = \mtx{cccc}{\overline{u_1} & \overline{u_2} & \ldots & \overline{u_n}}\mtx{c}{v_1 \\ v_2 \\ \vdots \\ v_n} = \overline{u_1}v_1 + \overline{u_2}v_2 + \ldots + \overline{u_n}v_n.\]
\end{Def}\vs



\begin{Exam} Let $\bb u = (1+i,\ i,\ 3-i)$ and $\bb v = (1+i,\ 2,\ 4i)$. Find $\bb u\cdot \bb v$ and $\bb v\cdot \bb u$.%, and $\Vert \bb u\Vert$.

\[\bb u\cdot \bb v = (\overline{1+i})(1+i) + \overline{i}(2) + (\overline{3-i})(4i) = (1-i)(1+i) - 2i + (3+i)(4i) = 2 - 2i + 12i - 4 = \fbox{$-2+10i$}\]
\[\bb v \cdot \bb u = (\overline{1+i})(1+i) + \overline{2}(i) + (\overline{4i})(3-i) = (1-i)(1+i) + 2i - (4i)(3-i) = 2 + 2i - 12i - 4 = \fbox{$-2 - 10i$}\]
%\[\Vert \bb u\Vert = \sqrt{(\overline{1+i})(1+i) + (\overline{i})i + (\overline{3-i})(3-i)} = \sqrt{ (1-i)(1+i) + (-i)i + (3+i)(3-i)} = \sqrt{2 + 1 + 10} = \fbox{$\sqrt{13}$}\qedhere\]
\end{Exam}\vs

Note that since $\R^n\subseteq \C^n$ and $\overline{x} = x$ whenever $x\in \R$, the definition of the Hermitian product generalizes the notion of the dot product. But as the dot product is a possible operation on $\C^n$ (and there are situations where one might allow the dot product of complex vectors), we will not refer to this generalization as the dot product, but instead as the Hermitian product. Both the dot product over $\R^n$ and the Hermitian product over $\C^n$ will be denoted $\bb u \cdot \bb v$. In an attempt to unify the vocabulary here, we will call the dot product on $\R^n$ and the Hermitian product on $\C^n$ the (standard) \textbf{inner product}\footnotemark[8] on $F^n$, where $F$ could be $\R$ or $\C$. \\

\begin{Thm}\label{thm:inner} Let $F$ be $\R$ or $\C$. Let $\bb u, \bb v, \bb w\in F^n$ and let $c\in F$. Then

\begin{enumerate}[!THM!,start=1]
\begin{multicols}{2}
\item\label{2nd} $\bb u \cdot(\bb v + \bb w) = \bb u \cdot \bb v  + \bb u \cdot \bb w$;
\item\label{3rd} $\bb u\cdot (c\bb v) = c(\bb u\cdot \bb v)$;
\end{multicols}
\begin{multicols}{2}
\item\label{1st} $\bb u \cdot \bb v = \bb v \cdot \bb u$;
\item\label{4th} $\bb u \cdot \bb u \ge 0$, and $\bb u \cdot \bb u = 0$ if and only if $\bb u = \bb 0$.
\end{multicols}
\end{enumerate}
\end{Thm}
%\begin{proof}
%We will prove part \emph{(\ref{2nd})}. Let $\bb u = (u_1, \ldots, u_n)$, $\bb v = (v_1, \ldots, v_n)$, and $\bb w = (w_1, \ldots, w_n)$. Then 
%\begin{eqnarray*}
%(\bb u + \bb v)\cdot \bb w &=& (u_1+v_1,\ldots, u_n+v_n)\cdot \bb w = (u_1+v_1)w_1 + \ldots + (u_n+v_n)w_n\\
% &=& u_1w_1 + v_1w_1 + \ldots + u_nw_n + v_nw_n = (u_1w_1 + \ldots + u_nw_n) + (v_1w_1 + \ldots + v_nw_n)\\
% &=& \bb u\cdot \bb w + \bb v \cdot \bb w.
%\end{eqnarray*}
%
%The remaining properties are proved similarly.
%\end{proof}\vs
Properties \ref{2nd} and \ref{3rd} above show us that inner multiplication on the left is a linear transformation $F^n\to F$ for each fixed vector $\bb u\in F^n$. When $F=\R$, \ref{1st} simplifies to just be $\bb u\cdot \bb v=\bb v\cdot \bb u$, that is, the dot product is \emph{symmetric}. Combining all these three properties shows that $(\bb u+\bb v)\cdot \bb w = \bb u\cdot \bb w + \bb v\cdot \bb w$ and $(c\bb u)\cdot \bb v = c(\bb u\cdot \bb v)$ whenever $\bb u, \bb v, \bb w\in \R^n$ and $c\in \R$. Thus, the dot product is linear in the first factor and linear in the second factor, that is, we say the dot product is \emph{bilinear}. These properties derive from the properties of transposition.\\

Conversely, the Hermitian product is \emph{conjugate-symmetric}, that is, the vector commute at the price of complex conjugation, as shown in \ref{1st}. As such, inner multiplication on the right is not exactly a linear transformation. Like with the real vector spaces, $(\bb u+\bb v)\cdot \bb w = \bb u\cdot \bb w + \bb v\cdot \bb w$ still holds for all $\bb u, \bb v, \bb w\in \C^n$. On the other hand, we get $(c\bb u)\cdot \bb v= \overline{c}(\bb u\cdot \bb v)$, that is, we can only factor scalars from the first factor if we take their conjugate. Thus, the Hermitian product is linear in the second factor and \emph{conjugate-linear} in the second factor, that is, we say the Hermitian product is \emph{sequilinear}. These properties derive from the properties of transposition.\\

Lastly, \ref{4th} is commonly referred to as \emph{positive-definite property}\footnotemark[3], \emph{positive} because $\bb x\cdot \bb x\ge 0$ and \emph{definite} because $\bb u\cdot \bb u=0$ if and only if $\bb u=0$. 



\begin{Def} The \textbf{length} (or \textbf{norm}) of a vector $\bb v\in F^n$ is the nonnegative scalar $\Vert \bb v\Vert$ defined by
\[\Vert \bb v\Vert = \sqrt{\bb v\cdot \bb v} = \sqrt{v_1^2+v_2^2 + \ldots + v_n^2}.\] We say that $\bb v$ is a \textbf{unit vector} if $\Vert \bb v\Vert = 1$.
\end{Def}\vs

It is useful to note that $\Vert \bb v\Vert^2 = \bb v\cdot \bb v$.\\

\begin{Thm} Let $\bb u%, \bb v 
\in F^n$ and $c\in F$. Then 
%\begin{enumerate}[!THM!, start=1]
%\begin{multicols}{2}
%\item $\Vert \bb u+\bb v\Vert \le \Vert \bb u\Vert + \Vert \bb v\Vert$;
%\item 
$$\Vert c\bb v\Vert = |c|\Vert \bb v\Vert$$
%\end{multicols}
%\item $\Vert\bb u\Vert \ge 0$, and $\Vert \bb u\Vert = 0$ if and only if $\bb u=\bb 0$.
%\end{enumerate}
\end{Thm}\vs
%\begin{proof}
%First of all,
%\[\Vert c\bb v\Vert^2 = (c\bb v)\cdot(c\bb v) = c(\bb v \cdot (c\bb v)) = c^2(\bb v \cdot \bb v) = (|c|(\bb v \cdot \bb v))^2 = (|c|\Vert \bb v\Vert)^2.\] To finish, take square roots.
%\end{proof}\vs

\begin{Exam} Let $\bb v = \vr{1\\0\\2\\-2} \in \R^4$. Then the length of $\bb v$ is 
\[\Vert \bb v\Vert = \sqrt{\bb v\cdot \bb v} = \sqrt{1(1) + 0(0)+ 2(2) -2(-2) } = \sqrt{1+4+4} = \sqrt{9} = \fbox{$3$}.\] Since the length of $\bb v$ is 3, $\bb v$ is not a unit vector. On the other hand, let $\bb u = \dfrac{1}{3}\bb v = \mtx{c}{1/3\\0\\2/3\\ -2/3}$. Then 
\[\Vert \bb u\Vert = \left\Vert \frac{1}{3} \bb v\right\Vert = \frac{1}{3}\Vert \bb v\Vert = \frac{1}{3}(3) = \fbox{$1$}.\] Therefore, $\bb u$ is a unit vector in the same direction as $\bb v$.
\end{Exam}\vs

The process of constructing a unit vector in the same direction as a given vector is called \textbf{normalization}. The normalization of any nonzero vector $\bb v$ is given by $\dfrac{1}{\Vert \bb v\Vert}\bb v$. The zero vector cannot be normalized. In particular, the zero vector does not point in any direction.\\

\begin{Exam} Let $\bb u = (1+i,\ i,\ 3-i)$. Find  $\Vert \bb u\Vert$.

\[\Vert \bb u\Vert = \sqrt{(\overline{1+i})(1+i) + (\overline{i})i + (\overline{3-i})(3-i)} = \sqrt{ (1-i)(1+i) + (-i)i + (3+i)(3-i)} = \sqrt{2 + 1 + 10} = \fbox{$\sqrt{13}$}\qedhere\]
\end{Exam}\vs

\begin{Def} For $\bb u, \bb v\in F^n$, the \textbf{distance} between $\bb u$ and $\bb v$, denoted as $\dist(\bb u, \bb v)$, is the length of the vector $\bb u -\bb v$, that is, 
\[\dist(\bb u, \bb v) = \Vert \bb u-\bb v\Vert.\]
\end{Def}\vs

\begin{Exam} Let $\bb u = \vr{7\\1}$ and $\bb v = \vr{3\\2}$. Then
\[\dist(\bb u, \bb v) = \Vert \bb u - \bb v\Vert = \left\Vert \vr{4\\-1}\right\Vert = \sqrt{4^2+(-1)^2} = \fbox{$\sqrt{17}$}. \qedhere\]
\end{Exam}\vs

%In physics, \textbf{work} is a force $\bb F$ applied over a distance $\bb d$. Intuitively, work is a measure of effort expended when moving an object by applying a force to it. Unlike velocity and force, work is a scalar.\\
%
%\begin{Thm} If a constant force $\bb F$ is applied to an object and moves the object in a straight line a distance $\bb d$, then the work $W$ performed by the force is 
%\begin{equation} W = \bb F\cdot \bb d\end{equation} where $\theta$ is the angle between the force $\bb F$ and $\bb d$.
%\end{Thm}\vs
%
%\begin{Exam} A force $\bb F =  35\bb i-12\bb j$ (in pounds) is used to push an object up a ramp. The resulting movement of the object is represented by the displacement vector $\bb d = 15\bb i +4\bb j$ (in feet). Find the work done by the force.\\
%\begin{center}
%\begin{tikzpicture}
%\draw[thick] (0,0) --  ++(15/20,4/20) -- ++(-4/20, 15/20) -- ++(-15/20,-4/20) -- cycle;
%\draw[thick] (0,0) -- (15/4,4/4);
%\draw[thick, dashed] (-1,0) -- (15/4,0);
%\draw[ultra thick, red, ->] (15/40, 19/40)++(0,0) -- ++(15/10,4/10) node[ above] {$\bb d$};
%\draw[ultra thick, blue, ->] (15/40, 19/40)++(0,0) -- ++(35/10,-12/10) node[midway, below] {$\bb F$};
%\end{tikzpicture}
%\end{center}
%\[\text{Work} = \bb F\cdot \bb d  = 35(15) + (-12)(4) = 525-48 = \fbox{480 ft-lb}\]
%\end{Exam}

%%%%%%%%%%%%%%%%%% Exercises %%%%%%%%%%%%%%%%%%%
\startExercises{inner}

\noindent For Exercises \ref{exer:computerdotstart}-\ref{exer:computerdotstop}, compute the given quantity. 
\begin{enumerate}[!HW!, start=1]
\begin{multicols}{3}
\item\label{exer:computerdotstart} Compute $3\bb u\cdot 2\bb v$, if \\ $\bb u =\vr{1\\ -2}$, $\bb v=\vr{2\\-3}$\columnbreak % Anthony Nguyen
\itemspade Compute $\bb u\cdot \bb v$, if\\ $\boldsymbol{u} = \vr{ 1\\ -1}$, $\boldsymbol{v} = \vr{2\\ -3}$ \columnbreak %Albert Dot Product: Basic Computation in R^2
\itemspade Compute $\bb u\cdot \bb v$, if\\ $\boldsymbol{u} = \vr{ 7\\ 1\\ -5}$, $\boldsymbol{v} = \vr{1\\ -1\\ 2}$ %Albert Dot Product: Basic Computation in R^3
\end{multicols}
\begin{multicols}{3}

\itemspade Compute $\bb u\cdot \bb v$, if\\ $\boldsymbol{u}=\vr{ 2\\ 1\\ -3\\ 2}$, $\boldsymbol{v} = \vr{ 7\\ 1\\ -1\\ 2}$\columnbreak %Albert Dot Product: Basic Computation in R^4
\item Compute $5\bb u\cdot 2\bb v$, if\\ $\bb u =\vr{7\\1\\-5}$,  $\bb v=\vr{1\\-1\\2}$\columnbreak  %Anthony Nguyen
\item Compute $2\bb u\cdot 3\bb v$, if\\ $\bb u =\vr{7\\1\\-3\\2}$, $\bb v=\vr{7\\1\\-1\\2}$ %Anthony Nguyen
\end{multicols}

\begin{multicols}{2}
\itemspade Compute $\bb u\cdot \bb v$, $\bb u\cdot \bb w$, and $\bb v\cdot \bb w$, if\\
$\bb u = \vr{i\\ 2i\\3i}$, $\bb v = \mtx{c}{4\\-2i\\ 1+i}$, $\bb w = \mtx{c}{2-i\\ 2i\\ 5+3i}$\columnbreak %Anton 5.3.11, 12 p. 324
\itemspade Compute $\bb u\cdot \bb v$, $\bb u\cdot \bb w$, and $\bb v\cdot \bb w$, if\\
\mbox{}\hspace{-20 pt}$\bb u = \mtx{c}{1+i\\ 4\\ 3i}$, $\bb v = \mtx{c}{3\\-4i\\2+3i}$, $\bb w = \mtx{c}{1-i\\ 4i\\ 4-5i}$%Anton 5.3.11, 12 p. 324
\end{multicols}

\begin{multicols}{2}
\itemspade Compute $(\boldsymbol{u} +  \boldsymbol{v}) \cdot \boldsymbol{w}$, if\\ $\boldsymbol{u} = \vr{ 1\\ -5\\ 0}$, $\boldsymbol{v} = \vr{ 3\\ 2\\ -3}$, and $\boldsymbol{w} = \vr{ 1\\ 0\\ -1}$\columnbreak %Albert Dot Product Properties: Order of Operations with Vector Sums
\itemspade Compute $2\boldsymbol{u} \cdot ( \boldsymbol{v} + \boldsymbol{w})$, if\\ $\boldsymbol{u} = \vr{ -1\\ -1\\ 3}$, $\boldsymbol{v} = \vr{ 1\\ -2\\ 1}$, $\boldsymbol{w} = \vr{ 0\\ -5\\ 3}$ %Albert Dot Product: Computation with a Vector Sum and a Scalar Multiple
\end{multicols}
\itemspade Find $(A\boldsymbol{u})\cdot \boldsymbol{v}$ where $A= \mtx{ccc}{ 1&2&3\\ 0&-1&-1\\ 1&0&1}$, $\boldsymbol{u} = \vr{ -2\\ 1\\ -1}$, $\boldsymbol{v} = \vr{ -1\\ -1\\ 3}$ %Albert Dot Product: Computation with a Matrix Product

\begin{multicols}{4}
\itemspade Compute $\Vert\bb u\Vert$, if\\ $\boldsymbol{u} = \vr{ 7\\ 1\\ 5}$\columnbreak %Albert Vector Norm: Basic Computation in R^3
\itemspade Compute $\Vert\bb u\Vert$, if\\ $\boldsymbol{u} = \vr{2\\ 1\\ -3\\ 2}$\columnbreak %Vector Norm: Basic Computation in Four Dimensions

\itemspade Compute $\Vert\bb u\Vert$, if\\ $\bb u = \mtx{c}{2-i\\ 4i\\ 1+i}$\columnbreak

\itemspade Compute $\Vert\bb u\Vert$, if\\ $\bb u = \mtx{c}{6\\ 1+4i\\ 6-2i}$
\end{multicols}

\itemspade Compute $\dist(\bb u, \bb v)$, if\\ $\boldsymbol{u} = \vr{ 1\\ 1\\ 1}$,  $\boldsymbol{v} = \vr{5\\ 2\\ 2}$ %Albert Vector Distance: Distance in 3-space


\itemspade Compute $\dist(\bb u-\bb v, 2\bb w + \bb x)$, if\\ $\boldsymbol{u} = \vr{ 1\\ -2\\ 1}$, $\boldsymbol{v} = \vr{ 0\\ -5\\ 3}$, $\boldsymbol{w} = \vr{ 3\\ -1\\ -2}$,  $\boldsymbol{x} = \vr{ 0\\ 1\\ 0}$ %Albert Vector Distance: Between Linear Combinations

\item\label{exer:computerdotstop} Compute $\dist((\boldsymbol{u} \cdot \boldsymbol{v})\boldsymbol{w}, 2\boldsymbol{x})$, if\\ $\boldsymbol{u} = \vr{ 3\\ -1\\ -2}$, $\boldsymbol{v} = \vr{ 1\\ -2\\ 3}$, $\boldsymbol{w} = \vr{ -2\\ 1\\ -1}$, $\boldsymbol{x} = \vr{ 0\\ 1\\ 0}$ %Albert Vector Distance: Between Scalar Multiples Involving Dot Products
\end{enumerate}

\begin{enumerate}[!HW!]
\begin{multicols}{2}
\item Find $x$ such that $\vr{ 1\\ 1\\ x} \cdot \vr{ x\\ 2\\ -3} = 1$. %Albert Dot Product: Computation with Variable Coordinates

\item For which real numbers $x$ make $\vr{ 1/2 \\ 1/3 \\ x }$ a unit vector? %Albert Unit Vector: One Unknown Coordinate
\end{multicols}
\end{enumerate}

%%%%%%%%%%%%%%%%%%% Footnotes %%%%%%%%%%%%%%%%%%%
 \mbox{}\vfill
 
 \footnotetext[2]{Many textbooks alternatively define the Hermitian product as $\bb u \cdot \bb v =  \bb u^\top \overline{\bb v}$. Although this does not at all change the theory and applications of the Hermitian product, it does change intermediate calculations. Be cautious if comparing with other sources since there is no universal consensus.}
 
 \footnotetext[8]{More generally, an \emph{inner product} on $F^n$ is any function $F^n\times F^n\to F$ which satisfies the axioms of \thmref{thm:inner}.}
 
 \footnotetext[3]{This property is the reason we will not be considering finite fields in this chapter. The dot product over any field is always symmetric and bilinear. The positive-definite condition fails for these fields and many others. Positive-definition is needed to establish the geometric properties we seek.}
\pagebreak