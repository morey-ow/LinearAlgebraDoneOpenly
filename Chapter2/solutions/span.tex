\startSolutions{span}{Vector Equations}

\begin{enumerate}[!HW!, start=1]
\begin{multicols}{2}
\item $\footnotesize x_1\vr{1\\1\\3}+x_2\vr{3\\6\\1}+x_3\vr{0\\-1\\2} = \vr{7\\8\\3}$
\itemspade {$\footnotesize x_1\vr{3\\-1} + x_2\vr{7\\-2} + x_3\vr{-5\\6} = \vr{-5\\4}$}
\end{multicols}
\begin{multicols}{2}
\itemspade  \mbox{$\scriptsize x_1\vr{-2\\1\\3} + x_2\vr{6\\-2\\-19} + x_3\vr{-11\\5\\31} = \vr{6\\-3\\-17}$}
\item $\footnotesize x_1\vr{1\\3\\5}+x_2\vr{2\\1\\2}+x_3\vr{2\\0\\1}=\vr{1\\4\\3}$
\end{multicols}
\begin{multicols}{3}
\itemspade yes, $\bb b = 3\bb a_1 + 2\bb a_2$\columnbreak %anon
\itemspade  no, the corresponding linear system is inconsistent %anon
\itemspade yes, $\bb b = 3\bb a_1+4\bb a_2$
\end{multicols}
\begin{multicols}{3}
\item yes, $\bb b = 3\bb a_1 + \bb a_2$
\item yes, $\bb a_1+2\bb a_2+3\bb a_3+4\bb a_4$ %Christopher Newton
\item no %Kaden Allred
\end{multicols}
\itemspade yes, $\bb b = -\bb a_1 + \bb a_2 + \bb a_3$, for example. There are, in fact, infinitely many possible linear combinations. A dependency relation occurs whenever $x_1=2-x_3$ and $x_2=3-2x_3$. %anon

\itemspade Hint: Although we know nothing about the vectors $\bb v_1, \ldots, \bb v_n$, we can still control the scalars $x_i\in F$ such that 
\[x_1\bb v_1 + x_2\bb v_2 + \ldots + x_n\bb v_n = \bb 0.\] What should we set $x_i$ equal to?
%
%%\begin{proof}
%%Note that $\bb 0 = \bb 0 + \bb 0 + \ldots + \bb 0 = 0\bb v_1 + 0\bb v_2 + \ldots + 0\bb v_n\in \Span\{\bb v_1,\ldots, \bb v_n\}$.
%%\end{proof}



\item\label{hw:subsetspan} Hint: To show that $\Span\{\bb u_1, \bb u_2\} \subseteq \Span\{\bb v_1, \bb v_2, \bb v_3\}$ take an arbitrary vector in $\Span\{\bb u_1, \bb u_2\}$, namely $a_1\bb u_1+a_2\bb u_2$ for scalars $a_1, a_2\in F$, and argue why it is in $\Span\{\bb v_1, \bb v_2, \bb v_3\}$, that is, $a_1\bb u_1+a_2\bb u_2 = b_1\bb v_1+b_2\bb v_2+b_3\bb v_3$ for some scalars $b_1,b_2,b_3\in F$. By assumption we know that $\bb u_1=c_1\bb v_1+c_2\bb v_2+c_3\bb v_3$ and $\bb u_2=d_1\bb v_1+d_2\bb v_2+d_3\bb v_3$. 

\item\label{hw:extravectorspan} Hint: Show that $\Span\{\bb v_1, \bb v_2\} \subseteq \Span\{\bb v_1, \bb v_2, \bb v_3\}$ and that $\Span\{\bb v_1, \bb v_2\} \supseteq \Span\{\bb v_1, \bb v_2, \bb v_3\}$. That is, take an arbitrary vector in one set and argue why it is in the other set. To show $\Span\{\bb v_1, \bb v_2\} \subseteq \Span\{\bb v_1, \bb v_2, \bb v_3\}$, set one of the coefficients to zero. To show $\Span\{\bb v_1, \bb v_2\} \supseteq \Span\{\bb v_1, \bb v_2, \bb v_3\}$ then suppose $\bb v_3 = a\bb v_1 + b \bb v_2$ and combine like-terms. Or, if you already did \exerref{hw:subsetspan}, you could apply that exercise twice for a very fast argument.
%%
%%\begin{proof}
%%Let $\bb x\in \Span\{\bb v_1, \bb v_2\}$. Then there exists $c_1,c_2\in \R$ such that $\bb x= c_1\bb v_1 + c_2\bb v_2$. Then 
%%\[\bb x= c_1\bb v_1 + c_2\bb v_2 + 0\bb v_3 \in \Span\{\bb v_1, \bb v_2, \bb v_3\}.\] Thus, $\Span\{\bb v_1, \bb v_2\} \subseteq \Span\{\bb v_1, \bb v_2, \bb v_3\}$.\\
%%
%%Let $\bb x\in \Span\{\bb v_1, \bb v_2, \bb v_3\}$. Then there exists $c_1,c_2, c_3\in \R$ such that $\bb x= c_1\bb v_1 + c_2\bb v_2+c_3\bb v_3$. Since $\bb v_3 \in \Span\{\bb v_1, \bb v_2\}$, there exists $d_1,d_2\in \R$ such that $\bb v_3= d_1\bb v_1 + d_2\bb v_2$. Then 
%%\begin{eqnarray*}
%%\bb x &=& c_1\bb v_1 + c_2\bb v_2+c_3\bb v_3 = c_1\bb v_1 + c_2\bb v_2+c_3(d_1\bb v_1 + d_2\bb v_2)\\
%% &=&  c_1\bb v_1 + c_2\bb v_2+(c_3d_1)\bb v_1 + (c_3d_2)\bb v_2 = (c_1+c_3d_1)\bb v_1 + (c_2+c_3d_2)\bb v_2 \in \Span\{\bb v_1, \bb v_2\}.  
%%\end{eqnarray*} Thus, $\Span\{\bb v_1, \bb v_2\} \supseteq \Span\{\bb v_1, \bb v_2, \bb v_3\}$. We conclude that $\Span\{\bb v_1, \bb v_2\} = \Span\{\bb v_1, \bb v_2, \bb v_3\}$.
%%\end{proof}\vs

\item\label{hw:linearcombospanner} Hint: Use \exerref{hw:extravectorspan} to show that $\Span\{\bb v_1, \bb v_2\} =\Span\{\bb v_1, \bb v_2, \bb v_1+\bb v_2\}$ and $\Span\{\bb v_1, \bb v_1+\bb v_2\} = \Span\{\bb v_1, \bb v_1+\bb v_2, \bb v_2\} $.

\begin{multicols}{3}
\item \mbox{Hint: Generalize \exerref{hw:subsetspan}.} 

\item \mbox{Hint: Generalize \exerref{hw:extravectorspan}.} 

\item \mbox{Hint: Generalize \exerref{hw:linearcombospanner}.} 
\end{multicols}
\item Hint: Use \exerref{hw:changespanner}. 
\end{enumerate}

\vspace{-15 pt}