
\begin{center} 
\emph{``A leader is someone who helps improve the lives of other people or improve the system they live under.''\\ -- Sam Houston}
\end{center}

\section{Reduction of Linear Systems}\label{sec:echelon}
The key to solving linear systems is to row reduce the augmented matrix to an echelon form. To accomplish this, conduct the following recursive algorithm: identify the leftmost nonzero column. This becomes a pivot column, with the pivot in the top position. If this pivot entry is zero, interchange the pivot row with any nonzero row below it. Next, zero out each entry below the pivot by replacing this row with itself plus a multiple of the pivot row. Consider next the minor matrix where this pivot row and pivot column are removed. Repeat this process on the minor matrix until an echelon form is obtained.  \\

The above algorithm for row reduction to echelon form coupled with the technique of back-substitution to solve the reduced system is call \textbf{Gaussian Elimination}. This provides an efficient procedure to solve linear systems.\\

%NEW
\begin{Exam}\label{exam:GaussElim} Reduce the matrix $A$ to echelon form and solve the associated linear system using Gaussian Elimination, where $A = \mtx{rrrrr|r}{0&0&0&-2&3&-7\\-2&-4&-6&1&-1&-3\\4&8&12&1&-2&16}$.\\

We begin by identifying the leftmost nonzero column of the matrix, which is the first column of $A$. As such, we place a pivot position in the first row of the first column. As this pivot position contains a zero value, we will interchange Rows 1 and 2, since Row 2 contains a nonzero value. (Note that the red double-arrow below will be shorthand we commonly use to denote the interchange of two rows in a matrix.)
\[\begin{tikzpicture}
\path (0,0) node {$\mtx{rrrrr|r}{ \fbox{$0$}&0&0&-2&3&-7\\-2\ &-4&-6&1&-1&-3\\4\ &8&12&1&-2&16}\quad \sim \mtx{crrrr|r}{\fbox{$-2$}&-4&-6&1&-1&-3\\0 &0&0&-2&3&-7\\4
 &8&12&1&-2&16}$};
%\path (0,5) node {\phantom{x}};
\begin{scope}[tips=proper]
\draw[ultra thick, red, stealth-stealth] (-0.5,-0.1) edge[bend right] (-0.5,0.85);
\end{scope}
\end{tikzpicture} \]
Now we proceed to place zero values in the all the entries below the pivot position in Column 1 using Row Replacement. As Row 2 already has a zero entry in Column 2 we may disregard it for now. To create a zero entry in Row 3 we will replace Row 3 with $\text{Row} 3 + 2\cdot\text{Row} 1$, as shown. (Note that red statement ``(Row $i$ + $c$Row $j$)'' written adjacent to Row $i$ will indicate we are replacing Row $i$ with Row $i$ + $c$Row $j$. For convenience and to avoid arithmetic errors, we write $c$Row $j$ write above Row $i$ in red to indicate what values we will add together to produce the next matrix in the row reduction.) 
\[\mtx{crrrr|r}{\fbox{$-2$}\phantom{\Big.^{\color{red}\ -4}}&-4\phantom{\Big.^{\color{red}\ -8}}&-6\phantom{\Big.^{\color{red}\ -12}}&1\phantom{\Big.^{\color{red}\ 2}}&-1\phantom{\Big.^{\color{red}\ -2}}&-3\phantom{\Big.^{\color{red}\ -6}}\\
0\phantom{\Big.^{\color{red}\ -4}}\ &0\phantom{\Big.^{\color{red}\ -8}}&0\phantom{\Big.^{\color{red}\ -12}}&-2\phantom{\Big.^{\color{red}\ 2}}&3\phantom{\Big.^{\color{red}\ -2}}&-7\phantom{\Big.^{\color{red}\ -6}}\\
4\Big.^{\color{red}\ -4}\ &8\Big.^{\color{red}\ -8}&12\Big.^{\color{red}\ -12}&1\Big.^{\color{red}\ 2}&-2\Big.^{\color{red}\ -2}&16\Big.^{\color{red}\ -6}}
\hspace{-7 pt}\begin{array}{c} \mbox{} \\ \mbox{} \\ \color{red} \footnotesize  (\text{Row 3 + 2 Row 1}) \end{array}
\sim \mtx{rcrrr|r}{\fbox{$-2$}&-4&-6&1&-1&-3\\0&0&0&-2&3&-7\\0&0&0&3&-4&10}\]

This completes the row reduction in Column 1. We now search for the next pivot position in the minor matrix where Row 1 and Column 1 (the location of the first pivot) are ignored. We notice there are only zero entries remaining in Column 2, so we skip this column. We do the same for Column 3. The leftmost nonzero column is then Column 4. So we place the pivot in the $(2,4)$ position. As this entry is already nonzero, no interchange is necessary here. 
\[\mtx{rrrcr|r}{-2&-4&-6&1&-1&-3\\0&0&0&\fbox{$-2$}&3&-7\\0&0&0&3&-4&10}\]

Next, to remove the $3$ below this second pivot we replace Row 3 with Row 3 + $\frac{3}{2}$Row 2. This is demonstrated below. 
\[\mtx{rrrcr|r}{-2&-4&-6&1&-1&-3\\0&0&0&\fbox{$-2$}&3&-7\\0&0&0&3&-4&10}\sim \mtx{rrrcr|r}{-2&-4&-6&1&-1&-3\\0&0&0&\fbox{$-2$}&3&-7\\0&0&0&0&1/2&-1/2} \]

This completes the row reduction in the second pivot column. The third and final pivot would then be placed in the $(3,5)$ position, as this is the leftmost nonzero column when Rows 1 and 2 and Columns 1-4 are ignored. As there are no entries below Row 3, the row reduction is completed automatically.
\[\mtx{rrrrc|r}{-2&-4&-6&1&-1&-3\\0&0&0&-2&3&-7\\0&0&0&0&\fbox{$1/2$}&-1/2} \]
Finally, row reduction ends when we reach the end of the coefficient matrix. Thus, our matrix is in echelon form. In order to solve the associated linear system, we will rewrite the augmented matrix as a system of equations and solve by back-substitution.
\[\begin{linear} -2x_1\ &-\ &4x_2\ &-\ &6x_3\ &+\ &x_4\ &-\ &x_5\ &=\ &-3\\ &&&&&-\ &2x_4\ &+\ &3x_5\ &=\ &-7\\ &&&&&&&&\frac{1}{2}x_5\ &=\ &-\frac{1}{2}
\end{linear}\] %x_5=-1, x_4=2, x_3=s, x_2=t, x_1=3-2s-3t
\end{Exam}

\begin{Rem} Many like to deviate from the Gaussian Elimination algorithm in order to avoid the somewhat more cumbersome computations that follows by having entries which are fractions and having to do arithmetic by hand.  One strategy to follow to avoid sometimes unnecessary fractions includes always choosing a row so that the pivot position is a one. Perhaps there is a row with already a one in it. Perhaps a row could be rescaled so there is a row in the pivot position, but if chosen incorrectly we might introduce fractions we are trying to avoid. Finally, a combination of row replacements can produce a pivot one if the greatest common divisor between two entries in the same column is itself one. \\

For example, in \examref{exam:GaussElim} when working on the second pivot, instead of replacing Row 3 with Row 3 + $\frac{3}{2}$Row 2, we could have replaced Row 2 with Row 2 + Row 3 and then replaced Row 3 with Row 3 -- 3Row 2, as illustrated below. It takes more row operations but avoiding all fractions.
\[\mtx{rrrcr|r}{-2&-4&-6&1&-1&-3\\0&0&0&\fbox{$-2$}&3&-7\\0&0&0&3&-4&10} \sim \mtx{rrrcr|r}{-2&-4&-6&1&-1&-3\\0&0&0&\fbox{$1$}&-1&3\\0&0&0&3&-4&10}\]\[\sim \mtx{rrrcr|r}{-2&-4&-6&1&-1&-3\\0&0&0&\fbox{$1$}&-1&3\\0&0&0&0&-1&1} \qedhere\]
\end{Rem}\vs

Because the row reduced echelon form is more desirable, we can continue to row reduce a matrix in echelon form that was produced by Gaussian elimination by transforming it into row reduced echelon form. This method is known as \textbf{Gauss-Jordan Elimination}. To perform this algorithm, first perform Gaussian elimination to produce an echelon form. This is known as the \emph{forward phase}. For the \emph{backward phase}, rescale the rightmost pivot positions to one, if not already done, and create zeros above the pivot position. Repeat this process for each pivot, moving right to left. \\

\begin{Exam} Row reduce the matrix $A$ to reduced echelon form, where $A = \mtx{rrrr|r}{0&1&-3&-1&-2\\ 1&1&2&4&3\\3&7&-6&8&1\\0&-1&3&4&-4}$.\\ %QUIZ 1

\setlength{\columnsep}{30pt}
\begin{multicols}{2}
We begin with the forward phase of row reduction. We select position $(1,1)$ for the first pivot. As there is a zero in this pivot, we interchange Row 1 and Row 2:\columnbreak

\mbox{}
\vspace{-20 pt}
\[\begin{tikzpicture}
\path (0,0) node {$\mtx{rrrr|r}{\fbox{$0$}&1&-3&-1&-2\\ 1&1&2&4&3\\3&7&-6&8&1\\0&-1&3&4&-4}\quad\sim \mtx{crrr|r}{\fbox{$1$}&1&2&4&3\\0&1&-3&-1&-2\\3&7&-6&8&1\\0&-1&3&4&-4}$};
%\path (0,5) node {\phantom{x}};
\begin{scope}[tips=proper]
\draw[ultra thick, red, stealth-stealth] (-0.35,-0.45) edge[bend right] (-0.35,1.15);
\end{scope}
\end{tikzpicture} \]
\end{multicols}
%\begin{multicols}{2}
Next, we replace Row 3 with Row 3 -- 3 Row 1:%\columnbreak
%\mbox{}
%\vspace{-20 pt}
\[ \mtx{rrrr|r}{ 
\fbox{$1$}\phantom{\Big.^{\color{red} -3}}&1\phantom{\Big.^{\color{red}\ -3}}&2\phantom{\Big.^{\color{red}\ -6}}&4\phantom{\Big.^{\color{red}\ -12}}&3\phantom{\Big.^{\color{red}\ -9}}\\
0\phantom{\Big.^{\color{red}\ -3}}&1\phantom{\Big.^{\color{red}\ -3}}&-3\phantom{\Big.^{\color{red}\ -6}}&-1\phantom{\Big.^{\color{red}\ -12}}&-2\phantom{\Big.^{\color{red}\ -9}}\\
3\Big.^{\color{red}\ -3}&7\Big.^{\color{red}\ -3}&-6\Big.^{\color{red}\ -6}&8\Big.^{\color{red}\ -12}&1\Big.^{\color{red}\ -9}
\\0\phantom{\Big.^{\color{red}\ -3}}&-1\phantom{\Big.^{\color{red}\ -3}}&3\phantom{\Big.^{\color{red}\ -6}}&4\phantom{\Big.^{\color{red}\ -12}}&-4\phantom{\Big.^{\color{red}\ -9}}}
\hspace{-7 pt}\begin{array}{c} \mbox{} \\ \mbox{} \\ \color{red} \footnotesize  (\text{Row 3 -- 3 Row 1}) \\ \mbox{}\end{array}
\sim \mtx{crrr|r}{ \fbox{$1$}&1&2&4&3\\0&1&-3&-1&-2\\0&4&-12&-4&-8\\0&-1&3&4&-4}\]
%\end{multicols}
%\begin{multicols}{2}
Next, we move the pivot to position $(2,2)$. Using this pivot, we replace Row 3 with Row 3 -- 4 Row 2 and replace Row 4 with Row 4 + Row 1:
\[\mtx{rcrr|r}{ 1&1&2&4&3\\0&\fbox{1}&-3&-1&-2\\0&4&-12&-4&-8\\0&-1&3&4&-4}
\sim \mtx{rrrr|r}{ 
1&1\phantom{\Big.^{\color{red} -4}}&2\phantom{\Big.^{\color{red}\ 12}}&4\phantom{\Big.^{\color{red}\ 4}}&3\phantom{\Big.^{\color{red}\ 6}}\\
0&\fbox{1}\phantom{\Big.^{\color{red}\ 4}}&-3\phantom{\Big.^{\color{red}\ 12}}&-1\phantom{\Big.^{\color{red}\ 4}}&-2\phantom{\Big.^{\color{red}\ 6}}\\
0&4\Big.^{\color{red} -4}&-12\Big.^{\color{red}\ 12}&-4\Big.^{\color{red}\ 4}&-8\Big.^{\color{red}\ 8}\\
0&-1\Big.^{\color{red}\ 1}&3\Big.^{\color{red}\ -3}&4\Big.^{\color{red} -1}&-4\Big.^{\color{red} -2}}
\hspace{-7 pt}\begin{array}{l} \mbox{} \\ \mbox{} \\ \color{red} \footnotesize  (\text{Row 3 -- 4 Row 1}) \\ \color{red} \footnotesize  (\text{Row 4 + Row 2})\mbox{}\end{array}
\sim \mtx{rcrr|r}{1&1&2&4&3\\0&\fbox{1}&-3&-1&-2\\0&0&0&0&0\\0&0&0&3&-6}\]
%\end{multicols}
We next move the pivot to position $(3,4)$ as there are only zeros in $(3,3)$ and $(4,3)$. As the third row is a zero row, we next interchange Rows 3 and 4. We note that this matrix is now in echelon form. To continue to RREF, we scale Row 3 by $\frac{1}{3}$.
\[\begin{tikzpicture}
\path (0,0) node {$\mtx{rrrc|r}{1&1&2&4&3\\0&1&-3&-1&-2\\0&0&0&\fbox{$0$}&0\\0&0&0&3&-6}\quad\sim \mtx{rrrc|r}{1&1&2&4&3\\0&1&-3&-1&-2\\0&0&0&\fbox{$3$}&-6\\0&0&0&0&0} \hspace{-7 pt}\begin{array}{l} \mbox{} \\ \mbox{}  \\ \color{red} \footnotesize  \frac{1}{3}(\text{Row 3})\mbox{}\\ \mbox{}\end{array} \sim \mtx{rrrc|r}{1&1&2&4&3\\0&1&-3&-1&-2\\0&0&0&\fbox{$1$}&-2\\0&0&0&0&0}$};
%\path (0,5) node {\phantom{x}};
\begin{scope}[tips=proper]
\draw[ultra thick, red, stealth-stealth] (-3.35,-1.25) edge[bend right] (-3.35,-0.4);
\end{scope}
\end{tikzpicture} \] We begin now the backward phase of row reduction. In order to zero-out the entries above the pivot, we replace Row 1 with Row 1 -- 4 Row 3 and Row 2 with Row 2 + Row 3:
\[\mtx{rrrc|r}{1&1&2&4\Big.^{\color{red}\ -4}&3\Big.^{\color{red}\ 8\phantom{-}}\\0&1&-3&-1\Big.^{\color{red}\ 1\phantom{-}}&-2\Big.^{\color{red}\ -2}\\0&0&0&\fbox{$1$}\phantom{\Big.^{\color{red}\ -4}}&-2\phantom{\Big.^{\color{red}\ -4}}\\0&0&0&0\phantom{\Big.^{\color{red}\ -4}}&0\phantom{\Big.^{\color{red}\ -4}}} \hspace{-3 pt}\begin{array}{c} \color{red} \footnotesize  (\text{Row 1 -- 4 Row 3})\mbox{} \\ \color{red} \footnotesize  (\text{Row 2 + Row 3})\mbox{}  \\ \mbox{}\\ \mbox{}\end{array} \sim \mtx{rrrc|r}{1&1&2&0&11\\0&1&-3&0&-4\\0&0&0&\fbox{$1$}&-2\\0&0&0&0&0}\]
We next move back to the pivot in position $(2,2)$. As this entry is already one, we proceed to eliminate the one above it. This is accomplished by replacing Row 1 with Row 1 -- Row 2. This final matrix is the row reduced echelon form of the original matrix. All the pivot columns are indicated.
\[\mtx{rcrr|r}{1&1\Big.^{\color{red}\ -1}&2\Big.^{\color{red}\ 3}&0&11\Big.^{\color{red}\ 4}\\0&\fbox{$1$}\phantom{\Big.^{\color{red}\ -1}}&-3\phantom{\Big.^{\color{red}\ 3}}&0&-4\phantom{\Big.^{\color{red}\ 6}}\\0&0\phantom{\Big.^{\color{red}\ -1}}&0\phantom{\Big.^{\color{red}\ 3}}&1&-2\phantom{\Big.^{\color{red}\ 6}}\\0&0\phantom{\Big.^{\color{red}\ -1}}&0\phantom{\Big.^{\color{red}\ 3}}&0&0\phantom{\Big.^{\color{red}\ 6}}} \hspace{-3 pt}\begin{array}{c} \color{red} \footnotesize  (\text{Row 1 -- Row 2})\mbox{} \\ \mbox{}  \\ \mbox{}\\ \mbox{}\end{array} \sim \mtx{ccrc|r}{\fbox{1}&0&5&0&15\\0&\fbox{1}&-3&0&-4\\0&0&0&\fbox{$1$}&-2\\0&0&0&0&0}\qedhere\]
\end{Exam}\vs

\begin{Thm} Each matrix is row equivalent to one and only one reduced echelon matrix.
\end{Thm}\vs

The proof of this important theorem essentially follows from Gauss-Jordan elimination.

\begin{Exam} \begin{multicols}{2} %Skyler Clark
Row-reducing a matrix over an alternative field, such as $\C$ or $\Z_p$ does not change the algebra whatsoever. It only changes the arithmetic associated to this field. To illustrate this fact, we demonstrate how to row reduce a complex matrix.

\[\mtx{ccc}{4+9i & 6+3i & 0 \\ 1+3i & 2+i & 0\\ -13+11i & -1+12i & 2+5i}\]
\end{multicols}

Because the first column is nonzero, the first pivot position is in the $(1,1)$-position. If we follow the Gaussian Elimination algorithm, we would replace rows below this pivot by subtracting some multiple of Row 1. But to do this, it will essentially require us to divide Row 1 by $4+9i$. While we certainly can do this, this can be a very cumbersome calculation for humans (especially those who are still a little uncomfortable with complex arithmetic). Because of this, we demonstrate a slight variation to Gaussian Elimination that, while is technically less efficient from a computational point of view, places less cognitive baggage on the reader. Instead of replacing Row 2 with $\text{Row 2} - \frac{1+3i}{4+9i}\text{Row 1}$, we replace Row 1 with $\text{Row 1} - 3\text{Row} 2$. This will be immediately followed by replacing Rows 2 and 3 with $\text{Row 2} - (1+3i)\text{Row 1}$ and $\text{Row 3} - (-13+11i)\text{Row 1}$, respectively.
\[\mtx{ccc}{\fbox{$4+9i$} & 6+3i & 0 \\ 1+3i & 2+i & 0\\ -13+11i & -1+12i & 2+5i} \sim \mtx{ccc}{\fbox{$1$} & 0 & 0 \\ 1+3i & 2+i & 0\\ -13+11i & -1+12i & 2+5i} \sim \mtx{ccc}{\fbox{$1$} & 0 & 0 \\ 0 & 2+i & 0\\ 0 & -1+12i & 2+5i}\]

Our pivot position then moves to the $(2,2)$-position. We again will deviate from the standard algorithm to scale Row 2 by $\frac{1}{2+i}$. From here, replace Row 3 with $\text{Row 3} - (-1+12i)\text{Row 2}$. Finally, with the final pivot in the $(3,3)$-position, we scale Row 3 by $\frac{1}{2+5i}$, which provides the RREF of the original matrix.
\[\mtx{ccc}{\fbox{$1$} & 0 & 0 \\ 0 & \fbox{$2+i$} & 0\\ 0 & -1+12i & 2+5i} \sim \mtx{ccc}{\fbox{$1$} & 0 & 0 \\ 0 & \fbox{$1$} & 0\\ 0 & -1+12i & 2+5i}\sim \mtx{ccc}{\fbox{$1$} & 0 & 0 \\ 0 & \fbox{$1$} & 0\\ 0 & 0 & \fbox{$2+5i$}} \sim \mtx{ccc}{\fbox{$1$} & 0 & 0 \\ 0 & \fbox{$1$} & 0\\ 0 & 0 & \fbox{$1$}} \qedhere\]
\end{Exam}

%%%%%%%%%%%%%%%%%%% Exercises %%%%%%%%%%%%%%%%%%%
\startExercises{echelon}

\noindent For Exercises \ref{exer:rrefrealstart}-\ref{exer:rrefrealstop}, compute the row reduced echelon form for the matrix. List the row operations used to transform the matrix to this form. Indicate the first time the matrix is in echelon form. Answers may vary.
\begin{enumerate}[!HW!, start=1,]
\begin{multicols}{4} %NEW
\item\label{exer:rrefrealstart} $\mtx{rr}{1&2\\-4&5}$ 
\itemspade $\mtx{rrr}{1 & 2 & 3 \\ 2 & 3 & 3 \\ 2 & 4 & 5}$ 
\itemspade $\mtx{rrrr}{1&2&2&1\\2&1&-2&-2\\1&-1&-4&-3}$ %Runtian Tu
\itemspade $\mtx{rrrr}{1&5&6&-5\\3&0&-2&1\\0&1&3&3\\-3&5&0&1}$
\end{multicols}
\begin{multicols}{4}
\item $\mtx{rrr|r}{1&2&0&9\\0&2&4&6\\0&0&5&10}$ %Sofia Jones
\item $\mtx{rrr|r}{8&16&0&14\\0&0&4&7\\0&9&3&9}$ %Sofia Jones
\item $\mtx{rrrr|r}{3&0&6&9&18\\0&2&4&0&8\\0&0&0&3&6}$ %Sofia Jones
\itemspade $\mtx{rr}{1+i & -3+7i \\ 2-i & 3+2i }$
\end{multicols}
\begin{multicols}{3}
\itemspade $\mtx{rr}{2 & 7 \\ 5 & 11 } \pmod{17}$
\itemspade $\mtx{rrr}{0 & 5 & 6 \\ 2 & 1 & 2 \\ 6 & 5 & 9} \pmod7$
\item\label{exer:rrefrealstop} \mbox{$\mtx{rrrr}{1&0&0&1\\1&2&0&-1\\3&-1&0&4\\1&4&5&1}\pmod 5$}%anon
\end{multicols}
\end{enumerate}

\noindent For Exercises \ref{exer:gausssolvestart}-\ref{exer:gausssolvestop}, solve the linear system using Gaussian elimination, that is, convert the linear system into an augmented matrix, reduce it to any echelon form, switch it back into a system of equations, and solve the reduced system using back-substitution. 
\begin{enumerate}[!HW!]
\begin{multicols}{2}
\item\label{exer:gausssolvestart} $\begin{linear}
x\ &-\ &y\ &+\ &3z\ &=\ &1\\
2x\ &-\ &2y\ &+\ &5z\ &=\ &6\\
2x\ &+\ &3y\ &-\ &4z\ &=\ &2
\end{linear}$ %Jaren Frandsen
\item $\begin{linear} %Shelby Bartlett
&&2y\ &+\ &6z\ &=\ &4\\
3y\ &+\ &2z\ &+\ &x\ &=\ &5\\
3z\ &+\ &3x\ &+\ &3y\ &=\ &3
\end{linear}$
\end{multicols}
\begin{multicols}{2}
\item $\begin{linear} 3x_1\ &+\ &2x_2\ &+\ &x_3\ &=\ &1\\
3x_1\ &+\ &8x_2\ &+\ &3x_3\ &=\ &5\\
&&3x_2\ &+\ &x_3\ &=\ &6
\end{linear}$ %Ethan Hoyer

\item
\mbox{$\begin{linear} %Bridger Hildreth
6x\ &+\ &3y\ &+\ &7z\ &\equiv\ &20\\
&&18y\ &+\ &9z\ &\equiv\ &5\\
&&3y\ &+\ &4z\ &\equiv\ &10\\
&& &&3z\ &\equiv\ &3
\end{linear}\pmod 5$}
\end{multicols}

\item\label{exer:gausssolvestop} \mbox{$\begin{linear} %anon
x_1\ &+\ &3x_2\ &+\ &5x_3\ &+\ &4x_4\ &+\ &6x_5\ &\equiv\ &0\\
2x_1\ &+\ &4x_2\ &+\ &4x_3\ &+\ &3x_4\ &+\ &x_5\ &\equiv\ &5\\
&&&&& &x_4\ && &\equiv\ &5\\
6x_1\ &+\ &2x_2\ &+\ &6x_3\ &+\ &3x_4\ && &\equiv\ &2
\end{linear}\pmod 7$}

\end{enumerate}

\noindent For Exercises \ref{exer:gaussjordansolvestart}-\ref{exer:gaussjordansolvestop}, solve the linear system using Gauss-Jordan elimination, that is, convert the linear system into an augmented matrix, reduce it to row reduced echelon form, and use this to solve the linear system.
\begin{enumerate}[!HW!]
\begin{multicols}{3}
\item\label{exer:gaussjordansolvestart}
$\begin{linear} 
2x\ &+\ &4y\ &\equiv\ &0\\
x\ &+\ &3y\ &\equiv\ &2
\end{linear} \pmod 5$ %Da Huo

\item $\begin{linear} x\ &+\ &2y\ &=\ &2\\ &-\ &4y\ &=\ &8\\ 2x\ &+\ &2y\ &=\ &6\end{linear}$ %Mariah Clayson
\itemspade $\begin{linear} x\ &+\ &y\ &=\ &8\\ x\ &-\ &y\ &=\ & 4\end{linear}$  %(6,2)
\end{multicols}
\begin{multicols}{3}
\itemspade $\begin{linear} x\ &-\ &y\ &&&=\ &6\\ 2x\ &&&-\ &3z\ &=\ & 16\\ &&2y\ &+\ &z\ &=\ &4\end{linear}$   %NEW
\item $\begin{linear}
&& 3y\ &+\ &6z\ &=\ &3\\
x\ &+\ & 2y\ &-\ &3z\ &=\ &2\\
4x\ & &  &-\ &3z\ &=\ &0
\end{linear}$ %Jaren Frandsen
\item $\begin{linear}
&&3x_2\ &-\ &2x_3\ &=\ &0\\
x_1&+\ &x_2\ &+\ &3x_3\ &=\ &12\\
6x_1&+\ &2x_2\ &+\ &x_3\ &=\ &13
\end{linear}$ %Alexis Borell
\end{multicols}
\begin{multicols}{3}
\item $\begin{linear}
x_1\ &+\ &4x_2\ &+\ &7x_3\ &=\ &3\\
2x_1&+\ &5x_2\ &+\ &8x_3\ &=\ &3\\
3x_1&+\ &6x_2\ &+\ &9x_3\ &=\ &3
\end{linear}$ %Darby Parise
\itemspade $\begin{linear} 2x\ &-\ &2y\ &-\ &2z&=\ &2\\ 2x\ &+\ &3y\ &+\ &z\ &=\ & 2\\ 3x\ &+\ &2y\ && &=\ &0\end{linear}$.   
\item $\begin{linear} 7x\ &+\ &8y\ &+\ &9z\ &=\ &10\\ 4x\ &+\ &5y\ &+\ &6z\ &=\ &10\\ x\ &+\ &2y\ &+\ &3z\ &=\ &10 \end{linear}$ %Jaden Torgerson
\end{multicols}
\begin{multicols}{3}
\item $\begin{linear} 2x_1\ &+\ & x_2\ &-\ &x_3\ &=\ &-10\\ 
x_1\ &-\ &3x_2\ &-\ &2x_3\ &=\ &-4\\ 
3x_1\ &+\ &x_2\ &+\ &x_3\ &=\ &15\end{linear}$ %Corbin Robinson
\item $\begin{linear}
2x_1\ &+\ &x_2\ &+\ &4x_3\ &=\ &199\\
2x_1\ &+\ &10x_2\ &+\ &15x_3\ &=\ &984\\
-x_1\ &&&+\ &10x_3\ &=\ &58
\end{linear}$ %Mitch Zufelt
\item $\begin{linear} 5x_1\ &+\ &2x_2\ &+\ &6x_3\ &=\ &11\\
&&x_2\ &+\ &2x_3\ &=\ &5\\
3x_1\ &+\ &2x_2\ &+\ &x_3\ &=\ &7\\
2x_1\ &+\ &x_2\ &+\ &4x_3\ &=\ &6\end{linear}$  %Kory Adams
\end{multicols}
\begin{multicols}{3}
\item $\begin{linear}
x\ &+\ &y\ &+\ &z\ &+\ &w\ &=\ &1\\
&&3y\ &+\ &z\ && &=\ &1\\
x\ &+\ &y\ &+\ &2z\ & & &=\ &0\\
2x\ & & &+\ &z\ &+\ &4w\ &=\ &0
\end{linear}$ %Ashley Taylor
\item\label{exer:gaussjordansolvestop} $\begin{linear} 
2x\ &=\ &10\ &-\ &4y\ &+\ &2z\ \\ 
-10y\ &=\ &20\ &-\ &16x\ &-\ &3z\ &-\ &3\ &+\ &13y\ &+\ &16\ &-\ & 2z\ \\
-3z\ &=\ &-4x\ &+\ &3z\ &-\ &2y\ &+\ &z\ 
\end{linear}$ %Caleb Goodrich
\end{multicols}
\end{enumerate}

%%%%%%%%%%%%%%%%%%% Footnotes %%%%%%%%%%%%%%%%%%%