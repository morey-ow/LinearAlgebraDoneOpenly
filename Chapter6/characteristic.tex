\begin{center} 
\emph{``There seems to be some perverse human characteristic that likes to make easy things difficult.''\\ -- Warren Buffett}
\end{center}

\section{The Characteristic Polynomial}\label{sec:characteristic}
%\begin{Thm} If $\bb x_1, \ldots,  \bb x_r$ are eigenvectors of an $n\times n$ matrix $A$ which correspond to distinct eigenvalues $\lambda_1, \ldots, \lambda_r$, respectively, then $\{\bb x_1, \ldots, \bb x_r\}$ is linearly independent.
%\end{Thm}
%\begin{proof}
%We proceed by induction. Clearly, if $\{\bb x_1\}$ is a set of eigenvectors, then $\{\bb x_1\}$ is linearly independent since $\bb x_1 \neq \bb 0$.\\
%
%For our induction hypothesis, we assume that if $\{\bb x_1, \ldots, \bb x_k\}$ is a set of eigenvectors of $A$ corresponding to distinct eigenvalues then $\{\bb x_1, \ldots, \bb x_k\}$ is linearly independent for all $k < r$.\\
%
%Let $\{\bb x_1, \ldots, \bb x_r\}$ be a set of eigenvectors $A$ corresponding to distinct eigenvalues and let 
%\[c_1\bb x_1 + \ldots + c_r\bb x_r = \bb 0.\] Then
%\begin{eqnarray*}
% c_1\bb x_1 + \ldots + c_r\bb x_r &=& \bb 0\\
%A(c_1\bb x_1 + \ldots + c_r\bb x_r) &=& A\bb 0\\
%c_1(A\bb x_1) + \ldots + c_r(A\bb x_r) &=& \bb 0\\
%c_1(\lambda_1\bb x_1) + \ldots + c_r(\lambda_r\bb x_r) &=& \bb 0
%\end{eqnarray*} Likewise, 
%\[c_1(\lambda_r\bb x_1) + \ldots + c_r\lambda_r\bb x_r = \bb 0.\] Subtracting these two equations gives:
% \[c_1(\lambda_1-\lambda_r)\bb x_1 + \ldots + c_{r-1}(\lambda_{r-1} - \lambda_r)\bb x_{r-1} = \bb 0.\] By our induction hypothesis, $c_i(\lambda_i-\lambda_{r}) = 0$ for all $i < r$. But  $\lambda_i-\lambda_{r}\neq 0$, since all the eigenvalues are distinct. This implies that $c_i = 0$ for all $i< r$. Therefore, 
%\[c_r\bb x_r = \bb 0.\] Since $\bb x_r\neq \bb 0$, $c_r = 0$. Therefore, $\{\bb x_1, \ldots, \bb x_r\}$ is linearly independent and the result follows by induction.
%\end{proof}\vs

As observed previously, for an $n\times n$ matrix $A$, a scalar $\lambda$ is an eigenvalue if and only if $(A-\lambda I)\bb x = \bb 0$ has a nontrivial solution if and only if $A-\lambda I$ is singular if and only if $\det(A-\lambda I) = 0$. In this context, determinants will be a valuable tool to compute eigenvalues.\\

\begin{Def} Let $A$ be an $n\times n$ matrix. Treating $\lambda$ as a variable, the value $\det(A-\lambda I)$ is a polynomial of degree $n$, called the \textbf{characteristic polynomial} of $A$. The roots of the characteristic polynomial are exactly the eigenvalues of $A$ since the roots are the solutions to the equation $\det(A-\lambda I)=0$. The (\textbf{algebraic}) \textbf{multiplicity} of each eigenvalue is its multiplicity in the characteristic polynomial.
\end{Def}\vs

\begin{Exam} Let $A = \mtx{rrr}{3&6&-8\\0&0&6\\0&0&2}$ and $B = \mtx{rrr}{4&0&0\\-2&1&0\\5&3&4}$.

Then the characteristic polynomial of $A$ is  
\begin{eqnarray*}
\det(A-\lambda I) &=&  \dtx{rrr}{3-\lambda&6&-8\\0&-\lambda&6\\0&0&2-\lambda} = (3-\lambda)(-\lambda)(2-\lambda)\\
&=& (6-5\lambda+\lambda^2)(-\lambda) = -6\lambda + 5\lambda^2 - \lambda^3.
\end{eqnarray*} Note that the eigenvalues of $A$ are 3, 0, and 2, where all multiplicities are one.\\

The characteristic polynomial of $B$ is 
\begin{eqnarray*}
\det(B-\lambda I) &=&  \dtx{rrr}{4-\lambda&0&0\\-2&1-\lambda&0\\5&3&4-\lambda} = (4-\lambda)(1-\lambda)(4-\lambda)\\
&=& (16-8\lambda+\lambda^2)(1-\lambda) = 16-24\lambda +9\lambda^2 -\lambda^3.
\end{eqnarray*} Note that the eigenvalues of $B$ are 4 and 1, where $4$ has multiplicity two and $1$ has multiplicity one.
\end{Exam}\vs

Let $A$ be a matrix with eigenvalue $\lambda$. Suppose that $m$ is the geometric multiplicity of $\lambda$ ($\nullity(A-\lambda I)$) and $n$ is the algebraic multiplicity of $\lambda$. Then $1 \le m \le n$, that is, the algebraic multiplicity is an upper bound for the geometric multiplicity.\\

\begin{Exam} Find the eigenvalues of $A = \mtx{rr}{2&3\\3&-6}$.\\

We can find the eigenvalues of $A$ by computing the characteristic polynomial of $A$ and factoring it.
\begin{eqnarray*}
\det(A-\lambda I) &=& \dtx{cc}{2-\lambda&3\\3&-6-\lambda} = (2-\lambda)(-6-\lambda) - 3(3)\\
&=& (-12+4\lambda+\lambda^2)-9 = -21 + 4\lambda + \lambda^2\\
&=& (\lambda+7)(\lambda-3)
\end{eqnarray*} Therefore, the eigenvalues of $A$ are $\lambda = -7, 3$.
\end{Exam}\vs

\begin{Exam} If $A = \mtx{rr}{0&-1\\1&0}$, then the characteristic polynomial of $A$ is 

\[\dtx{cc}{-\lambda & -1\\1&-\lambda} = \lambda^2+1 = (\lambda -i)(\lambda +i)\] and the complex eigenvalues of $A$ are $\pm i$. Since \[A-iI = \mtx{rr}{-i&-1\\1&-i}  \sim \mtx{rr}{i&1\\0&0} \sim  \mtx{rr}{1&-i\\0&0},\] we have that $\vr{i\\1}$ is an eigenvector. Note that 
\[A\vr{i\\1} = \mtx{rr}{0&-1\\1&0}\vr{i\\1} = \vr{-1\\i} = i\vr{i\\1}.\] Likewise, 
\[A+iI = \mtx{rr}{i&-1\\1&i}  \sim \mtx{rr}{i&-1\\0&0} \sim  \mtx{rr}{1&i\\0&0}.\] Thus, $\vr{-i\\1}$ is an eigenvector of $A$ and 
\[A\vr{-i\\1} = \mtx{rr}{0&-1\\1&0}\vr{-i\\1} = \vr{-1\\-i} = -i\vr{-i\\1}.\] In particular, $A$ is diagonalizable with 
\[A = \mtx{rr}{i&-i\\1&1}\mtx{rr}{i&0\\0&-i}\mtx{rr}{-i/2&1/2\\i/2&1/2}.\]
\end{Exam}\vs

Let $A$ be an $n\times n$ matrix with \emph{real} entries. Then $\overline{A\bb x} = \overline{A}\;\overline{\bb x} = A\overline{\bb x}$. If $\lambda$ is an eigenvalue of $A$ and $\bb x$ is a corresponding eigenvector, then 
\[A\overline{\bb x} = \overline{A}\;\overline{\bb x} =\overline{A\bb x} = \overline{\lambda \bb x} = \overline{\lambda}\;\overline{\bb x}.\] Then the conjugate of $\bb x$ is also an eigenvector of $A$ whose corresponding eigenvalue is the conjugate of $\lambda$. We saw this in the last example. In greater generality, the eigenvalues of a matrix of a field $F$ might lie in some \emph{extension field} $E$ such that $F\subseteq E$.\\

\begin{Thm} An $n\times n$ matrix $A$ is nonsingular if and only if $0$ is not an eigenvalue of $A$. In particular,  if $A\bb x = \lambda \bb x$, then $A^{-1}x = \dfrac{1}{\lambda}\bb x$.
\end{Thm}
%\begin{proof} Let $A$ be singular, which implies that $\det(A) = 0$. Then $\det(A-\lambda I_n) = 0$ and $\lambda = 0$ is a root of the characteristic polynomial. Therefore, $\lambda = 0$ is an eigenvalue. The converse is handled similarly.\\
%
%Let $A\bb x = \lambda \bb x$. Then $A^{-1}(A\bb x) = A^{-1}(\lambda \bb x) \qRightarrow \bb x = \lambda(A^{-1}\bb x) \qRightarrow \dfrac{1}{\lambda}\bb x= A^{-1}\bb x$.
%\end{proof}\vs

\begin{Exam} Let $A = \mtx{rrr}{3&6&-8\\0&0&6\\0&0&2}$. Then its eigenvalues are $\lambda = 3, 0, 2$. In particular, $0$ is an eigenvalue of $A$. Even though, $\bb 0$ cannot be an eigenvector, $0$ can be an eigenvalue. Notice that the eigenvalues of $\lambda =0$ are simply the vectors in $\nul(A)$, that is, the nullity of $A$ is simply the geometric multiplicity of $0$ as an eigenvalue of $A$. In fact, 

\[A\vr{-2\\1\\0} = \mtx{rrr}{3&6&-8\\0&0&6\\0&0&2}\vr{-2\\1\\0} = \vr{0\\0\\0} = 0\vr{-2\\1\\0}. \qedhere\] 
\end{Exam}

\begin{Def} Let $A$ and $B$ be $n \times n$ matrices. We say that two matrices are \textbf{similar} if there exists a nonsingular $n\times n$ matrix $P$ such that $PAP^{-1} =  B$. 
\end{Def}

\begin{Exam} The matrices $A = \mtx{rrr}{1&0&-7\\5&1&2\\-4&2&0}$ and $B =\mtx{rrr}{15&-18&-2\\17&-17&-4\\7&-22&4}$ are similar since $P = \mtx{rrr}{1&-2&-1\\1&-1&0\\1&-4&-2}$ is nonsingular with $P^{-1} =  \mtx{rrr}{2&0&-1\\2&-1&-1\\-3&2&1}$ and 
\[PAP^{-1} = \mtx{rrr}{1&-2&-1\\1&-1&0\\1&-4&-2}\mtx{rrr}{1&0&-7\\5&1&2\\-4&2&0}\mtx{rrr}{2&0&-1\\2&-1&-1\\-3&2&1} = \mtx{rrr}{1&-2&-1\\1&-1&0\\1&-4&-2}\mtx{rrr}{23&-14&-8\\6&3&-4\\-4&-2&2}\] \[ = \mtx{rrr}{15&-18&-2\\17&-17&-4\\7&-22&4} = B. \qedhere\]
\end{Exam}

\begin{Thm} If $A$ and $B$ are two similar matrices, then they have the same characteristic polynomial and hence the same eigenvalues (with the same multiplicities).
\end{Thm}
%\begin{proof}
%If $B = PAP^{-1}$, then 
%\begin{eqnarray*}
%B-\lambda I &=& PAP^{-1}  - \lambda I =  PAP^{-1}  - (\lambda I)(PP^{-1})\\
%& = & PAP^{-1}  - P(\lambda I)P^{-1} = P(A-\lambda I)P^{-1}.
%\end{eqnarray*} Therefore, 
%\begin{eqnarray*}
%\det(B-\lambda I) &=& \det( P(A-\lambda I)P^{-1}) = \det(P)\det(A-\lambda I)\det(P^{-1})\\  &=& \det(P)\det(P^{-1})\det(A-\lambda I) = \det(PP^{-1})\det(A-\lambda I)\\ &=& \det(I)\det(A-\lambda I). \qedhere
%\end{eqnarray*}
%\end{proof}\vs

%A similar proof can be used to show that similar matrices have the same determinant. 
Likewise, similar matrices have the same determinant, rank, nullity, and trace. Thus, if any two matrices differ on one of these \emph{invariants} then they cannot be similar.\\

\begin{Exam} The matrices $A=\mtx{rr}{1 & 2\\ 0  & -2}$ and $B = \mtx{rr}{3&1\\-1&2}$ are not similar since $\tr(A) = 1-2 = -1\neq 5 = 3+2 = \tr(B)$.
\end{Exam}\vs

%%%%%%%%%%%%%%%%%% Exercises %%%%%%%%%%%%%%%%%%%
\startExercises{characteristic}

\noindent For Exercises \ref{exer:findeigenvaluesstart}-\ref{exer:findeigenvaluesstop}, find the eigenvalues and bases for the eigenspaces of the matrix $A$ given. Answers may vary.
\begin{enumerate}[!HW!, start=1, label=$\spadesuit$ \arabic*., ref=\arabic*]
\begin{multicols}{3}
\item\label{exer:findeigenvaluesstart} $\mtx{rr}{10&6\\-12&-8}$ %NEW
\item $\mtx{rr}{6&1\\-1&4}$ %NEW
\item $\mtx{rr}{35&48\\-24&-33}$ %NEW  
\end{multicols}
\begin{multicols}{3}
\item $\mtx{rr}{4&-5\\1&0}$ %Anton 5.3.15 p. 324
\item $\mtx{rr}{-1&-5\\4&7}$ %Anton 5.3.16
\item $\mtx{rr}{5&-2\\1&3}$ %Anton 5.3.17
\end{multicols}
\begin{multicols}{3}
\item $\mtx{rrr}{-6  & -2 & -2 \\ 17 & 6 & 5 \\ 7 & 2 & 3}$
\item $\mtx{rrr}{17&5&5\\-41&-12&-14\\-19&-6&-4}$
\item\label{exer:findeigenvaluesstop} $\mtx{rrr}{3&1&1\\-2&-1&-1\\-7&-2&-2}$
\end{multicols}
\end{enumerate}

\noindent For Exercises \ref{exer:notsimilarstart}-\ref{exer:notsimilarstop}, explain why the matrices $A$ and $B$ below are NOT similar. (Sure, do it QUICKLY if you want a challenge, UNDER 60 SECONDS!)
\begin{enumerate}[!HW!, label=$\spadesuit$ \arabic*., ref=\arabic*]
\begin{multicols}{2}
\item\label{exer:notsimilarstart} $\mtx{rr}{1&0\\0&4}$, $\mtx{rr}{1&-3\\5&2}$ 
\item\label{exer:notsimilarstop} $\mtx{rr}{1&0\\0&4}$, $\mtx{rr}{2&-3\\0&3}$
\end{multicols}
\end{enumerate}


%%%%%%%%%%%%%%%%%%% Footnotes %%%%%%%%%%%%%%%%%%%
 \mbox{}\vfill
 
\pagebreak