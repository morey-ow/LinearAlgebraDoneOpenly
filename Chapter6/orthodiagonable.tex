\begin{center} 
\emph{``It's always good to take an orthogonal view of something. It develops ideas.'' -- Ken Thompson}
\end{center}

\section{Orthogonal Diagonalization}\label{sec:orthodiagonable}
Symmetric matrices interestingly bring together the theory of eigenvectors and inner products. Recall a real square matrix $U$ is called \emph{orthogonal} if $U^\top U = I$, that is, $U^{T} = U^{-1}$ and complex matrix $U$ is called \emph{unitary} if $U^*U=I$, that is, $U^*=U^{-1}$.\\

\begin{Thm} If $A$ is a symmetric matrix or a Hermitian matrix, then any two eigenvectors with distinct eigenvalues are orthogonal.
\end{Thm}
%\begin{proof}
%Let $(\lambda, \bb x)$ and $(\mu, \bb y)$ be two eigenpairs of $A$ with $\lambda \neq \mu$. Then 
%\[(A\bb x)\cdot \bb y = (\lambda\bb x)\cdot \bb y = \lambda(\bb x \cdot \bb y).\] On the other hand, 
%\[(A\bb x)\cdot \bb y = (A\bb x)^\top \bb y =  (\bb x^\top A^\top )\bb y = \bb x^\top (A\bb y) = \bb x \cdot (A\bb y) = \mu(\bb x\cdot \bb y).\] Therefore, 
%$\lambda(\bb x\cdot \bb y) = \mu(\bb x \cdot \bb y)$, which implies that $(\lambda - \mu)(\bb x\cdot \bb y) = 0$. Since $\lambda \neq \mu$, it cannot be that $\lambda - \mu = 0$. Therefore, $\bb x \cdot \bb y = 0$.
%\end{proof}
\vs

\begin{Exam} We note that $A = \mtx{rrr}{6&-2&-1\\-2&6&-1\\-1&-1&5}$ is symmetric. It can also be checked that 
\[\lambda_1 = 8 : \bb v_1 = \vr{-1\\1\\0};\qquad \lambda_2 =6 : \bb v_2 = \vr{-1\\-1\\2};\qquad \lambda_3 = 3 : \bb v_3 = \vr{1\\1\\1} \] are all eigenvectors of $A$. Furthermore, $\bb v_1\cdot \bb v_2 = \bb v_1 \cdot \bb v_3 = \bb v_2 \cdot \bb v_3 = 0$. Normalizing these vectors:
\[\bb u_1  = \mtx{c}{-1/\sqrt{2}\\1/\sqrt{2}\\0},\qquad \bb u_2 = \mtx{c}{-1/\sqrt{6} \\ -1/\sqrt{6} \\ 2/\sqrt{6}},\qquad \bb u_3 = \mtx{c}{1/\sqrt{3}\\1/\sqrt{3}\\1/\sqrt{3}},\]  we get a diagonalization of $A$:
\[A = PDP^{-1} = \mtx{ccc}{ -1/\sqrt{2} & -1/\sqrt{6} & 1/\sqrt{3} \\ 1/\sqrt{2} & -1/\sqrt{6} & 1/\sqrt{3} \\ 0 & 2/\sqrt{6} & 1/\sqrt{3}}\mtx{ccc}{8&0&0\\0&6&0\\0&0&3}\mtx{ccc}{-1/\sqrt{2} & 1/\sqrt{2} & 0 \\ -1/\sqrt{6} & -1/\sqrt{6} & 2/\sqrt{6} \\ 1/\sqrt{3} & 1/\sqrt{3} & 1/\sqrt{3}}\] using an orthogonal matrix $P$.
\end{Exam}\vs

\begin{Def} We say a real matrix $A$ is \textbf{orthogonally diagonalizable} if there exists an orthogonal matrix $P$ and diagonal matrix $D$ such that $A = PDP^\top  = PDP^{-1}$. We say a complex matrix $A$ is \textbf{unitarily diagonalizable} if there exists an Hermitian matrix $P$ and diagonal matrix $D$ such that $A = PDP^* = PDP^{-1}$.
\end{Def}\vs

The matrix in the previous  example is orthogonally diagonalizable. In fact, if $A$ is orthogonally diagonalizable, then $A = PDP^\top $ and $A^\top  = (PDP^\top )^\top  = (P^\top )^\top D^\top P^\top  = PD^\top P^\top  = PDP^\top $, that is, $A^\top  = A$. Thus, $A$ is symmetric. The converse is also true.\\

\begin{Thm} A real matrix $A$ is orthogonally diagonalizable if and only if $A$ is symmetric. A complex matrix $A$ is unitarily diagonalizable if and only if $A$ is Hermitian.\end{Thm}

The method of computing an orthogonal diagonalization is the same as any other diagonalization except we require that our basis of eigenvectors be orthonormal. Normalizing the eigenvectors is simple enough. On the other hand, we cannot simply apply the Gram-Schmidt procedure to an eigenbasis, because the end result may not be an eigenbasis. Instead, we must apply the Gram-Schmidt procedure to a basis for each distinct eigenspace. Since different eigenspaces are mutually orthogonal, the union of these orthogonal bases gives an orthogonal eigenbasis.\\

\begin{Exam} Let $A = \mtx{rrr}{3&-2&4\\-2&6&2\\4&2&3}$. It can be shown that $\lambda = 7, -2$ are the eigenvalues of $A$ and 
\[\nul(A-7I) = \Span\left\{\vr{1\\0\\1}, \vr{-1/2\\1\\0}\right\},\qquad \nul(A+2I) =\Span\left\{\vr{-1\\-1/2\\1}\right\}\] are the eigenspaces. Applying Gram-Schmidt to the eigenspace of $\lambda =7$, we get the orthogonal basis 
\[\nul(A-7I) = \Span\left\{\vr{1\\0\\1}, \vr{-1/4\\1\\1/4}\right\}.\] Therefore, \[\left\{\vr{1\\0\\1}, \vr{-1/4\\1\\1/4}, \vr{-1\\-1/2\\1}\right\}\] is an orthogonal eigenbasis. After normalizing, the set \[\left\{\vr{1/\sqrt{2}\\0\\1/\sqrt{2}}, \vr{-1/\sqrt{18}\\4/\sqrt{18}\\1/\sqrt{18}}, \vr{-2/3\\-1/3\\2/3}\right\}\] is an orthonormal eigenbasis. Therefore, 
\[A= \mtx{ccc}{1/\sqrt{2}&-1/\sqrt{18} &-2/3  \\0 &4/\sqrt{18} &-1/3 \\1/\sqrt{2} &1/\sqrt{18} &2/3}\mtx{ccc}{7&0&0\\0&7&0\\0&0&-2}\mtx{ccc}{1/\sqrt{2}&0&1/\sqrt{2}\\ -1/\sqrt{18}&4/\sqrt{18}&1/\sqrt{18}\\ -2/3&-1/3&2/3}\] is an orthogonal diagonalization of $A$.
\end{Exam}

In particular, unitary matrices are the complex analogue of orthogonal matrices. In fact, all of the previous theorems about orthogonal matrices remain true when considering unitary matrices. The same can be said for symmetric and Hermitian matrices. \\

The set of eigenvalues of a matrix $A$ is called the \textbf{spectrum} of $A$.\\% A \textbf{spectral theorem} is a theorem which describes the eigenvalues of some matrix.\\

\begin{Thm}[The Spectral Theorem for Symmetric (Hermitian) Matrices] An $n\times n$ symmetric (Hermitian) matrix $A$ has the following properties:
\begin{enumerate}[!THM!, start=1]
\item $A$ has $n$ real eigenvalues, counting multiplicities;
\item The geometric and algebraic multiplicities of each eigenvalue of $A$ are equal;
\item The eigenspaces of $A$ are mutually orthogonal;
\item $A$ is orthogonally/unitarily diagonalizable.
\end{enumerate}
\end{Thm}
%\begin{proof} The first part is all that remains to be proven. Let $A$ be a Hermitian matrix with $A\bb x = \lambda \bb x$. Since 
%\[\bb x^*A\bb x = \bb x^*(A\bb x) = \bb x^*(\lambda \bb x)= \lambda (\bb x^*\bb x) = \lambda(\bb x\cdot \bb x) = \lambda\Vert\bb x\Vert^2.\] Thus, $\lambda = \dfrac{\bb x^*A\bb x}{\Vert \bb x\Vert^2}$. Since $\Vert \bb x\Vert^2$ is necessarily real, we need to show that $\bb x^*A\bb x$ is also real. This can be shown by proving $\overline{\bb x^*A\bb x} = \bb x^*A\bb x$. Of course, $\bb x^*A\bb x\in \C$, and so this is equivalent to proving $(\bb x^*A\bb x)^* = \bb x^*A\bb x$. Thus, 
%\[(\bb x^*A\bb x)^* = \bb x^*A^*(\bb x^*)^* = \bb x^*A^*\bb x = \bb x^*A\bb x,\] since $A^*=A$.
%\end{proof}
\vs


\begin{Exam}\label{uni} Diagonalize the Hermitian matrix $A = \mtx{cc}{2 & 1+i \\ 1-i & 3}$.\\

We begin with its characteristic polynomial: 
\begin{eqnarray*}\det(\lambda I - A) &=& \mtx{cc}{\lambda-2 & -1-i \\ -1+i & \lambda-3} = (\lambda -2)(\lambda -3) - (-1-i)(-1+i)\\
&=& \lambda^2-5\lambda + 6 - 1-1 = \lambda^2-5\lambda +4 = (\lambda-4)(\lambda -1).
\end{eqnarray*}
 Therefore, the eigenvalues of $A$ are $\lambda = 1,\ 4$. We then investigate its eigenspaces:
 \begin{multicols}{2}
 $\lambda = 1;\quad \mtx{cc}{-1 & -1-i \\ -1+i & -2} \sim \mtx{cc}{1 & 1+i \\ 0 & 0 }$\\
 $\bb v_1 = \mtx{c}{-1-i \\ 1};\quad \bb p_1 = \dfrac{\bb v_1}{\Vert\bb v_1\Vert} = \mtx{c}{\frac{-1-i}{\sqrt{3}} \\ \frac{1}{\sqrt{3}}}$\\

     $\lambda = 4;\quad \mtx{cc}{2 & -1-i \\ -1+i & 1} \sim \mtx{cc}{2 & -1-i \\ 0 & 0 }$\\
    $\bb v_2 = \mtx{c}{\frac{1}{2}(1+i) \\ 1};\quad \bb p_2 = \dfrac{\bb v_2}{\Vert\bb v_2\Vert} = \mtx{c}{\frac{1+i}{\sqrt{6}} \\ \frac{2}{\sqrt{6}}}$\\
\end{multicols}

Note that:
\begin{eqnarray*}
\mtx{cc}{2 & 1+i \\ 1-i & 3}\mtx{c}{-1-i \\ 1} &=& \mtx{c}{-2-2i+1+i \\ -2+3} = \mtx{c}{-1-i \\ 1}\\
\mtx{cc}{2 & 1+i \\ 1-i & 3}\mtx{c}{\frac{1}{2}(1+i) \\ 1} &=& \mtx{c}{1+i + 1+i  \\ 1 +3} = 4\mtx{c}{\frac{1}{2}(1+i) \\ 1}
\end{eqnarray*}
Let $P = \mtx{cc}{\frac{-1-i}{\sqrt{3}} & \frac{1+i}{\sqrt{6}} \\ \frac{1}{\sqrt{3}} & \frac{2}{\sqrt{6}}}$, which is a unitary matrix. Let $D = \mtx{cc}{1 & 0 \\ 0 & 4}$. Then 
\[A = PDP^* = \mtx{cc}{\frac{-1-i}{\sqrt{3}} & \frac{1+i}{\sqrt{6}} \\ \frac{1}{\sqrt{3}} & \frac{2}{\sqrt{6}}}\mtx{cc}{1 & 0 \\ 0 & 4}\mtx{cc}{\frac{-1+i}{\sqrt{3}} & \frac{1}{\sqrt{3}} \\ \frac{1-i}{\sqrt{6}} & \frac{2}{\sqrt{6}}} = \mtx{cc}{2 & 1+i \\ 1-i & 3}.\]
As can be expected, the eigenvalues of $A$ are real despite $A$ being a non-real matrix.
\end{Exam}\vs

Let $A$ be a symmetric matrix. Then it has an orthogonal diagonalization given as:
\begin{eqnarray*}
A &=& PDP^\top  = \mtx{ccc}{\bb u_1 & \ldots & \bb u_n}\mtx{ccccc}{\lambda_1 &  & & & 0\\& & \ddots & & \\0 &  & & & \lambda_n}\mtx{c}{\bb u_1^\top \\\vdots \\ \bb u_n^\top } = \mtx{ccc}{\lambda_1\bb u_1 & \ldots & \lambda_n\bb u_n}\mtx{c}{\bb u_1^\top \\\vdots \\ \bb u_n^\top }\\
&=& \lambda_1\bb u_1\bb u_1^\top  + \ldots + \lambda_n\bb u_n\bb u_n^\top  = \lambda_1(\bb u_1 \otimes \bb u_1) + \ldots + \lambda_n(\bb u_n \otimes \bb u_n).
\end{eqnarray*} This last line is called a \textbf{spectral decomposition} of $A$. Each of matrices $B_i = \bb u_i\bb u_i^\top  = \bb u_i \otimes \bb u_i$, the outer product of $\bb u_i$ with itself,  is an $n\times n$ symmetric matrix with rank 1. The range of $B_i$ is $\Span\{\bb u_i\}$. Furthermore, $B_iB_j = 0$ if $i\neq j$ and $B_i^2 = B_i$, since $\{\bb u_1, \ldots, \bb u_n\}$ is a orthonormal set. Thus, the $B_i$'s  are idempotent and pairwise ``orthogonal.'' When considering complex vectors, the outer product becomes $\bb u \otimes \bb v = \bb u \bb v^*$.\\

\begin{Exam} The matrix $A = \mtx{cc}{7&2\\2&4}$ is symmetric and has an orthogonal diagonalization given by
\[A = \mtx{cc}{2/\sqrt{5} & -1/\sqrt{5} \\ 1/\sqrt{5} & 2/\sqrt{5}}\mtx{cc}{8&0\\0&3}\mtx{cc}{2/\sqrt{5} & 1/\sqrt{5} \\ -1/\sqrt{5} & 2/\sqrt{5}}.\] Let $P = \mtx{cc}{\bb u_1 & \bb u_2} = \mtx{cc}{2/\sqrt{5} & -1/\sqrt{5} \\ 1/\sqrt{5} & 2/\sqrt{5}}$. Then 
\begin{eqnarray*}
\bb u_1 \otimes \bb u_1 = \bb u_1\bb u_1^\top  &=& \mtx{c}{2/\sqrt{5}\\1/\sqrt{5}}\mtx{cc}{2/\sqrt{5} & 1/\sqrt{5}} = \mtx{cc}{4/5 & 2/5 \\ 2/5 & 1/5},\\
\bb u_2 \otimes \bb u_2 =  \bb u_2\bb u_2^\top  &=& \mtx{c}{-1/\sqrt{5}\\2/\sqrt{5}}\mtx{cc}{-1/\sqrt{5} & 2/\sqrt{5}} = \mtx{cc}{1/5 & -2/5 \\ -2/5 & 4/5}\\
8\bb u_1\bb u_1^\top  + 3\bb u_2\bb u_2^\top  &=& 8\mtx{cc}{4/5 & 2/5 \\ 2/5 & 1/5} + 3\mtx{cc}{1/5 & -2/5 \\ -2/5 & 4/5} = \mtx{cc}{7&2\\2&4} = A.\qedhere
\end{eqnarray*}
\end{Exam}\vs

%%%%%%%%%%%%%%%%%% Exercises %%%%%%%%%%%%%%%%%%%
\startExercises{orthodiagonable}

\noindent For Exercises \ref{exer:orthodiagonalizationstart}-\ref{exer:orthodiagonalizationstop}, compute an orthogonal (or unitary) diagonalization for the matrix $A$. 
\begin{enumerate}[!HW!, start=1, label=$\spadesuit$ \arabic*., ref=\arabic*]
\begin{multicols}{3}
\item\label{exer:orthodiagonalizationstart} $\mtx{rr}{17&5\\5&-7}$ %new
\item\label{exer:orthodiagonalizationsecond} $\mtx{rrr}{5&17&8\\17&5&8\\8&8&14}$ %new
\item\label{exer:orthodiagonalizationthird} $\mtx{rrrr}{5&3&-3&3\\3&5&3&-3\\-3&3&5&3\\3&-3&3&5}$ %new
\end{multicols}
\begin{multicols}{2}
\item $\mtx{cc}{4 & 1-i \\ 1+i & 5}$ %Anton 7.5.13 p.443
\item\label{exer:orthodiagonalizationstop} $\mtx{ccc}{5 & 0 & 0\\ 0 & -1 & -1+ i \\ 0 & -1-i & 0}$ %Anton 7.5.17 p.443
\end{multicols}
\end{enumerate}

\noindent For Exercises \ref{exer:spectraldecompstart}-\ref{exer:spectraldecompstop}, compute a spectral decomposition for the matrix $A$. 
\begin{enumerate}[!HW!, label=$\spadesuit$ \arabic*., ref=\arabic*]
\begin{multicols}{3}
\item\label{exer:spectraldecompstart} $A$ from Exercise \ref{exer:orthodiagonalizationstart} %new
\item $A$ from Exercise \ref{exer:orthodiagonalizationsecond} %new
\item\label{exer:spectraldecompstop} $A$ from Exercise \ref{exer:orthodiagonalizationthird} %new
\end{multicols}
\end{enumerate}

\begin{enumerate}[!HW!]
\item Show that the inverse of an orthogonal matrix is orthogonal. \\ %new
\item Show that the product of two orthogonal matrices is orthogonal.\\ %new
\item Show that the determinant of an orthogonal matrix is $\pm 1$.\\ %new
\item Show that if $U$ is an orthogonal matrix then $\Vert U\bb x\Vert = \Vert \bb x\Vert$. %new
\end{enumerate}


%%%%%%%%%%%%%%%%%%% Footnotes %%%%%%%%%%%%%%%%%%%
 \mbox{}\vfill
 
\pagebreak